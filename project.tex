%%% Relativistic quantum fields notes.

\documentclass{momento}

\usepackage{danielphysics}
\usepackage{tensor}
%\usepackage{breqn}
\usepackage{mathtools}

\usepackage{simplewick, slashed}
\usepackage{tikz-feyn}
\title{Relativistic Quantum Fields}
\author{Daniel Williams}

\providecommand{\Lag}{\mathcal{L}}       %The Lagrangian density
\providecommand{\Ham}{\mathcal{H}}       %The Hamiltonian density
\providecommand{\Op}[1]{{\widehat{#1}}}  %A Quantum mechanical operator
\providecommand{\hcon}[1]{#1^{\dagger}}  %The Hermitian conjugate
\providecommand{\hOp}[1]{\hcon{\Op{#1}}}
\providecommand{\normbracket}[1]{\raisebox{0.1em}{:}  {#1}  \raisebox{.1em}{:}}
\providecommand{\nOp}[1]{\raisebox{0.1em}{:}  \Op{#1}  \raisebox{.1em}{:}}
\providecommand{\nhOp}[1]{\raisebox{0.1em}{:}  \Op{#1}  \raisebox{.1em}{:}}
\providecommand{\nm}[1]{@(#1)\ }
\providecommand{\pd}[1]{\partial_{#1}}
\providecommand{\pu}[1]{\partial^{#1}}
\providecommand{\wicon}[2]{\Op{#1}^{\bullet} \Op{#2}^{\bullet}}
\providecommand{\tOrd}{\mathcal{T}}
\providecommand{\ps}{\slashed{p}}
\providecommand{\ds}{\slashed{\partial}}
\allowdisplaybreaks
\begin{document}
\frontmatter
{
\thispagestyle{empty}
\begin{tikzpicture}[remember picture,overlay]
  \fill[fill=accent-blue] (current page.south west) rectangle (current page.north east);
  %\fill[fill=white, yshift=-10cm]  (current page.north east) rectangle (current page.north west);
  \def\nbrcircles {377}
  \def\outerradius {30mm}
  \def\deviation {.9}
  \def\fudge {.62}

  \newcounter{cumulArea}
  \setcounter{cumulArea}{0}

  \pgfmathsetmacro {\goldenRatio} {(1+sqrt(5))}
  \pgfmathsetmacro {\meanArea} {pow(\outerradius * 10 / \nbrcircles, 2) * pi}
  \pgfmathsetmacro {\minArea} {\meanArea * (1 - \deviation)}
  \pgfmathsetmacro {\midArea} {\meanArea * (1 + \deviation) - \minArea}

  \foreach \b in {0,...,\nbrcircles}{
    % mod() must be used in order to calculate the right angle.
    % otherwise, when \b is greater than 28 the angle is greater
    % than 16384 and an error is raised ('Dimension too large').
    % -- thx Tonio for this one.
    \pgfmathsetmacro{\angle}{mod(\goldenRatio * \b, 2) * 180}

    \pgfmathsetmacro{\sratio}{\b / \nbrcircles}
    \pgfmathsetmacro{\smArea}{\minArea + \sratio * \midArea}
    \pgfmathsetmacro{\smRadius}{sqrt(\smArea / pi) / 2 * \fudge}
    \addtocounter{cumulArea}{\smArea};

    \pgfmathparse{sqrt(\value{cumulArea} / pi) / 2}
    \fill[opacity=0.3] (\angle:\pgfmathresult) circle [radius=\smRadius] ;
}  

  \node at (current page.center) [text width=16cm, yshift=10cm] 
    {\color{white}\fontsize{72pt}{106pt}\center \selectfont\sffamily Relativistic Quantum Fields};

  \node at (current page.south) [text width= \textwidth, yshift=5cm] 
    {\color{white}\fontsize{32pt}{120pt}\selectfont \sffamily Daniel Williams};

\end{tikzpicture}
}
\newpage
\maketitle

\tableofcontents

\mainmatter
\chapter{Classical Field Theory}
\label{cha:class-field-theory}

%% Classical Fields

A physical field is normally governed by two restrictions; a
differential equation, and boundary conditions, but they can be
described using the Lagrangian and the principle of least action.

\begin{definition}[Action]
  Action is the quantity
  \[ S = \int \Lag(\phi, \partial_\mu \phi) \dd[4]{x} \]
  for $\Lag$ the Lagrangian, itegrated over all space-time.
\end{definition}

A field adopts a configuration to reduce the action to a minimum. To
find this we look for the infinitessimal variations which leave the
action unchanged, i.e.
\begin{equation}
  \label{eq:1}
  \phi(x) \to \phi(x) + \delta \phi(x) \quad | \quad S \to S + \delta S 
\end{equation}
with $\delta S = 0$.

Then
\begin{align*}
  \delta S &= 
     \delta \phi \pdv{\phi} \int \Lag(\phi, \partial_{\nu} \phi) \dd[4]{x} \\ 
     &\quad+ \delta(\partial_{\mu} \phi ) \pdv{(\partial \mu
        \phi)} \int \Lag(\phi, \partial_{\nu} \phi) \dd[4]{x} \\
     &= \int \qty( \delta \phi \pdv{\Lag}{\phi} + \partial_{\mu} (\delta
        \phi) \pdv{\Lag}{(\partial_{\mu} \phi)} ) \dd[4]{x}
\end{align*}
since, for a well-behaved field we have $\delta (\partial_{\mu} \phi)
= \partial_{\mu}(\delta \phi)$, by the mixed-derivatives theorem.
Now, 
\begin{equation}
  \label{eq:2}
  \partial_{\mu} \qty( \delta \phi \pdv{\Lag}{(\partial_{\mu} \phi)} ) = 
  \partial_{\mu} (\delta \phi) \pdv{\Lag}{(\partial_{\mu} \phi)} + \delta \phi \partial_{\mu} \qty( \pdv{\Lag}{(\partial_{\mu} \phi)} )
\end{equation}
which, given that a total derivative evaluated across all of space
will vanish, since it is equivalent to being evaluated at either end
of the region, and so
\begin{equation}
  \label{eq:3}
  \delta S = \int \qty[ \delta \phi \pdv{\Lag}{\phi} + \partial_{\mu} \qty(\delta \phi \pdv{\Lag}{(\partial_{\mu} \phi)}) - \delta \phi \partial_{\mu} \qty( \pdv{\Lag}{(\partial_{\mu}\phi)})] \dd[4]{x}
\end{equation}

This is true for all $\delta \phi$, and so we can get the
Euler-Lagrange equations,
\begin{fequation}[Euler-Lagrange Equations]
  \label{eq:euler-lagrange}
%\begin{equation}
  \pdv{\Lag}{\phi} - \partial_{\mu} \qty( \pdv{\Lag}{(\partial_{\mu} \phi )}) = 0 
%\end{equation}
\end{fequation}

\begin{example}[Using the Euler-Lagrange Equations]
  Consider a Lagrangian
  \[ \Lag = \half \partial_{\mu} \phi \partial^{\mu} \phi \] the terms
  the terms for the Lagrangian are then
  \[ \pdv{\Lag}{\phi} = 0 \] and \[ \pdv{\Lag}{(\partial_{\mu} \phi)}
  = \partial^{\mu} \phi \] Which gives the wave equation:
  \[ \partial_{\mu} \partial^{\mu} \phi = \qty( \pdv[2]{t} -
  \vec{\nabla}^2 ) \phi = 0 \]
\end{example}

\section{Noether's Theorem}
\label{sec:noethers-theorem}

\begin{theorem}[Noether's Theorem]
  If action is unchanged under a transformation then there exists a
  conserved current which is associated with the symmetry.
\end{theorem}

Consider the infinitessimal transformation of both coordinates and the
fields,
\begin{align*}
  x^{\mu} \to x^{\prime \mu} &= x^{\mu} + \delta x^{\mu} = x^{\mu} + \tensor{X}{^\mu_\nu} \omega^{\nu} 
  \\ \phi \to \phi^{\prime} &= \phi + \delta \phi = \phi + \tensor{\Phi}{_{\nu}} \omega^{\nu}
\end{align*}
which are parametrised by an infintessimal parameter $\omega^{\nu}$.

The change in the field can be due to both a change in the function
defining the field, $\delta_{\phi} \phi$, and in the coordinates,
$\partial_{\mu} \phi \delta x^{\mu}$. The change in the Lagrangian is
then
\begin{align*}
  \delta \Lag &= \partial_{\mu} \Lag \delta x^{\mu} + \pdv{\Lag}{\phi} \delta_{\phi}\phi + \pdv{\Lag}{(\partial_{\nu} \phi)} \delta_{\phi} (\partial_{\nu} \phi) \\
&= \partial_{\mu} \Lag \delta x^{\mu} + \partial_{\nu} \qty( \pdv{\Lag}{(\partial_{\nu} \phi)}) \delta_{\phi} \phi + \pdv{\Lag}{(\partial_{\nu} \phi)} \partial_{\nu} (\delta_{\phi} \phi ) \\ 
&= \partial_{\mu} \Lag \delta x^{\mu} + \partial_{\nu} \qty( \pdv{\Lag}{(\partial_{\nu} \phi )} \delta_{\phi} \phi )
\end{align*}
This is the change in the Lagrangian, the change in the action is 
\begin{equation}
  \label{eq:4}
  \delta S = \delta \qty( \int \dd[4]{x} \Lag )
\end{equation}
the integration measure changes, and requires a Jacobian:
\[ \dd[4]{x^{\prime}} = \qty| \pdv{x^{\prime}}{x} | \dd[4]{x} =
(1+ \partial_{\mu} \delta x^{\mu} ) \dd[4]{x} \]
so
\begin{align*}
  \delta S &= \int \dd[4]{x} (\delta \Lag + \Lag \partial_{\mu} \delta x^{\mu} \\
  &= \int \dd[4]{x} \qty( \partial_{\mu} \Lag \delta x^{\mu}
  + \partial_{\nu} \qty( \pdv{\Lag}{(\partial_{\nu}
    \phi)} \partial_{\phi} \phi ) + \Lag \partial_{\mu} \delta x^{\mu}
  ) \\ &= \int \dd[4]{x} \partial_{\mu} \qty( \pdv{\Lag}{(\partial_{\mu} \phi)} \delta_{\phi} \phi + \Lag \delta x^{\mu} )
\end{align*}
Now writing $\delta x^{\mu} = \tensor{X}{^{\mu}_{\nu}} \omega^{\nu}$, and $\delta \phi = \tensor{\Phi}{_{\nu}}\omega^{\nu}$, then
\[ \delta_{\phi} \phi = \delta \phi - \partial_{\mu} \phi - \delta x^{\mu} = (\Phi_{\nu} - \partial_{\mu} \phi \tensor{X}{^{\mu}_{\nu}} ) \omega^{\nu} \]
which can be written in terms of the divergence of a current,
\[ \delta S = - \int \dd[4]{x} \partial_{\mu} \tensor{j}{^{\mu}_{\nu}}
\omega^{\nu} \]
where
\begin{equation}
  \label{eq:5}
  \tensor{j}{^{\mu}_{\nu}} = - \pdv{\Lag}{(\partial_{\mu} \phi)} 
                          ( \tensor{\Phi}{_{\nu}} - \partial_{\rho} \phi
                            \tensor{X}{^{\rho}_{\nu}}) - \Lag \tensor{X}{^{\mu}_{\nu}}
\end{equation}
which can be rearranged to give
\begin{fequation}
  \begin{equation}
\label{eq:6}
 \tensor{j}{^{\mu}_{\nu}} = \qty( \pdv{\Lag}{(\partial_{\mu} \phi) } \partial_{\rho} \phi - \tensor{g}{^{\mu}_{\rho}} \Lag ) \tensor{X}{^{\rho}_{\nu}} - \pdv{\Lag}{(\partial_{\mu} \phi)} \Phi_{\nu} 
\end{equation}
\end{fequation}
To ensure that the action is invariant under this transformation this current must be conserved, i.e. 
\begin{fequation}
  \begin{equation}
    \label{eq:7}
    \partial_{\mu} \tensor{j}{^{\mu}_{\nu}} = 0
  \end{equation}
\end{fequation}
which is Noether's Theorem.

\begin{example}[The energy-momentum tensor.]
  Consider space-time translations:
  \begin{align}
    x^{\mu} \to x^{\prime \mu} = x^{\mu} + \omega^{\nu} & \implies \tensor{X}{^{\mu}_{\nu}} = \tensor{g}{^{\mu}_{\nu}} \\
\phi \to \phi^{\prime} = \phi & \implies \Phi_{\nu} = 0
  \end{align}
  Substituting these into the expression for conserved current we have
    \begin{equation}
      \label{eq:8}
      \tensor{T}{^{\mu}^{\nu}} = \pdv{\Lag}{(\partial_{\mu} \phi)} \partial_{\nu} \phi - \tensor{g}{^{\mu}^{\nu}} \Lag
    \end{equation}
    which is central to General relativity. There are a number of
    physical consequences of this tensor's existence. Consider a part
    of the tensor,
    \[ \tensor{J}{^{\mu}} = \tensor{T}{^{0}^{\mu}} =
    \pdv{\Lag}{(\partial_0 \phi)} \partial^{\mu}\phi -
    \tensor{g}{^0^{\mu}} \Lag \] and then integrate its divergence
    over a three-dimensional volume $V$,
    \[ \int_V \partial_{\mu} J^{\mu} \dd{V} = \int_V \qty(
    \pdv{J^0}{t} + \vec{\nabla} \cdot \vec{J} ) \dd{V} \]

    By Noether's theorem we know the divergence is zero, so, applying
    Gauss's Theorem,
    \[ \int_V \pdv{J^0}{t} \dd{V} = - \int_V \vec{\nabla} \cdot
    \vec{J} \dd{V} = - \int_A \vec{J} \cdot \dd{\vec{A}} \] Thus any
    change in the total $J^0$ in the volume must be due to a current
    $J$ flowing through the surface of the volume.

The conserved quantity associated with time transformations is the Hamiltonian,
\begin{equation}
  \label{eq:9}
  H = \int \tensor{T}{^0^0} \dd[3]{x}
\end{equation}
while the quantity associated with space transformations is the three momentum operator
\begin{equation}
  \label{eq:10}
  P^i = \int \tensor{T}{^0^i} \dd[3]{x}
\end{equation}
\end{example}

%%% Local Variables: 
%%% mode: latex
%%% TeX-master: "../project"
%%% End: 



\chapter{Free Scalar Field Theory}
\label{cha:free-scalar-field}

  A scalar field associates a scalar with every point in a physical
  space. The scalar may be a number or a physical quantity, and must
  be coordinate invariant.

  % \begin{illustration}
  %   Some examples of free scalar fields are temperature distributions
  %   in space, and the Higgs field.
  % \end{illustration}

  Scalar field theory constitutes the simplest possible field theory,
  with a Lagrangian
  \begin{equation}
    \label{eq:15}
    \Lag = \half \partial_{\mu} \phi \partial^{\mu} \phi - \half m^2 \phi^2
  \end{equation}


\begin{derivation}
  The components of the Euler-Lagrange equations from the Lagrangian
  in equation \eqref{eq:15} are
  \begin{subequations}
    \begin{align}
      \pdv{\Lag}{\phi} &= -m^2 \phi \\
      \pdv{\Lag}{(\partial_{\mu} \phi)} &= \partial^{\mu} \phi \\
      \text{so \quad} \partial_{\mu} \qty(\pdv{\Lag}{(\partial_{\mu}
        \phi)}) &= \partial_{\mu} \partial^{\mu} \phi
    \end{align}
  \end{subequations}
\end{derivation}

thus the Euler-Lagrange equation for this Lagrangian is a wave
equation, specifically, the Klein-Gordon equation.


\begin{fequation}[Klein-Gordon Equation]
  \label{eq:kleingordonscalar}
  (\partial^2 + m^2) \phi = 0
\end{fequation}
This is the relativistic relation between energy and momentum---a
relativistic Schr\"odinger equation, and, when first postulated (in
fact, by Schr\"odinger) caused confusion due to the apparent negative
energy states in its solution.

% \begin{illustration} We can see that the Klein-Gordon equation is a
%   relativistic statement of Schr\"odinger's equation by considering
%   the equation's form in non-covariant notation.
%   \begin{equation*}
%     \label{eq:16} \qty( - \pdv[2]{t} + \nabla^2 ) \phi = m^2 \phi
%   \end{equation*}
% \end{illustration}

\section{On-shell and normalisation}
\label{sec:norm-meas}

We need to obey the on-shell condition, so
\begin{equation}
  \label{eq:49}
  p^2_\mu = E^2 -\vec{p}^2 = m^2 \implies E = \pm \sqrt{\vec{p}^2 + m^2}
\end{equation}
and, in order to maintain normalisation we need to introduce a
normalisation factor into any integration. This has the form
\[ \nm{\vec{k}} = \frac{\dd[3]{k}}{(2 \pi)^3 2 E(\vec{k})} \] 
while being an odd-looking quantity is the covariant choice
for normalisation. 

\begin{derivation}
  We can show that this is the covariant choice by considering a
  function $f(x)$, and then integrating it,

\begin{align*}
  \int \nm{\vec{k}} f(k) &= \int  \nm{\vec{k}} \dd{E^2} \delta(E^2 - \vec{k}^2 - m^2) f(k) \\ &
=  \int  \nm{\vec{k}} 2 E\dd{E} \delta(E^2 - \vec{k}^2 - m^2) f(k) \\
&=  \int  \nm{\vec{k}} 2 \pi\delta(E^2 - \vec{k}^2 - m^2) f(k) 
\end{align*}
where the $\delta(E^2 - \vec{k}^2 - m^2)$ condition represents the
requirement for the state to be located ``on-shell''.
\end{derivation}

\section{Plane-wave solutions of Klein-Gordon}
\label{sec:plane-wave-solutions}

By taking a plane wave solution we have the general form of the
potential being
  \begin{equation}
    \phi(x) = \int \nm{\vec{k}} \left(
      a(\vec{k}) \exp(- i k \vdot x) + a^{*}(\vec{k}) \exp(i k \vdot
      x) \right)
  \end{equation}
  Taking a plane wave solution of equation \ref{eq:kleingordonscalar},
  \[ \phi \propto \exp(i k \vdot x) = \exp( i (k^0 t - \vec{k} \vdot
  \vec{x})) \]

\begin{derivation}
  \[ (k^0)^2 - \vec{k}^2 = m^2 \Longleftrightarrow    k^0 = \pm \sqrt{k^2 + m^2} \]
  where both the positive and the negative solutions are
  valid. Letting $E(k)= + \sqrt{k^2 + m^2}$, we have a general form
  for the potential:
  \begin{equation}
    \label{eq:18}
    \phi(x) = \int \nm{\vec{k}} \left(
      a(-\vec{k}) e^{i\qty(E(\vec{k}) t - \vec{k} \vdot \vec{x})} +
      a^{*}(\vec{k}) e^{i\qty(E(\vec{k}) t - \vec{k} \vdot \vec{x})}
    \right)
  \end{equation}
  letting $\vec{k} \to - \vec{k}$ in the first integral,
  \begin{equation}
    \phi(x) = \int \nm{\vec{k}} \left(
      a(\vec{k}) \exp(- i k \vdot x) + a^{*}(\vec{k}) \exp(i k \vdot
      x) \right)
  \end{equation}
 We can make this look simpler by defining
\begin{equation}
  \label{eq:50}
  e_k(x) = \nm{\vec{k}} \exp(- ik \vdot x) 
\end{equation}
Giving us
\begin{equation}
  \label{eq:51}
  \phi(x) = \int \qty( a(\vec{k}) e_k(x) + a^{*}(\vec{k}) e_k^{*}(x) )
\end{equation}

\end{derivation}



\section{The Energy Function and Hamiltonian}
\label{sec:energy-moment-tens}

The energy momentum tensor is

\newcommand{\metricten}{\ten{g}{^{\mu}^{\nu}}}

\begin{align*}
  \label{eq:19}
  \ten{T}{^{\mu}^{\nu}} &= \pdv{\Lag}{(\partial_{\mu}\phi)} \partial^{\nu} \phi - \metricten \Lag \\
&= \partial^{\mu} \phi \partial^{\nu} \phi - \metricten \qty(\half \partial_{\rho} \phi \partial^{\rho} \phi - \half m^2 \phi )
\end{align*}

Then

\begin{equation}
  \label{eq:20}
  \ten{T}{^{00}} = \partial^0 \phi \partial^0 \phi - 
     \half \qty( \partial_0 \phi \partial^0 \phi 
               - \nabla \phi \vdot \nabla \phi
               - m^2 \phi  )
\end{equation}
So the Hamiltonian is
\begin{align}
  \label{eq:21}
  H &= \int \ten{T}{^{00}} \dd[3]{x} \nonumber\\
 &= \half \int \qty( (\partial_0 \phi)^2 + (\nabla \phi)^2 + m^2\phi^2 ) \dd[3]{x}
\end{align}
This works for a classical theory, but in quantum-mechanical theories
we require $H$ to be an operator, however, we can't simply convert it
into an operator, we also need to turn the fields into operators.

\section{Second Quantisation}
\label{sec:second-quantisation}

The second quantisation is the process of turning fields into
operators, as opposed to the approach of the first quantisation where
observables are made into operators.

In order to do this we define the canonically conjugate momentum
  \begin{equation} 
\pi(x) = \pdv{\Lag}{(\partial_0 \phi)} = \partial_0 \phi(x)
  \end{equation}
  
So we can postulate that $\phi$ and $\pi$ are operators which satisfy the equal-time commutation relations
\begin{subequations}
\begin{align}
  \comm{\Op{\phi}(\vec{x}, t)}{\Op{\pi} (\vec{x}, t)} &= i \delta^3(\vec{x} - \vec{y}) \\
\comm{\Op{\phi}(\vec{x}, t)}{\Op{\phi}(\vec{y}, t)} &= \comm{\Op{\pi}(\vec{x}, t)}{\Op{\pi}(\vec{y}, t)} = 0
\end{align}
\end{subequations}

This doesn't change the Klein-Gordon equation, although it now acts
like an operator, so
\begin{equation}
  \label{eq:22}
  \Op{\phi}(\vec{x}) = \int \qty(\Op{a}(\vec{k}) e_k + \hcon{\Op{a}}(\vec{k}) e^{*}_{k})
\end{equation}

where the quantities $\Op{a}$ and $\hcon{\Op{a}}$ are now operators. The corresponding equation for $\pi$ is

\begin{equation}
  \label{eq:23}
  \Op{\pi} (x) = \partial_0 \Op{\phi}(x) = 2i \int E(k) \qty( - \Op{a} (\vec{k}) e_k + \hOp{a}(\vec{k}) e^{*}_k )
\end{equation}

\section{Creation and Annihilation Operators}
\label{sec:creat-annih-oper}

\begin{derivation}
  We can find the coefficients $\Op{a}$ and $\hOp{a}$ can be found
  through an inverse Fourier transform.  For $\Op{\phi}$,
\begin{align*}
\int & \dd[3]{x}  \ \Op{\phi} \ e^{-ik \vdot x} \\ 
     & =  \int \nm{k'}   \bigg( \Op{a}(\vec{k}') \int \dd[3]{x} e^{-i (k+k') \vdot x}   \\ 
     & \qquad \qquad \quad+ \hOp{a}(\vec{k}') \int \dd[3]{x} e^{-i(k'-k) \vdot x} \bigg) \\[1em] 
     & =  \int \frac{\dd[3]{k'}}{2 E(\vec{k})} \bigg(\Op{a}(\vec{k}') \delta^3(\vec{k} + \vec{k}') e^{-i (E(\vec{k}) - E(\vec{k}'))t}  \\
     & \qquad \qquad \quad+  \hOp{a}(\vec{k}') \delta^3(\vec{k}' - \vec{k}) e^{i(E(\vec{k}') - E(\vec{k}))t} \bigg) \\ 
     & = \frac{1}{2 E(\vec{k})} \Big(\Op{a}(-\vec{k}) e ^{-i2E(\vec{k})t} + \hOp{a}(\vec{k}) \Big)
\end{align*}
and for $\Op{\pi}$,
\begin{equation*}
  \int \dd[3]{x} \Op{\pi}(x) e^{-i k \vdot x} = \frac{i}{2}
   \qty( -\Op{a}(-\vec{k}) e^{-i 2 E(\vec{k})t} + \hOp{a}(\vec{k}) )
\end{equation*}

In each case using the definition of the $\delta$-function, and 
\[ 
   e^{i(k' - k) \vdot x} = e^{-i (\vec{k}' - \vec{k}) \vdot x}
                       e^{i \qty( E(\vec{k}') - E(\vec{k}) ) t}
\]

So we now have
\begin{subequations}
\begin{align}
\label{eq:25}
  \int \dd[3]{x} \qty[ E(\vec{k}) \Op{\phi}(x) - i \Op{\pi}(x) ]
                 e^{-i k \vdot x} &= \hOp{a}(\vec{k}) & \\
\label{eq:26}
  \int \dd[3]{x} \qty[ E(\vec{k}) \Op{\phi}(x) + i \Op{\pi}(x) ]
                 e^{-i k \vdot x} &= \Op{a}(-\vec{k}) e^{-2 i E(\vec{k}) t}
\end{align}
\end{subequations}
Equation (\ref{eq:26}) is in need of some further manipulation; we
split the space and the time components, so
\begin{equation*}
  \int \qty[ E(\vec{k}) \Op{\phi}(x) + i \Op{\pi}(x)] e^{i \vec{k} \vdot \vec{x}} e^{-i E(\vec{k}) t} = a(-k) e^{-2i E(\vec{k} t)} 
\end{equation*}
Then, multiplying by $\exp(2 i E(\vec{k}) t)$,
\[ 
  a(-k) = \int \qty[ E(\vec{k}) \Op{\phi}(x) + i \Op{\pi}(x)] e^{i \vec{k} \vdot \vec{x}} e^{2 i E(\vec{k}) t}
\]
and replacing $\vec{k} \to - \vec{k}$,
\begin{subequations}
  \begin{align}
   \label{eq:28}
    \Op{a}(\vec{k}) &= \int \qty[ E(\vec{k}) \Op{\phi} + i \Op{\pi}(x) ] e^{i k \vdot x} \\
   \label{eq:29}
    \hOp{a}(\vec{k}) &= \int \qty[E(\vec{k}) \Op{\phi} - i \Op{\pi}(x) ] e^{-i k \vdot x}
  \end{align}
\end{subequations}
Which are the annihilation (\ref{eq:28}) and creation (\ref{eq:29}) operators.
\end{derivation}

The commutators of these quantities are

\begin{subequations}
  \begin{equation}
    \label{eq:33}
    \comm{\Op{a}(\vec{k})}{\hOp{a}(\vec{p})} =  (2 \pi)^3 2 E(\vec{k}) \delta^3(\vec{k} - \vec{p})
  \end{equation}
  \begin{equation}
    \label{eq:32}
    \comm{\Op{a}(\vec{k})}{\Op{a}(\vec{p})} = 0
  \end{equation}
  \begin{equation}
\label{eq:52}
    \comm{\hOp{a}(\vec{k})}{\hOp{a}(\vec{p})} = 0
  \end{equation}
\end{subequations}

\begin{bigderiv*}

\begin{derivation}

  \begin{flalign*} 
    \comm{\Op{a}(\vec{k})}{\hOp{a}(\vec{p})} 
    &= \int \dd[3]{x} \int \dd[3]{y} \qty( -i E(\vec{k}) \comm{\Op{\phi}}{\Op{\pi}}
    -i E(\vec{p}) \comm{\Op{\pi}}{\Op{\phi}})
    e^{i(k \vdot x - p \vdot y)} \nonumber \\
    &\quad= \int \dd[3]{x} \int \dd[3]{y} \qty( E(\vec{k} ) \delta^3(\vec{x} - \vec{y}) 
    + E(\vec{p}) \delta^3(\vec{x} - \vec{y}) )
    e^{i (k \vdot x - p \vdot y)} \nonumber\\
    &\quad= \int \dd[3]{x} \qty( E(\vec{k}) + E(\vec{p}) ) e^{i (k-p) \vdot x} \nonumber\\
    &\quad = (2 \pi)^3 2 E(\vec{k}) \delta^3(\vec{k} - \vec{p})
  \end{flalign*}\hfill
\end{derivation}
\begin{derivation}
  \begin{flalign*} 
    \comm{\Op{a}(\vec{k})}{\Op{a}(\vec{p})}
    & = \int \dd[3]{x} \int \dd[3]{y} \qty( -i E(\vec{k}) \comm{\Op{\phi}}{\Op{\pi}}
    -i E(\vec{p}) \comm{\Op{\phi}}{\Op{\pi}})
    e^{i(k \vdot x - p \vdot y)} \nonumber \\
    &\quad = \int \dd[3]{x} \int \dd[3]{y} \qty( E(\vec{k} ) \delta^3(\vec{x} - \vec{y}) 
    - E(\vec{p}) \delta^3(\vec{x} - \vec{y}) )
    e^{i (k \vdot x - p \vdot y)} \nonumber\\
    &\quad = \int \dd[3]{x} \qty( E(\vec{k}) - E(\vec{p}) ) e^{i (k-p) \vdot x} \nonumber\\
    &\quad = (2 \pi)^3 \qty( E(\vec{k}) - E(\vec{p}) ) \delta^3(\vec{k} - \vec{p}) \nonumber\\
    &\quad = 0
  \end{flalign*}
\end{derivation}
  \caption{The derivations of the commutation relations for the
    creation and annihilation operators, equations \eqref{eq:33} to \eqref{eq:52}.}
\end{bigderiv*}

\section{The Energy Operator}
\label{sec:energy-momentum}

Returning to the Hamiltonian, which is a conserved quantity,

\begin{equation}
  \label{eq:24}
  \Op{H} = \half \int \qty( \Op{\pi}^2 +(\nabla \phi)^2+m^2 \Op{\phi}^2)
\end{equation}

and then substituting the operators from above (see derivation \ref{deriv:ham}),

\begin{equation}
  \label{eq:27}
  \Op{H} = \half \int \nm{k} E(\vec{k}) \qty[ \Op{a}(\vec{k}) \hOp{a}(\vec{k}) + \hOp{a}(\vec{k}) \Op{a}(\vec{k})]
\end{equation}
This can be compared to the quantum harmonic oscillator, where
\begin{equation}
  \label{eq:31}
  \Op{H} = \half \hbar \omega ( \Op{a} \hOp{a} + \hOp{a} \Op{a})
\end{equation}

We can then postulate a lowest energy state, $\ket{0}$, the vacuum, such that
\begin{equation}
  \label{eq:34}
  \Op{a}(\vec{k}) \ket{0} = 0
\end{equation}

\providecommand\normmeasure[1]{ \frac{\dd[3]{#1}}{(2 \pi)^3 2 E(\vec{#1})}}

The energy of the vacuum state is then
\begin{align*}
 E_0 &= \ev{\hOp{H}}{0} \\ &= \frac{1}{4} \int \frac{\dd[3]{k}}{(2 \pi)^3} \qty( \ev{\Op{a}(\vec{k}) \hOp{a}(\vec{k})}{0} + \ev{\hOp{a}(\vec{k}) \Op{a}(\vec{k})}{0}) 
\end{align*}
The second term vanises, as $\Op{a}(\vec{k}) \ket{0} = 0$, and the first term is
\begin{align*}
 \ev{\Op{a}(\vec{k}) \hOp{a}(\vec{k})}{0} &=  \ev{(2 \pi)^3 2 E(\vec{k}) \delta^3(\vec{k}-\vec{k}) + \hop{a}(\vec{k})\Op{a}(\vec{k})}{0}\\
&= (2 \pi)^3 2 E(\vec{k}) \delta^3(0)
\end{align*}
and so
\[ E_0 = \half \delta^3(0) \int \dd[3] k E(\vec{k}) = \infty \]

\begin{bigderiv*}[t]
  Since 
\begin{align*}   
\Op{\pi} (x) &=  \frac{i}{2} \int \frac{\dd[3]{k}}{(2 \pi)^3} \qty( - \Op{a} (\vec{k}) e^{-ik \vdot x} + \hOp{a}(\vec{k}) e^{ik \vdot x} ) &
\Op{\phi}(\vec{x}) &= \int \frac{\dd[3]{k}}{(2 \pi)^3 2 E(\vec{k}) }
                      \qty(\Op{a}(\vec{k}) e^{-ik \vdot x} + \hcon{\Op{a}}(\vec{k}) e^{i k \vdot x}) 
\end{align*}
\[                    \nabla \Op{\phi}(\vec{x}) &= i \int \frac{\dd[3]{k}}{(2 \pi)^3 2 E(\vec{k}) }
                     \vec{k} \qty(\Op{a}(\vec{k}) e^{-ik \vdot x} - \hcon{\Op{a}}(\vec{k}) e^{i k \vdot x}) \]
then
\begin{align*}
  \Op{H} &= \half \int \bigg( \Op{\pi}^2 + (\nabla \Op{\phi})^2 + m^2 \Op{\phi}^2 \bigg) \dd[3]{x} \\
   &= \half \int \dd[3]{x} \nm{k} \nm{p} 
   \bigg(  (-E(\vec{k})E(\vec{p}) - \vec{k} \vdot \vec{p} +m^2) \Op{a}(\vec{k}) \Op{a}(\vec{k}) e^{-i (k + p) \vdot x} 
   + (-E(\vec{k})E(\vec{p}) - \vec{k} \vdot \vec{p} +m^2) \Op{a}(\vec{k}) \Op{a}(\vec{k}) e^{ i (k + p) \vdot x} \\
   &\quad\qquad+ (E(\vec{k})E(\vec{p}) - \vec{k} \vdot \vec{p} +m^2) \Op{a}(\vec{k}) \Op{a}(\vec{k}) e^{ -i (k - p) \vdot x} 
   + (E(\vec{k})E(\vec{p}) - \vec{k} \vdot \vec{p} +m^2) \Op{a}(\vec{k}) \Op{a}(\vec{k}) e^{i (k - p) \vdot x} \bigg)
\end{align*}
since $E^2 = \vec{k}^2 + m^2$, 
\begin{align*}
  \phantom{\Op{H}}&= \half \int \dd[3]{x} \normmeasure{k}  \frac{\dd[3]{p}}{2E(\vec{p})} 
     \bigg(  (-E(\vec{k}) + \vec{k} + m^2) \Op{a}(\vec{k}) \Op{a}(\vec{k}) \exp(-2i E(\vec{p})t) \delta^3(\vec{k} + \vec{p}) \\
     &\qquad+ (-E(\vec{k}) + \vec{k} + m^2) \Op{a}(\vec{k}) \Op{a}(-\vec{k}) \exp(-2i E(\vec{p})t) \delta^3(\vec{k} + \vec{p}) 
     + (E(\vec{k}) + \vec{k} + m^2) (\Op{a}(\vec{k}) \Op{a}(\vec{k}) + \hOp{a}(\vec{k}) \Op{a}(\vec{k}) ) \delta^3(\vec{k} - \vec{p}) \bigg)
\end{align*}
The first two terms cancel using the relation $E(\vec{k}) = \sqrt{\vec{k}^2 +m^2}$.

\caption{The derivation of the expression for the energy function,
  equation \eqref{eq:31}.}
\label{deriv:ham}
\end{bigderiv*}

This result isn't really surprising, since the field was constructed
as the infinite sum over harmonic oscillators. Given that the ground
state of an oscillator has non-zero energy the vacuum energy will pick
up an infinite number of such terms.

To manage this we introduce the concept of normal ordering.

\begin{illustration}
  An exception (perhaps?) here is the energy-momentum tensor in
  general relativity where we expect the vacuum energy to show up as a
  cosmological constant.
\end{illustration}


\section{Normal Ordering}
\label{sec:normal-ordering}

Normal ordering is the process by which the vacuum energy is
subtracted from all other energies in the theory. In a normally
ordered product all of the annihilation operators are moved to the
right of all of the creation operators, so
\begin{equation}
  \label{eq:30}
  \normbracket{ \Op{a}(\vec{k}) \hOp{a}(\vec{k})} := \hOp{a}(\vec{k}) \Op{a}(\vec{k})
\end{equation}
and so
\begin{align*}
  \nOp{H} &= \half \int \nm{k} E(\vec{k}) \qty( \normbracket{ \Op{a}(\vec{k}) \hOp{a}(\vec{k})} + \normbracket{ \hOp{a}(\vec{k}) \Op{a}(\vec{k})}) \\
&= \int \nm{k} E(\vec{k}) \hOp{a}(\vec{k}) \Op{a}(\vec{k})
\end{align*}
which makes the vacuum energy
\[ E_0 = \ev{\nOp{H}}{0} = 0 \]
and the energy of any state $\ket{\vec{k}} = \hOp{a}(\vec{k}) \ket{0}$ is then
% \begin{equation}
%   \label{eq:17}
%   \nOp{H} \ket{\vec{k}} = E(\vec{k}) \ket{\vec{k}}
% \end{equation}

\begin{derivation}
  \begin{align}
    \nOp{H} \ket{\vec{k}} &= \half \int \frac{\dd[3]{\vec{p}}}{(2\pi)^3}  \hOp{a}(\vec{p}) \Op{a}(\vec{p}) \hOp{a}(\vec{k}) \ket{0} \nonumber\\
&= \half \int \frac{\dd[3]{\vec{p}}}{(2\pi)^3}  \hOp{a}(\vec{p}) \comm{\Op{a}(\vec{p})}{\hOp{a}(\vec{k})} \ket{0} \nonumber\\
&= \half \int \frac{\dd[3]{\vec{p}}}{(2\pi)^3}  \hOp{a}(\vec{p}) (2 \pi)^3 2 E(\vec{p}) \delta^3(\vec{k} - \vec{p}) \ket{0} \nonumber\\
&= \int \dd[3]{p} E(\vec{p}) \delta^3(\vec{k} - \vec{p}) \hOp{a}(\vec{p}) \ket{0} \nonumber\\
\nOp{H} \ket{\vec{k}} &= E(\vec{k}) \ket{\vec{k}}
  \end{align}
\end{derivation}

\section{The Momentum Operator}
\label{sec:momentum-operator}

The momentum operator can be approached in the same way as the energy
operator, returning to the enery-momentum tensor.

\begin{equation}
  \label{eq:35}
  \ten{T}{^{0i}} = \pdv{\Lag}{(\partial_0 \Op{\phi})} \partial^i \Op{\phi} - \ten{g}{^{0i}} \Lag = \Op{\pi} \partial^i \Op{\phi}
\end{equation}

then the momentum operator is
\begin{equation}
  \label{eq:36}
  \Op{\vec{P}} = \half \int \nm{k} \vec{k} \qty( \Op{a}(\vec{k}) \hOp{a}{\vec{k}} + \hOp{a}(\vec{k}) \Op{a}{\vec{k}} )
\end{equation}
when combined with the Hamiltonian we can form the four-momentum operator,
\begin{fequation}[Momentum Operator]
  \label{eq:37}
  \Op{P} = \half \int \nm{k} k^{\mu} \qty( \Op{a}(\vec{k}) \hOp{a}{\vec{k}} + \hOp{a}(\vec{k}) \Op{a}{\vec{k}} )
\end{fequation}

\begin{bigderiv*}
  \begin{align*}
    \Op{\vec{P}} & = - \int \Op{\pi} \nabla \Op{\phi} \dd[3]{x}   \\ 
                 & = \half \int \dd[3]{x} \frac{\dd[3]{k}}{(2\pi)^3} \normmeasure{p} \qty( 
                         - \Op{a}(\vec{k}) \exp(-i k \vdot x) + \hOp{a}(\vec{k}) \exp(i k \vdot x) ) \vec{p} 
                           \qty( \Op{a}(\vec{p}) \exp(-i p \vdot x) - \hOp{a}(\vec{p}) \exp(i p \vdot x) 
                      )                                                               \\
                 & = \half \int \dd[3]{x} \frac{\dd[3]{k}}{(2\pi)^3}  \frac{\dd[3]{p}}{2 E(\vec{p})} \ \vec{p} \bigg(  
                         - \Op{a}(\vec{k}) \Op{a}(\vec{p}) \delta^3(\vec{k}+\vec{p}) \exp[ - i (E(\vec{k}) + E(\vec{p}) )t] 
                         + \Op{a}(\vec{k}) \hOp{a}(\vec{p}) \delta^3(\vec{k}-\vec{p}) \exp[ - i (E(\vec{k}) - E(\vec{p}) )t] \\
                 &\qquad + \hOp{a}(\vec{k}) \Op{a}(\vec{p}) \delta^3(\vec{k}-\vec{p}) \exp[ i (E(\vec{k}) - E(\vec{p}) )t] 
                         - \hOp{a}(\vec{k}) \hOp{a}(\vec{p}) \delta^3(\vec{k}+\vec{p}) \exp[ i (E(\vec{k}) + E(\vec{p}) )t] 
                     \bigg)  \\
                 & = \half \normmeasure{k} \vec{k}  \qty( 
                           \Op{a}(\vec{k}) \Op(-\vec{k}) e^{-2i E(\vec{k})t} 
                         + \Op{a}(\vec{k}) \hOp{a}(\vec{k}) 
                         + \hOp{a}(\vec{k}) + \Op{a}(\vec{k}) 
                         + \hOp{a}(\vec{k}) \hOp{a}(-\vec{k}) e^{2i E(\vec{k}) t} 
                      ) 
  \end{align*}
  The first and last terms are antisymmetric in their argument,
  i.e. in $\vec{k} \to - \vec{k}$, so they cancel, leaving equation
  \eqref{eq:36}
\caption{The momentum operator.}
\end{bigderiv*}

\section{Multiparticle States}
\label{sec:multiparticle-states}


Quantum field theory allows the description of multiparticle states, for example, $\ket{\vec{k}_1, \vec{k}_2} = \hOp{a}(\vec{k}_2) \hOp{a}(\vec{k}_1) \ket{0}$, which is a two particle state. We now have 

\begin{equation}
  \label{eq:38}
  \nOp{H} \ket{\vec{k}_1, \vec{k}_2} = \qty( E(\vec{k}_1) + E(\vec{k}_2) ) \ket{\vec{k}_1, \vec{k}_2}
\end{equation}

The multiparticle state's ket is symmetric,
\[ \ket{\vec{k}_1, \vec{k}_2} = \hOp{a}(\vec{k}_2) \hOp{a}(\vec{k}_1) \ket{0} =  \hOp{a}(\vec{k}_1) \hOp{a}(\vec{k}_2) \ket{0} =  \ket{\vec{k}_2, \vec{k}_1} \]
and it is possible to define another operator, $\Op{N}$, the number operator, which gives a count of the number of particles:
\begin{fequation}[Number Operator]
  \label{eq:39}
  \Op{N} = \int \nm{k} \hOp{\vec{k}} \Op{\vec{k}}
\end{fequation}
such that for the number of particles $n$,
\[ \Op{N}  \ket{\vec{k}_1, \dots, \vec{k}_n} = n  \ket{\vec{k}_1, \dots, \vec{k}_n} \]

\begin{derivation}
\begin{align*}
  \nOp{H} & \ket{\vec{k}_1, \vec{k}_2} = \half \int \frac{\dd[3]{p}}{(2 \pi)^3} \hOp{a}(\vec{p}) \Op{a}(\vec{p}) \hOp{a}(\vec{k}_2) \hOp{a}(\vec{k}_1) \ket{0}  \\
&= \half \int \frac{\dd[3]{p}}{(2 \pi)^3} \hOp{a}(\vec{p}) \qty( \comm{\Op{a}(\vec{p})}{\hOp{a}(\vec{k}_2)} + \hOp{a}(\vec{k}_2) \Op{a}(\vec{p}) \hOp{a}(\vec{k}_1) \ket{0} ) \\
&= E(\vec{k}_2) \hOp{a}(\vec{k}_2) \hOp{a}(\vec{k}_1) \ket{0} \\ &\qquad+ \half \int \frac{\dd[3]{p}}{(2 \pi)^3} \hOp{a}(\vec{p}) \hOp{a}(\vec{k}_2)  \comm{\Op{a}(\vec{p})}{\hOp{a}(\vec{k}_1)} \ket{0} \\
&= E(\vec{k}_2) \hOp{a}(\vec{k}_2) \hOp{a}(\vec{k}_1) \ket{0} + E(\vec{k}_1) \hOp{a}(\vec{k}_1) \hOp{a}(\vec{k}_2) \ket{0} \\
&= \qty(E(\vec{k}_1) + E(\vec{k}_2) ) \ket{\vec{k}_1, \vec{k}_2}
\end{align*}
\end{derivation}

\section{Complex Scalar Fields}
\label{sec:complex-scalar-field}


If we allow the field $\Op{\phi}(x)$ to be complex-valued the
Lagrangian for the theory becomes
\begin{equation}
  \label{eq:40}
  \Lag = \partial_{\mu} \Op{\hcon{\phi}} \partial^{\mu} \Op{\phi} - m^2 \Op{\hcon{\phi}} \Op{\phi}
\end{equation}
There are now two Euler-Lagrange equations, and a Klein-Gordon equation
\begin{equation}
  \label{eq:41}
  \qty( \partial^2 +m^2 ) \Op{\phi} = 0
\end{equation}
Similar to the real field there are general solutions of the form 
\begin{subequations}
  \begin{align}
    \label{eq:42}
    \Op{\phi}(x) &= \int \nm{k} \qty( \Op{a}(\vec{k}) e^{-i k \vdot x} + \hOp{b}(\vec{k}) e^{i k \vdot x} )\\
    \hOp{\phi}(x) &= \int \nm{k} \qty( \hOp{a}(\vec{k}) e^{i k \vdot x} + \Op{b}(\vec{k}) e^{-i k \vdot x} )
  \end{align}
\end{subequations}
\begin{derivation}
  The components of the Euler-Lagrange equation for the $\hcon{\phi}(x)$ field are

  \begin{align*}
    \pdv{\Lag}{\hOp{\phi}} &= -m^2 \Op{\phi}, & \pdv{\Lag}{(\partial_{\mu} \hOp{\phi})} &= \partial^{\mu} \Op{\phi} 
  \end{align*}
\begin{equation}
  \text{so} \quad \partial_{\mu} \qty(  \pdv{\Lag}{(\partial_{\mu} \hOp{\phi})} ) = \partial_{\mu} \partial^{\mu} \Op{\phi}
  \end{equation}
\end{derivation}

So there are two sets of creation and annihilation operators, and a
new set of commutation relations,
\[
\begin{aligned}
  \comm{\Op{a}(\vec{k})}{\Op{a}(\vec{p})} & = \comm{\Op{a}(\vec{k})}{\Op{b}(\vec{p})}    && = \comm{\Op{b}(\vec{k})}{\Op{b}(\vec{p})}    = 0 \\
\comm{\hOp{a}(\vec{k})}{\hOp{a}(\vec{p})} & = \comm{\hOp{a}(\vec{k})}{\hOp{b}(\vec{p})}  && = \comm{\hOp{b}(\vec{k})}{\hOp{b}(\vec{p})}   =0 \\
& \quad\ \comm{\Op{a}(\vec{k})}{\hOp{a}(\vec{p})}  && = \comm{\hOp{a}(\vec{k})}{\Op{b}(\vec{p})}   = 0 \\
 \comm{\Op{a}(\vec{k})}{\hOp{a}(\vec{p})} &= \comm{\Op{b}(\vec{k})}{\hOp{b}(\vec{p})} && = (2 \pi)^3 2 E(\vec{k}) \delta^3(\vec{k}-\vec{p})
\end{aligned}
\]
This also means the four-momentum operator is
\begin{align}
  \label{eq:43}
  \Op{P}^{\mu} = \half \int \nm{k} k^{\mu} \bigg( &\Op{a}(\vec{k}) \hOp{a}(\vec{k}) + \hOp{a}(\vec{k}) \Op{a}(\vec{k}) \nonumber \\+& \Op{b}(\vec{k}) \hOp{b} (\vec{k}) + \hOp{b}(\vec{k}) \Op{b}(\vec{k}) \bigg)
\end{align}
and its normal-ordered counterpart is
\begin{equation}
  \label{eq:44}
  \normbracket{\Op{P}^{\mu}} = \int \nm{k} k^{\mu} \qty( \hOp{a}(\vec{k}) \Op{a}(\vec{k}) + \hOp{b}(\vec{k}) \Op{b}(\vec{k}) )
\end{equation}
We can operate on the example state
\begin{align*} 
\normbracket{\Op{P}^{\mu}} \hOp{a}(\vec{k}_2)\hOp{a}(\vec{k}_1) \ket{0} &= ( k_1^{\mu} + k_2^{\mu} ) \hOp{a}(\vec{k}_2) \hOp{a}(\vec{k}_1),  
 \\
\normbracket{\Op{P}^{\mu}} \hOp{a}(\vec{k}_2)\hOp{b}(\vec{k}_1) \ket{0} &= ( k_1^{\mu} + k_2^{\mu} ) \hOp{a}(\vec{k}_2) \hOp{b}(\vec{k}_1) 
\end{align*}
While this looks more complicated than the real scalar field it can be
expressed in a way which is more straight-forward.

\begin{equation}
  \label{eq:11}
  \Op{\phi}(x) = \frac{1}{\sqrt{2}} \qty( \Op{\phi_1}(x) + i \Op{\phi_2}(x) )
\end{equation}
this gives a Lagrangian
\[
  \Lag = \half \partial_{\mu} \Op{\phi_1} \partial^{\mu} \Op{\phi_1} 
- \half m^2 \Op{ \phi_1 }^2  + \half \partial_{\mu} \Op{\phi_2} \partial^{\mu} \Op{\phi_2}
- \half m^2 \Op{ \phi_2 }^2
\]
The model is now just two real scalar fields, and the momentum is the sum
\[ \Op{P}^{\mu} = \Op{P}_1^{\mu} + \Op{P}^{\mu}_2 \]
\[ \Op{P}^{\mu}_i = \half \int \normmeasure{k} k^{\mu} \qty( \Op{a}_i(\vec{k}) \hOp{a}_i(\vec{k}) + \hOp{a}_i(\vec{k}) \Op{a}_i(\vec{k})  ) \]
with
\[ \Op{a} = \frac{1}{\sqrt{2}} (\Op{a}_1 + i \Op{a}_2), \qquad \Op{b} = \frac{1}{\sqrt{2}} (\Op{a}_1 - i \Op{a}_2) \]

\section{Charge Conservation}
\label{sec:charge-conservation}

The complex scalar Lagrangian has a symmetry under $\Op{\phi}(x) \to
e^{i \theta} \Op{\phi}(x)$, provided $\theta^{*} \to \theta$, and
$\partial_{\mu} \theta = 0$,
\begin{align*}
  \Lag &= \partial_{\mu} \hOp{\phi} \partial^{\mu} \Op{\phi} - m^2 \hOp{\phi} \Op{\phi} 
  \\& \to 
  \partial_{\mu} \hOp{\phi} e^{-i \theta} e^{i \theta} \partial^{\mu} \Op{\phi} - m^2 \hOp{\phi} e^{-i \theta} e^{i \theta} \Op{\phi} = \Lag
\end{align*}
According to Noether's Theorem there should be a conserved current
associated with this symmetry; the infinitessimal transformation is
$\Op{\phi}(x) \to \Op{\phi} + i \theta \Op{\phi}(x)$, so
\[ \Op{\jmath}^{\mu} = -i \pdv{\Lag}{(\partial_{\mu} \Op{\phi})}
\Op{\phi} + i \pdv{\Lag}{(\partial_{\mu} \hOp{\phi})} = -i \qty(
\Op{\phi} \partial^{\mu} \hOp{\phi} - \hOp{\phi} \partial^{\mu}
\Op{\phi}) \]
This conserved current gives rise to a conserved charge, which is the time component:
\begin{equation}
  \label{eq:12}
  \Op{Q} = - \int i \qty( \Op{\phi} \hOp{\pi} - \hOp{\phi} \Op{\pi} ) \dd[3]{x}
\end{equation}
The normal ordered operator is
\begin{align*} 
\nOp{Q} &= \int \nm{k} \qty( \hOp{a}(\vec{k}) \Op{a}(\vec{k}) - \hOp{b}(\vec{k}) \Op{b}(\vec{k}) ) \\
&= \Op{N_a} - \Op{N_b}
\end{align*}
thus we have
\[ \comm{ \nOp{Q} }{ \hOp{a} } = \hOp{a}, \qquad \comm{ \nOp{Q} }{ \hOp{b} } = - \hOp{b} \]
The two species can be distinguished by eigenvalues of $\nOp{Q}$, 
\[ \nOp{Q} \hOp{a} \ket{0} = \hOp{a} \ket{0} \qquad \text{particle $a$ has a positive charge,}\]
\[ \nOp{Q} \hOp{b} \ket{0} = - \hOp{b} \ket{0} \qquad \text{particle
  $b$ has a negative charge.}\] This allows us to interpret $b$ as the
antiparticle of $a$.

\section{The Heisenberg Picture}
\label{sec:heisenberg-picture}

Notice that everything so far is in the Heisenberg picture; the
operators are time-dependent. Here
\begin{align*}
  \dd_t{} \Op{\phi}(x) &= i \comm{ \Op{H} }{ \Op{\phi}(x) } = \Op{\pi}(x) \\
  \dd_t{} \Op{\pi}(x) &= i \comm{ \Op{H} }{ \Op{\pi}(x) } = \nabla^2 \Op{\phi}(x) - m^2 \Op{\phi}
\end{align*}
which is the equation of motion, the Klein-Gordon equation. Similarly
the states are time independent.

\section{Causality}
\label{sec:causality}

Our original postulate was that the field operators satisfy same-time
commutation relations, for a scalar field, e.g.
\begin{align*}
  \comm{ \Op{\phi}(\vec{x}, t) }{ \Op{\pi} (\vec{y}, t) } &= i \delta^3(\vec{x} - \vec{y}) \\
  \comm{ \Op{\phi}(\vec{x}, t) }{ \Op{\phi}(\vec{y}, t) } &= \comm{ \Op{\pi}(\vec{x}, t) }{ \Op{\pi}(\vec{y}, t) } = 0
\end{align*}
If, however, the times are allowed to be different we must consider
causality. Under a Lorentz transformation space-like separations
remain space-like, and the commutator
$\comm{\Op{\phi}(\vec{x})}{\Op{\phi}(\vec{y})}$ is also Lorentz
invariant, since its operators are. Thus, if the commutator is zero in
one frame it must be zero in all frames. The same argument holds for
commutators involving $\Op{\phi}$ and $\Op{\pi}$.

\section{The Propagator}
\label{sec:propagator}

We are left with the conundrum of what the amplitude of a particle
propagating from a point $y$ to $x$ is; to do this we project the
inital state, $\Op{\phi}(y) \ket{0}$ onto the final state,
$\Op{\phi}(x) \ket{0}$. For the real field this is
\begin{align*}
  D(x-y) &= \bra{0} {\Op{\phi}(x) \Op{\phi}(y)} \ket{0} \\
         &= \int \nm{k} \nm{p} \bra{0} \Op{a}(\vec{k}) \hOp{a}(\vec{p}) \ket{0} e^{-i (k \vdot x - p \vdot y)} \\
         &= \int \nm{k} e^{-ik \vdot(x-y)}
\end{align*}
This amplitude is the propagator of the theory; for a space-like $x-y$
with $x^0 - y^0 = 0$ this becomes
\begin{equation}
  \label{eq:13}
  D(x-y) \approx e^{-m |\vec{x} - \vec{y}|}
\end{equation}
Notably the amplitude is non-zero outside the light-cone, however.
Looking at the commutator
\begin{align*} 
   \comm{\Op{\phi}(x)}{\Op{\phi}(y)} &= \int \nm{k} \nm{p}
                                          \bigg( \comm{\Op{a}(\vec{k})}{\hOp{a}(\vec{p})} e^{i (k \vdot x - p \vdot y)}
                                              \\ & \qquad+ \comm{\hOp{a}(\vec{k})}{\Op{a}(\vec{p})} e^{i (k \vdot x - p \vdot y)} \bigg)\\
                                     &= \int \nm{k} \qty( e^{-ik \vdot(x-y)} - e^{ik \vdot (x-y)}) \\
                                     &= D(x-y) - D(y-x)
 \end{align*}
 Again we see that this commutator is zero for space-like separations;
 if $x-y$ is space-like then there is no well-defined concept of which
 happened first; it is always possible to choose a frame where $x$
 happens before $y$, and vice-versa. As such we need to add the
 contribution to the amplitude for travelling in each direction, and
 these cancel. 

 In the complex scalar field the amplitude for a particle travelling
 $x$ to $y$ cancels with that of an antiparticle travelling from $y$
 to $x$.

\subsection{The Feynman Propagator}
\label{sec:feynman-propagator}

The Feynman propagator is defined as 
\begin{fequation}
  \label{eq:14}
\Delta~F (x-y) = 
\begin{cases}
  \bra{0} \Op{\phi}(x) \Op{\phi}(y) \ket{0} \quad \text{for} \quad x^0 > y^0 \\
  \bra{0} \Op{\phi}(y) \Op{\phi}(x) \ket{0} \quad \text{for} \quad x^0 < y^0
\end{cases}
\end{fequation}
\newcommand{\tord}{\mathcal{T}\,}
In order to write this more compactly we introduce the new notation, the time-ordering operator, $\tord$:
\begin{equation}
  \label{eq:45}
  \tord \Op{\phi}(x) \Op{\phi}(y) = 
  \begin{cases}
    \Op{\phi}(x) \Op{\phi}(y) \quad \text{for} \quad x^0 > y^0 \\
    \Op{\phi}(y) \Op{\phi}(x) \quad \text{for} \quad x^0 < y^0
  \end{cases}
\end{equation}
The time-ordering operator arranges an expression so that the
operators at earlier times are left of operators which occur later.
Thus
\begin{equation}
  \label{eq:46}
  \Delta~F (x-y) = \bra{0}   \tord \Op{\phi}(x) \Op{\phi}(y) \ket{0}
\end{equation}
From contour integration in the complex plane,
\begin{equation}
  \label{eq:47}
  i \Delta~F = - \int \frac{\dd[4]{k}}{(2 \pi)^4} \frac{e^{-ik \vdot(x-y)}}{k^2 - m^2 + i \epsilon}
\end{equation}
This is a Green's function for the Klein-Gordon equation
\begin{align*}
  \label{eq:47}
  ( \partial^2 + m^2 - i \epsilon) &i \Delta~F \\ &= - \int \frac{\dd[4]{k}}{(2 \pi)^4} 
                         \frac{(-k^2 +m^2 -i \epsilon) e^{-ik \vdot(x-y)}}{k^2 - m^2 + i \epsilon} \\
&= \int  \frac{\dd[4]{k}}{(2 \pi)^4} e^{-i k\vdot (x-y)} \\ &= \delta^4(x-y)
\end{align*}
Let $\mathcal{E} = (k^0)^2 - E^2(\vec{k}) + \frac{i \epsilon}{2 E(\vec{k})}$. The Feynman propagator is then
\begin{align*}
  \label{eq:47}
  i \Delta~F &= - \int \frac{\dd[4]{k}}{(2 \pi)^4} \frac{e^{-ik \vdot(x-y)}}{k^2 - m^2 + i \epsilon} \\
             &= - \int \dd{k^0} \frac{\dd[3]{k}}{(2 \pi)^4} \frac{e^{-ik \vdot(x-y)}}{(k^0)^2 - E^2(\vec{k}) +i \epsilon} \\
             &= - \int \dd{k^0} \frac{\dd[3]{k}}{(2 \pi)^4} 
                  \frac{e^{-ik^0(x^0-y^0)} e^{-ik \vdot(x-y)}}{\mathcal{E} \hcon{\mathcal{E}} }
\end{align*}

\begin{figure}[h!]
  \centering
  \begin{tikzpicture}[scale=0.8]

\draw[->] (-5, 0) -- (5, 0) node [below] {$\Re(k^0)$};
\draw[->] (0, -5) -- (0 , 2) node [above] {$\Im(k^0)$};

\draw [ultra thick, accent-blue] (4, 0) node [above] {$C$} arc (360:180: 4) -- cycle;

\fill (-2, 1) circle (.5mm);
\draw [<->, help lines] (-2.3, 1) -- (-2.3, 0) node [midway, left] {$\frac{\epsilon}{2E(\vec{k})}$};
\draw [<->, help lines] (-2, 1.3) -- (0, 1.3) node [midway, above] {$-E(\vec{k})$};

\fill (2, -1) circle (.5mm);
\draw [<->, help lines] (2.3, -1) -- (2.3, 0) node [midway, right] {$\frac{\epsilon}{2E(\vec{k})}$};
\draw [<->, help lines] (2, -1.3) -- (0, -1.3) node [midway, below] {$-E(\vec{k})$};

\end{tikzpicture}
  \caption{Integration in the complex plane.}
  \label{fig:complexplane}
\end{figure}
Let us write $ e^{-i k^0(x^0-y^0)} = e^{-i \Re k^0(x^0-y^0)} e^{\Im
  k^0(x^0-y^0)} $, and so long as $x^0>y^0$ we can complete the
contour in the lower-half plane; in the limit where the radius of the
half-sphere goes to infinity the extra contribution will vanish, and so
\begin{align}
  \label{eq:48}
  i \Delta~F(x-y) = - \oint_C \dd{k^0} \frac{\dd[3]{k}}{(2 \pi)^4} 
                  \frac{e^{-ik^0(x^0-y^0)} e^{-ik \vdot(x-y)}}{\mathcal{E} \hcon{\mathcal{E}} }
\end{align}
Noting that $k^0 = E(\vec{k})$. The prescription of $i \epsilon$
ensures that only one of the poles is enclosed in the integration, by
rotating the plane, lifting the poles off the axis; applying the
residue theorem,
\begin{align*}
  \label{eq:48}
  i \Delta~F(x-y) &= -2 \pi i \int \frac{\dd[3]{k}}{(2 \pi)^4} 
                  \frac{e^{-ik^0(x^0-y^0)} e^{-ik \vdot(x-y)}}{2 E(\vec{k})} \\
&= i \int \normmeasure{k} e^{-ik\vdot(x-y)} \\ &= i D(x-y)
\end{align*}
If $x^0<y^0$ then we need to complete the curve in the upper-half
plane to allow the extra contribution to vanish, picking out the pole
at $k^0=-E(\vec{k})$, then, 
\begin{align*}
  \label{eq:48}
  i \Delta~F(x-y) &= -2 \pi i \int \frac{\dd[3]{k}}{(2 \pi)^4} 
                  \frac{e^{iE(\vec{k})(x^0-y^0)} e^{ik \vdot(x-y)}}{2 E(\vec{k})} \\
&= i \int \normmeasure{k} e^{ik\vdot(x-y)}, \quad \vec{k} \to - \vec{k}\\ &= i D(y-x)
\end{align*}


%%% Local Variables: 
%%% mode: latex
%%% TeX-master: "../project"
%%% End: 

\chapter{Interacting Scalar Fields}
\label{cha:inter-scal-fields}
The interacting scalar field is motivated by the Lagrangian
\begin{equation}
  \label{eq:53}
  \Lag = \half \pd{\mu} \Op{\phi} \pu{\mu} \Op{\phi} 
  -\half m^2 \Op{\phi}^2 - \frac{\lambda}{4!} \Op{\phi}^4
\end{equation}
This is a real scalar field, and the only dimensionless coupling we
can add is $\phi^4$. The energy function from the energy-momentum
tensor is
\[ \Op{T}^{00} = \pdv{\Lag}{\pd{0} \Op{\phi}} \pu{0} \Op{\phi} -
\Op{\Lag} \] and from this we can generate a new Hamiltonian,
\begin{equation}
  \label{eq:53}
  \Op{H} = \Op{H}_0 + \Op{H}_{\rm int}
\end{equation}
where $\Op{H}_0$ is the Hamiltonian of the free scalar field. The new
part is
\begin{equation}
  \label{eq:54}
  \Op{H}~{int} = - \int \Op{\Lag}~{int} \dd[3]{x} = \int \frac{\lambda}{4!} \Op{\phi}^4(x) \dd[3]{x}
\end{equation}
Ultimately we want to develop this to be able to make predictions
about particle scattering which can then be tested by experiment. If
$\lambda$ is small enough we can use perturbation theory to calculate
scattering cross-sections.

\section{The Dirac interaction picture}
\label{sec:interaction-picture}

Previously 
\begin{equation}
  \label{eq:55}
  \Op{\phi}~H (\vec{x}, t) = e^{i \Op{H}(t-t_0)} \Op{\phi}~S (\vec{x}) e^{-i \Op{H}_0 (t-t_0)}
\end{equation}
with Heisenberg picture operators changing with time, but Schrodinger
ones constant. In principle this is all that is necessary;
$\Op{\phi}~H (\vec{x}, t_0) = \Op{\phi}~S(\vec{x})$ is the operator at
the beginning, and we know how it changes with time.

This is hard to solve, so we can define operators in the interaction picture,
\begin{equation}
  \label{eq:56}
  \Op{\mathcal{O}}~{I}(\vec{x}, t) = e^{i \Op{H}_0(t-t_0)} \Op{\mathcal{O}}~S(\vec{x}) e^{-i \Op{H}_0(t-t_0)}
\end{equation}
for $\Op{H}_0$ the Hamiltonian of the free theory.
Consider
\begin{equation}
  \label{eq:57}
  \Op{\phi}~I (\vec{x},t) = e^{i \Op{H}_0(t-t_0)} \Op{\phi}~S (\vec{x}) e^{-i \Op{H}_0(t-t_0)}
\end{equation}
we've solved this, since it's the field operator for the
non-interacting theory.
\begin{equation}
  \label{eq:58}
  \Op{\phi}~I (x) = \int \nm{k} \qty( \Op{a}(\vec{k}) e^{i k \vdot x} + \hOp{a} (\vec{k}) e^{i k \vdot x} )
\end{equation}
This is related to the field operator for the interacting theory in
the Heisenberg picture by
\begin{align*}
  \Op{\phi}~H (\vec{x}, t) &= e^{i \Op{H}(t-t_0)} \Op{\phi}~S (\vec{x}) e^{-i \Op{H}(t-t_0)} \\ 
&= e^{i \Op{H}(t-t_0)} \qty[ e^{-i \Op{H}_0(t-t_0)} \Op{\phi}~I e^{i \Op{H}_0(t-t_0)} ] e^{-i \Op{H}(t-t_0)} \\
&= \hOp{U}(t,t_0) \Op{\phi}~I (\vec{x}, t) \Op{U}(t, t_0)
\end{align*}
with the time evolution operator defined
\begin{fequation}[Time evolution operator]
  \label{eq:59}
  \Op{U}(t, t_0) = e^{i \Op{H}_0 (t-t_0)} e^{-i \Op{H}(t-t_0)}
\end{fequation}
The interacting and free Hamiltonians do not commute, and so
$\Op{U}(t, t_0) \neq e^{-i \Op{H}~{int}(t-t_0)}$, and the CBH
expansion must be used.  We can obtain a differential equation by
differentiating $\Op{U}$,
\begin{align*}
  i \pdv{\Op{U}}{t} &= i \qty( \pdv{t} e^{i \Op{H}_0(t-t_0)} ) e^{-i\Op{H}(t-t_0)} + i e^{\Op{H}_0(t-t_0)} \pdv{t} e^{-i \Op{H}(t-t_0)} \\
&= -e^{i \Op{H}_0(t-t_0)} \Op{H}_0 e^{-i \Op{H}(t-t_0)} + e^{i \Op{H}_0(t-t_0)} \Op{H} e^{-i \Op{H}(t-t_0)} \\
&= \underbracket{e^{i \Op{H}_0 (t-t_0)} \Op{H}~{int} e^{-i \Op{H} (t-t_0)}}_{\OP{H}~{int, I}} \times \underbracket{e^{i \Op{H}_0(t-t_0)} e^{{-i \Op{H}(t-t_0)}}}_{\Op{U}(t-t_0)}
\end{align*}
We now have the problem of solving $ i \pdv{t} \Op{U}(t-t_0) =
\Op{H}~{int,I} \Op{U}(t,t_0)$ for $\Op{U}(t_0, t_0)=1$, in integral form
\begin{equation}
  \label{eq:60}
  \Op{U}(t-t_0) = 1 - i \int_{t_0}^t \dd{t_1 \Op{H}~{int,I}(t_1) } \Op{U}(t_1, t_0)
\end{equation}
but this can clearly continue \emph{ad infinitum}, so
\begin{align*}
\Op{U} = \sum_{n=0}^{\infty} (-i)^n & \int_{t_0}^t \dd{t_1}  \\ & \int_{t_0}^{t_1} \dd{t_2} \dots \int_{t_0}^{t_{n-1}} \dd{t_n} \Op{H}~{int,I}(t_1) \dots \Op{H}~{int,I}(t_n)
\end{align*}
we can simplify this by changing the area we integrate over, when $n=2$ for example,
\begin{align*}
  \int_{t_0}^t \dd{t_1} \int_{t_0}^{t_1} \dd{t_2} & \Op{H}~{int,I}(t_1) \Op{H}~{int,I}(t_2) \\
&= \half \int_{t_0}^t \dd{t_1} \dd{t_2} \Op{T} \qty{\underbracket{\Op{H}~{int,I} (t_1) \Op{H}~{int,I}(t_2)}_{{\text{symmetric under } t_1 \leftrightarrow t_2}} }
\end{align*}
Then
\begin{align*}
  \Op{U}(t,t_0) &= \sum_{n=0}^{\infty} \frac{(-i)^n}{n!} \int_{t_0}^t \dd{t_1} \cdots \dd{t_n} \Op{T}\qty{\Op{H}~{int,I}(t_1) \cdots \Op{H}~{int,I}(t_n)} \\ &= \Op{T}e^{\qty( -i \int_{t_0}^t \dd{t'} \Op{H}~{int, I} (t') )}
\end{align*}

Thus, operators in the interaction picture evolve according to
\begin{equation}
  \label{eq:62}
  \Op{\phi}~{I}(\vec{x},t) = \Op{U}(t,t_0) \Op{\phi}~H (\vec{x}) \hOp{U}(t,t_0)
\end{equation}
with state vectors given
\begin{equation}
  \label{eq:63}
  \ket{\phi}~I = \Op{U}(t,t_0) \ket{\phi}~H
\end{equation}
and
\begin{equation}
  \label{eq:64}
  \Op{U}(t,t_0) = \Op{T} \exp( -i \int_{t_0}^t \dd{t'} \Op{H}~{int,I}(t'))
\end{equation}
Since $\Op{U}$ is an operator containing both creation and
annihilation operators the number of particles can change with time.

\section{The S matrix}
\label{sec:s-matrix}

Consider an initial state, $\ket{i}$ of particles, at a time
$t=-\infty$, which interact with each other before reaching a final
state, $\ket{f}$ at time $t=\infty$.  In the Heisenberg picture these
states are constant and the final and initial states would be
equal. After the interaction we make a new measurement of the energy
and momentum of the final state and it collapses to the final state
with a probability $\abs{\braket{f}{i}}}^2$; we need the Interaction
picture to calculate the states, however.
\begin{subequations}
  \begin{align}
    \label{eq:61}
    \Op{\phi}~{in} (\vec{x}) &= \lim_{t \to - \infty} \Op{\phi}~{H}(\vec{x},t) = \lim_{t \to - \infty} \Op{\phi}~{I}(\vec{x},t) = \Op{\phi}~S (\vec{x}) \\
   \Op{\phi}~{out} (\vec{x}) &= \lim_{t \to   \infty} \Op{\phi}~{H}(\vec{x},t) = \lim_{\substack{ t \to   \infty \\ t_0 \to - \infty}} \hOp{U}(t,t_0) \Op{\phi}~{I}(\vec{x},t) \Op{U}(t,t_0)
  \end{align}
\end{subequations}
and then the required projection is
\begin{equation}
  \label{eq:65}
  S_{fi} = \braket{f}{i}~{H} = \lim_{\substack{t \to \infty \\ t_0 \to - \infty}} \bra{f} \Op{U}(t,t_0) \ket{i}~{I} = \bra{f} \Op{S} \ket{i}~{I}
\end{equation}
which is the S-matrix.
\begin{align}
  \label{eq:66}
  S_{fi} &=  \lim_{\substack{t \to \infty \\ t_0 \to - \infty}} \bra{f} \Op{U}(t,t_0) \ket{i} \nonumber\\
&= \bra{f} \Op{T} \exp( -i \int_{t_0}^t \dd{t'} \Op{H}~{int,I}(t')) \ket{i} \nonumber\\
&= \bra{f} \Op{T} \exp( -i \int \frac{\lambda}{4!} \Op{\phi}~{I}^4(x) \dd[4]{x} )\ket{i}
\end{align}
This can now be calculated using perturbation theory,
\begin{align}
  \label{eq:67}
  \bra{f}\Op{S}\ket{i} = &\braket{f}{i} - i \frac{\lambda}{4!} \int \dd[4]{x} \bra{f} \Op{T}\Op{\phi}^4~{I}(x) \ket{i} \nonumber\\
&+\qty( -i \frac{\lambda}{4!} )^2 \int \dd[4]{x} \dd[4]{x'} \bra{f} \Op{T}\Op{\phi}^4~{I}(x) \Op{\phi}^4~I(x') \ket{i}
\end{align}
We can make use of Wick's theorem to compute solutions involving
normal-ordered products and propagators.

\section{The vacuum}
\label{sec:vacuum}

In the free theory the lowest energy state was $\ket{0}$, and was
related to the field function $\op{\phi}$; thus, in the interacting
picture we have a new vacuum, $\ket{\Omega}$, and any state in the
interaction picture is not an eigenstate of the free theory, since
they interact with the virtual particles from the vacuum. We shall
ignore this problem for now.

\section{Wick's Theorem}
\label{sec:wicks-theorem}

% Wick's theorem states that two operators, $\Op{A}$ and $\Op{B}$ may be
% contracted according to
% \begin{equation}
%   \label{eq:68}
%   \wicon{A}{B} = \Op{A}\Op{B} - \normbracket{\Op{A}\Op{B}}
% \end{equation}

Consider $\Op{T} \Op{\phi}(x) \Op{\phi}(y)$, and then split the
positive and negative frequency components,
\begin{equation}
  \label{eq:69}
  \Op{\phi}(x) = \Op{\phi}^+(x) + \Op{\phi}^-(x)
\end{equation}
with $\Op{\phi}^+ = \int \nm{k} \Op{a}(\vec{k}) e^{-ik \vdot x}$
and $\Op{\phi}^- = \int \nm{k} \Op{a}(\vec{k}) e^{ik \vdot x}$.
For $x^0 > y^0$,
\begin{align}
  \label{eq:70}
  \Op{T} \Op{\phi}(x) \Op{\phi}(y) &= \quad\Op{\phi}^+(x) \Op{\phi}^+(y) + \Op{\phi}^+(x) \Op{\phi}^-(y) \nonumber\\
                                   &\quad+   \Op{\phi}^-(x) \Op{\phi}^+(y) + \Op{\phi}^-(x) \Op{\phi}^-(y) \nonumber\\
&= \quad\Op{\phi}^+(x) \Op{\phi}^+(y) + \Op{\phi}^-(y) \Op{\phi}^+(x) \nonumber\\
                                   &\quad+   \Op{\phi}^-(x) \Op{\phi}^+(y) + \Op{\phi}^-(x) \Op{\phi}^-(y) \nonumber\\
&\quad+ \comm{\phi^+(x)}{\phi^-(y)} \nonumber\\
&= \normbracket{\Op{\phi}(x) \Op{\phi}(y)} + D(x-y)
\end{align}
If $x^0<y^0$ then
$
  \tOrd \Op{\phi}(x) \Op{\phi}(y) = \normbracket{\Op{\phi}(x) \Op{\phi}(y)} + D(y-x)
$, so for any $x^0$ and $y^0$, 
\begin{equation}
  \label{eq:72}
   \tOrd \Op{\phi}(x) \Op{\phi}(y) = \normbracket{\Op{\phi}(x) \Op{\phi}(y)} + \Delta~F(x-y)
\end{equation}
Thus, for fields we can write Wick's theorem as
\begin{align}
  \tOrd \Op{\phi}(x_1) \Op{\phi}(x_2) \cdots \Op{\phi}(x_n) =& \normbracket{ \Op{\phi}(x_1) \Op{\phi}(x_2) \cdots \Op{\phi}(x_n)}\nonumber\\ &+ \text{all contractions.}
\end{align}

\providecommand{\dww}[4]{\normbracket{\Op{\phi}(x_{#1}) \Op{\phi}(x_{#2})} \,  \Delta~F (x_{#3} - x_{#4}) }
\begin{example}[Wick Contractions]
\begin{align*} \allowdisplaybreaks
 \tOrd \Op{\phi}(x_1) \Op{\phi}(x_2) &\Op{\phi}(x_3) \Op{\phi}(x_4) \\ =& \quad\, \normbracket{\Op{\phi}(x_1) \Op{\phi}(x_2) \Op{\phi}(x_3) \Op{\phi}(x_4)} \\
&+ \dww{3}{4}{1}{2} \\&+ \dww{2}{4}{1}{3} \\ &+ \dww{2}{3}{1}{4} \\&+ \dww{1}{4}{2}{3} \\ &+ \dww{1}{3}{2}{4} \\&+ \dww{1}{2}{3}{4} \\
&+ \Delta~F (x_1 - x_2) \Delta~F (x_3 - x_4) \\ &+ \Delta~F (x_1 - x_3) \Delta~F(x_2-x_4) \\ &+ \Delta~F(x_1-x_4) \Delta~F (x_2-x_3) \\
\bra{0}  \tOrd \Op{\phi}(x_1) \Op{\phi}(x_2) &\Op{\phi}(x_3) \Op{\phi}(x_4) \ket{0} \\
=&\quad\, \Delta~F (x_1 - x_2) \Delta~F (x_3 - x_4) \\ &+ \Delta~F (x_1 - x_3) \Delta~F(x_2-x_4) \\ &+ \Delta~F(x_1-x_4) \Delta~F (x_2-x_3) \\
\end{align*}
\end{example}
A common notation, used to simplify the appearance of contractions is
\begin{equation}
  \label{eq:73}
  \contraction{}{\Op{\phi}}{(x)}{\Op{\phi}}{}
  \Op{\phi}(x) \Op{\phi}(y) = \Delta~F (x-y)
\end{equation}

\section{$2 \to 2$ scattering}
  Consider a system which has an initial state with two particles at
  momenta $\vec{k}_1$ and $\vec{k}_2$, and a final state with momenta
  $\vec{p}_1$ and $\vec{p}_2$.
\begin{center}
  \begin{tfeyn}
      \tfcol{k2,k1} 
      \tfcol{p2,p1}
      \tf[f]{k2,v1,p1}
      \tf[f]{k1,v2,p2}
      \fill (v1) circle (0.2);
      \draw (k1.west) node {$\vec{k}_1$}; \draw (k2.west) node {$\vec{k}_2$};
\draw (p1.east) node {$\vec{p}_1$}; \draw (p2.east) node {$\vec{p}_2$};
  \end{tfeyn}
\end{center}
The initial state is 
\[ \ket{\vec{k}_1, \vec{k}_2} = \hOp{a}(\vec{k}_2) \hOp{a}(\vec{k}_1) \ket{0} \]
while the final state is
\[ \ket{\vec{p}_1, \vec{p}_2} = \hOp{a}(\vec{p}_2) \hOp{a}(\vec{p}_1) \ket{0} \]
The first term in the expansion of the $S$-matrix is then
{ \tiny
\begin{align*}
  &\braket{\vec{p}_1, \vec{p}_2}{\vec{k}_1, \vec{k}_2} = \bra{0} \Op{a}(\vec{p}_2) \Op{a}(\vec{p}_1) \hOp{a}(\vec{k}_2) \hOp{a}(\vec{k}_1) \ket{0}\\
  &= \bra{0} \Op{a}\vec{p}_2\qty( \comm{\Op{a}(\vec{p}_1)}{\hOp{a}(\vec{k}_2)} + \hOp{a}(\vec{k}_2) \Op{a}(\vec{p}_1) ) \hOp{a}(\vec{k}_1) \ket{0} \\
&= \comm{\Op{a}(\vec{p}_1)}{\hOp{a}(\vec{k}_2)}  \bra{0} \Op{a}(\vec{p}_2) \hOp{a}(\vec{k}_1) \ket{0}  \\
&\qquad+  \bra{0} \Op{a}(\vec{p}_2) \hOp{a}(\vec{k}_2) \ket{0}  \comm{\Op{a}(\vec{p}_1)}{\hOp{a}(\vec{k}_2)} \ket{0}\\
&= \comm{\Op{a}(\vec{p}_1)}{\hOp{a}(\vec{k}_2)} \comm{\Op{a}(\vec{p}_1)}{\hOp{a}(\vec{k}_2)} + \comm{\Op{a}(\vec{p}_1)}{\hOp{a}(\vec{k}_2)} \comm{\Op{a}(\vec{p}_1)}{\hOp{a}(\vec{k}_2)}
\end{align*}
}
Since
{\tiny
\[ \comm{\Op{a}(\vec{p}_1)}{\hOp{a}(\vec{k}_1)} =(2\pi)^3 2 E(\vec{p}_1) \delta^3(\vec{p}_1 - \vec{k}_1) \]
}
We have {\tiny
\begin{equation*}
  (2 \pi)^6 E(\vec{k}_1) E(\vec{k}_2) \qty( \delta^3 (\vec{p}_1 - \vec{k}_1) \delta^3(\vec{p}_2 - \vec{k}_2) + \delta^3(\vec{p}_1 - \vec{k}_2) \delta^3(\vec{p}_2 - \vec{k}_1) )
\end{equation*} }
In diagrammatic form this is
\begin{equation*}
   \begin{tfeynma}[1em]
     \tfcol{k2,k1} \tfcol{p2,p1} \tf{k1,p1} \tf{k2,p2}
   \end{tfeynma}
 +
 \begin{tfeynma}[1em]
     \tfcol{k2,k1} \tfcol{p2,p1} \tf{k1,p2} 
\fill [white] (0.7em,2) circle (0.3);
\tf{k2,p1}
 \end{tfeynma}
\end{equation*}
This isn't scattering, so can be excluded from the calculation.
The second term is
\[ -i \frac{\lambda}{4!} \int \dd[4]{x} \bra{\vec{p}_1,\vec{p}_2} T \Op{\phi}^4(x) \ket{\vec{k}_1, \vec{k}_2} \]
By Wick's theorem,{\small
\[ T \Op{\phi}^4(x) =  \normbracket{\Op{\phi}^4(x)} + 6 \normbracket{\Op{\phi}^2(x)} \Delta~F(x-x) + 3 \Delta~F(x-x) \Delta~F(x-x)\]
}
The normal-ordered product gives
\begin{align*} -& i \frac{\lambda}{4} \int \dd[4]{x} \nm{q_1} \nm{q_2} \nm{q_3} \nm{q_4}  e^{i (q_1+q_2-q_3-q_4) \vdot x}\\
& \times \bra{0} \Op{a}(\vec{p}_2) \Op{a}(\vec{p}_1) \hOp{a}(\vec{q}_1) \hOp{a}(\vec{q}_2) \Op{a}(\vec{q}_3) \Op{a}(\vec{q}_4) \hOp{a}(\vec{k}_2) \hOp{a}(\vec{k}_1) \ket{0}
\end{align*}
Then, since, for example $\Op{a}(\vec{q}_4) \hOp{a}(\vec{k}_2) =
\comm{\Op{a}(\vec{q}_4)}{\hOp{a}(\vec{k}_4)} + \hOp{a}(\vec{k}_2) \Op{a}(\vec{q}_4)$, and working through
each of the terms,
 \begin{align*}
    \bra{0} & \Op{a}(\vec{p}_2) \Op{a}(\vec{p}_1) \hOp{a}(\vec{q}_1) \hOp{a}(\vec{q}_2) \Op{a}(\vec{q}_3) \Op{a}(\vec{q}_4) \hOp{a}(\vec{k}_2) \hOp{a}(\vec{k}_1) \ket{0} \\
 &= 4 \comm{\Op{a}(\vec{p}_1)}{\hOp{a}(\vec{q}_1)} \comm{\Op{a}(\vec{p}_2)}{\hOp{a}(\vec{q}_2)} \\ &\qquad\times\comm{\Op{a}(\vec{q}_3)}{\hOp{a}(\vec{k}_2)} \comm{\Op{a}(\vec{q}_4)}{\hOp{a}(\vec{k}_1)}
 \end{align*}
 Then, given the commutation relations, $
 \comm{\Op{a}(\vec{p}_1)}{\hOp{a}(\vec{q}_1)} = (2 \pi)^3 3
 E(\vec{p}_1) \delta^3(\vec{p}_1 - \vec{q}_1)$ and so forth, 
 \begin{align*}
   &= - i \lambda \int  \exp(i [p_1+p_2-k_1-k_2] \vdot x ) \\
&= - i \lambda  (2 \pi)^4 \delta^4(p_1 + p_2 - k_1 - k_2) \\
&=
\begin{tfeynma}[1em]
  \tfcol{k2,k1} \tfcol{p2,p1} \tf{k2,p1} \tf{k1,p2}
\end{tfeynma}
 \end{align*}
 The next part is the
\begin{align*} 
-i & \frac{\lambda}{4!} \int \bra{\vec{p}_1, \vec{p}_2} \normbracket{\Op{\phi}^2(x)} \ket{\vec{k}_1 \vec{k}_2} \Delta~F (x-x) \\
&= -2i \frac{\lambda}{4!} \int \dd[4]{x} \nm{q_1} \nm{q_2} e^{i(q_1-q_2) \vdot x} \Delta~F(x-x) \\ &\qquad\bra{0} \Op{a}(\vec{p}_2) \Op{a}(\vec{p}_1) \hOp{a}(\vec{q}_1) \Op{a}(\vec{q}_2)  \hOp{a}(\vec{k}_2) \hOp{a}(\vec{k}_1) \ket{0}\\
&= - i \frac{\lambda}{12} \int \frac{\dd[4]{k}}{(2 \pi)^4} \nm{q_1} \nm{q_2} \frac{\delta^4(q_1 - q_2)}{k^2 - m^2 + i \epsilon} \\ & \qquad \bra{0} \Op{a}(\vec{p}_2) \Op{a}(\vec{p}_1) \hOp{a}(\vec{q}_1) \Op{a}(\vec{q}_2)  \hOp{a}(\vec{k}_2) \hOp{a}(\vec{k}_1) \ket{0} \\
&= i \frac{\lambda}{12} \int \frac{\dd[4]{k}}{(2 \pi)^4} 2(2 \pi)^3  \frac{\delta^4(q_1 - q_2)}{k^2 - m^2 + i \epsilon} \\
& \qquad \bigg( E(\vec{k}_1) \delta^3 (\vec{p}_2 - \vec{p}_1) \delta^3 (\vec{p}_1 - \vec{k}_1) \delta^3(\vec{q}_2 - \vec{k}_2) \\
& \qquad \quad E(\vec{k}_1) \delta^3(\vec{p}_2 - \vec{k}_1) \delta^3 (\vec{p}_1 - \vec{q}_1) \delta^3 (\vec{q}_2 - \vec{k}_2) \\
& \qquad \quad E(\vec{k}_2) \delta^3(\vec{p}_2 - \vec{q}_1) \delta^3(\vec{p}_1 - \vec{k}_2) \delta^3 (\vec{q}_2 - \vec{k}_1) \\
& \qquad \quad E(\vec{k}_2) \delta^3(\vec{p}_2 - \vec{k}_2) \delta^3(\vec{p}_1 - \vec{q}_1) \delta^3 (\vec{q}_2 - \vec{k}_1) \bigg) \\
&= i \frac{\lambda}{6} \int \frac{\dd[4]{k}}{(2 \pi)^4} \frac{1}{k^2-m^2+ i \epsilon}\\ 
&\qquad [ E(\vec{k}_1) (\delta^3(\vec{p}_1 - \vec{k}_1) \delta^4(p_2 - k_2) + \delta^3(\vec{p}_2-\vec{k}_1) \delta^4(p_1-k_2) ) \\
&\qquad   E(\vec{k}_2) (\delta^3(\vec{p}_1 - \vec{k}_2) \delta^4(p_2 - k_1) + \delta^3(\vec{p}_2 - \vec{k}_2) \delta^4(p_1-k_1) )] \\
&= 
  \begin{tfeynma}   \tfcol{k2,k1}   \tfcol{p2,p1}   \tf{k1,p1} \tf{k2,p2} \draw ($(k2)!0.5!(p2)$) node (c2) {}; \draw (c2.center) to (c2.north west) to [loop] (c2.north east) to (c2.center);  \end{tfeynma}
+ \begin{tfeynma}   \tfcol{k2,k1}   \tfcol{p2,p1}   \tf{k1,p1} \tf{k2,p2} \draw ($(k1)!0.5!(p1)$) node (c2) {}; \draw (c2.center) to (c2.south east) to [loop] (c2.south west) to (c2.center); \end{tfeynma}
+ \begin{tfeynma}   \tfcol{k2,k1}   \tfcol{p2,p1}   \tf{k1,p2} \draw ($(k1)!0.5!(p2)$) node (c2) {}; \fill [white] (c2) circle (0.3);  \tf{k2,p1}   \draw (c2.center) to (c2.north west) to [loop] (c2.north east) to (c2.center); \end{tfeynma}
+ \begin{tfeynma}   \tfcol{k2,k1}   \tfcol{p2,p1}   \tf{k2,p1} \draw ($(k1)!0.5!(p2)$) node (c2) {}; \fill [white] (c2) circle (0.3);  \tf{k1,p2}  \draw [rounded corner](c2.center) to (c2.north west) to [loop] (c2.north east) to (c2.center);  \end{tfeynma}
\end{align*}
None of these lines transfer momentum, so there is no scattering here
either. Notice that the integral is also divergent (a point which is
addressed later). 

The last part is also not scattering:
\begin{equation*}
  -i \frac{\lambda}{6} \int \braket{\vec{p}_1, \vec{p}_2}{\vec{k}_1, \vec{k}_2} \Delta~F^2(x-x) \dd[4]{x} = \begin{tfeynma}[1em]   \tfcol{k2,k1}   \tfcol{p2,p1}   \tf{k1,p1} \tf{k2,p2} \end{tfeynma} 
  \begin{tfeynma}[0.3em][0.5ex]
    \tfcol{k2,k1} \tfcol{p2,p1} \tf{k1,p2} \tf{p1,k2} \tf[loop]{k1,p1} \tf[loop]{p2,k2}
  \end{tfeynma} +
\begin{tfeynma}[1em]   \tfcol{k2,k1}   \tfcol{p2,p1}   \tf{k1,p2} \draw ($(k1)!0.5!(p2)$) node (c2) {}; \fill [white] (c2) circle (0.3);  \tf{k2,p1} ; \end{tfeynma}
  \begin{tfeynma}[0.3em][0.5ex]
    \tfcol{k2,k1} \tfcol{p2,p1} \tf{k1,p2} \tf{p1,k2} \tf[loop]{k1,p1} \tf[loop]{p2,k2}
  \end{tfeynma}
\end{equation*}
Here the figure-of-eight diagrams are produced by the propagator
$\Delta~F^2(x-x)$, and are a consequence of using the free rather than
the interacting vacuum.

The next term in the perturbative expansion is
\[ \qty(-i \frac{\lambda}{4!})^2 \int \bra{\vec{p}_1,\vec{p}_2} T
\Op{\phi}^4(x) \Op{\phi}^4(y) \ket{\vec{k}_1, \vec{k}_2} \] The
time-ordering is now important since there are two events, $x$ and
$y$. These connected scattering events have the form
\begin{equation*}
\begin{tfeynma}[1em]
  \tfcol{k2,k1} 
  \tfcol{p2,p1}
  \tf[left]{k2,p2}
  \tf[right]{k1,p1}
\end{tfeynma}
+
\begin{tfeynma}[.45em]
  \tfcol{k2,k1} 
  \tfcol{p2,p1}
  \tf[left]{k1,k2}
  \tf[right]{p1,p2}
\end{tfeynma}
+
\begin{tfeynma}
    \tfcol{k2,k1} 
    \tfcol{p2,p1}
    \tf{p1,k2}
    \tf{p2, k1}
    \draw ($(k2)!0.75!(p1)$) node (c2) {};
    \draw  (c2.center)to (c2.north west) to [loop] (c2.north) to (c2.center);
\end{tfeynma} + \cdots
\end{equation*}
The momentum cirulating in the loop is unconstrained, and so must be
integrated over, but this integral will be divergent.

\section{Feynman Rules}
\label{sec:feynman-rules}

\begin{enumerate}
\item For each propagator, 
  \[ \begin{tfeynma}    \tfcol{a} \tfcol{b} \tf{a,b}  \end{tfeynma} = \frac{i}{k^2-m^2+i \epsilon}\]
\item For each vertex,
\[  \begin{tfeynma}\tfcol{k2,k1} \tfcol{p2,p1} \tf{k2,p1} \tf{k1,p2} \end{tfeynma} = -i \lambda \]
\item Momentum must be conserved at every vertex, so e.g.~
\[ (2 \pi)^4 \delta^4(p_1 + p_2 - k_1 - k_2) \]
\item Every unconstrained momentum must be integrated over,
\[ \int \frac{\dd[4]{k}}{(2 \pi)^4} \]
\item Factors must be included for the number of symmetrical
  arrangements of diagram possible.
\end{enumerate}

\section{The true vacuum}
\label{sec:true-vacuum}

The true vacuum of the full interacting Hamiltonian is represented as
$\ket{\Omega}$, and defined such that
\begin{equation}
  \label{eq:71}
  \Op{H}~{int} \ket{\Omega} = 0
\end{equation}

We can then define the \emph{$n$-point Green's function}, or $n$-point
correlator,
\begin{equation}
  \label{eq:74}
  G_n(x_1, x_2, \dots, x_n) \equiv \bra{\Omega} T \Op{\phi}~H(x_1) \Op{\phi}~H(x_2) \dots \Op{\phi}~H(x_n) \ket{\Omega}
\end{equation}
Assuming that the product of $x$s are time ordered already, the
Heisenberg fields can be converted to interaction fields by
\begin{equation}
  \label{eq:75}
  \Op{\phi}~H(\vec{x}) = \hOp{U}(t,t_0) \Op{\phi}~I(\vec{x}) \Op{U}(t,t_0)
\end{equation}
Then, using the relations
\[ \Op{U}(t_1,t_2) \Op{U}(t_2,t_3) = \Op{U}(t_1, t_3) \]
and
\[ \Op{U}(t_1,t_3) \hOp{U}(t_2, t_3) = \Op{U}(t_1, t_2) \]
it is possible to shorten each pair
\[ \begin{split}
&\Op{\phi}~H(x_1) \Op{\phi}~H(x_2) \\&\quad= 
\hOp{U}(t_1,t_0) \Op{\phi}~I(\vec{x}) \Op{U}(t_1,t_0) 
\hOp{U}(t_2,t_0) \Op{\phi}~I(\vec{x}) \Op{U}(t_2,t_0)
\\ &\quad=
\hOp{U}(t_1,t_0) \Op{\phi}~I(\vec{x}) \Op{U}(t_1,t_2) \dots
\end{split}
\]

Now, taking $\exp(-i \Op{H} t) \ket{0}$, and inserting a complete set of energy states,
\begin{equation}
  \label{eq:76}
  \begin{split}
    e^{(-i \Op{H} t )} \ket{0} &= e^{(-i \Op{H} t)} \ket{\Omega} \braket{\Omega}{0}+ \sum_{n \neq 0}  e^{(-i \Op{H} t)} \ket{n} \braket{n}{0}\\
&=  e^{(-i E_0 t)} \ket{\Omega} \braket{\Omega}{0}+ \sum_{n \neq 0}  e^{(-i E_n t)} \ket{n} \braket{n}{0}
  \end{split}
\end{equation}
Taking $t \to \infty$ then all but the first term vanish
(Riemann-Lebesgue lemma), so
\[ \lim_{t \to \infty} e^{-i \Op{H} t} \ket{0} = \lim_{t \to \infty} e^{-i E_0 t} \ket{\Omega} \braket{\Omega}{0} \]
Then
\begin{align*}
  \ket{\Omega} &= \lim_{t \to \infty} \qty( e^{-i E_0 t} \ket{\Omega}
  \braket{\Omega}{0})^{-1} e^{-i H t} \ket{0} \\
&=  \lim_{t \to \infty} \qty( e^{-i E_0 (t+t_0)} \braket{\Omega}{0} )^{-1} e^{-i H(t+t_0)}\ket{0} \\
&=  \lim_{t \to \infty} \qty( e^{-i E_0 (t+t_0)} \braket{\Omega}{0} )^{-1} e^{-i \Op{H}(t+t_0)} e^{-i \Op{H}_0(t+t_0)}\ket{0} \\
&=  \lim_{t \to \infty} \qty( e^{-i E_0 (t+t_0)} \braket{\Omega}{0} )^{-1} \Op{U}(t_0-t) \ket{0}
\end{align*}
By similar logic,
\begin{equation*}
  \bra{\Omega} = \lim_{t \to \infty} \bra{0} \Op{U}(t,t_0) \qty( e^{-i E_0(t-t_0)} \braket{0}{\Omega} )^{-1}
\end{equation*}

Thus
\begin{align*}
  \bra{\Omega} & \Op{\phi}~H(x_1) \Op{\phi}~H(x_2) \dots \Op{\phi}~H(x_n)  \ket{\Omega} \\
&=  \lim_{t \to \infty} \qty(\abs{\braket{0}{\Omega}}^2 e^{-i 2E_0 t} )^{-1}
\\ & \qquad \qquad \bra{0}T \Op{\phi}~I(x_1) \Op{\phi}~I(x_2) \dots \Op{\phi}~I(x_n) \Op{U}(t,-t) \ket{0}
\end{align*}

This is true for all $n$, so choosing $n=0$,
\begin{align*}
  \braket{\Omega} = \lim_{t \to \infty} \qty( \abs{ \braket{0}{\Omega} }^2 e^{-i 2 E_0 t} )^{-1} \bra{0} \Op{U}(t,-t) \ket{0} 
\end{align*}
This implies
\begin{equation}
  \label{eq:77}
  \lim_{t \to \infty} \qty( \abs{ \braket{0}{\Omega} }^2 e^{-i 2 E_0 t} )^{-1} = \lim_{t \to \infty} \frac{1}{\bra{0} \Op{U}(t,-t) \ket{0}}
\end{equation}
The \emph{$n$-point Green's function} is then
\begin{equation}
  \label{eq:78}
  G_n (x_1, x_2, \dots, x_n) \frac{\bra{0} T \Op{\phi}~I(x_1) \Op{\phi}~I(x_2) \dots \Op{\phi}~I(x_n) \Op{S}\ket{0}}{\bra{0} \Op{S} \ket{0}}
\end{equation}
This division is the justification for the removal of disconnected
diagrams, as they appear in both the numerator and the denominator.

\section{The LSZ Reduction formula}
\label{sec:lsz-reduct-form}

The $n$-point Green's function is related to the S-matrix expectation
values via the LSZ reduction formula
\begin{align*}
&  \bra{\vec{p}_1, \vec{p}_2, \dots, \vec{p}_n} \Op{S} \ket{\vec{k}_1, \vec{k}_2, \dots, \vec{k}_n} \\
&= i^{m+n} \int \dd[4]{x_1} \dots \dd[4]{x_m} \dd[4]{y_1} \dots \dd[4]{y_n} \\
& \times \exp(-i \qty[ k_1 \vdot x_1 + \cdots + k_m \vdot x_m - p_1 \vdot y_1 - \cdots p_n \vdot y_n])\\
& \times \qty( \partial_{x_1}^2 + m^2) \cdots \qty(\partial_{x_m}^2 + m^2) \qty( \partial_{y_1}^2 + m^2) \cdots (\partial_{y_n}^2 + m^2) \\
& \times G_n(x_1, x_2, \dots, x_n)
\end{align*}

%%% Local Variables: 
%%% mode: latex
%%% TeX-master: "../project"
%%% End: 


\chapter{The Free Dirac Field}
\label{cha:free-dirac-field}

Before the advent of quantum field theory the presence of negative
energy solutions to the Klein-Gordon equation was considered a major
problem in quantum theory. These states are normally considered to be
positive energy anti-particle states, a view known as the
Feynman-St\"uckelberg interpretation. Only quantum field theory can
adequately manage the production of particle-antiparticle pairs.

\section{The Dirac equation}
\label{sec:dirac-equation}

Dirac attempted to avoid the problem of negative energy states by
finding a field equation linear in its operators. This must take the
form
\[ E = \vec{\alpha} \vdot \vec{p} + \beta m \to i \pdv{\psi}{t} = \qty(-i \vec{\alpha} \vdot \vec{\nabla} + \beta m) \psi \] and we need to find $\vec{\alpha}$ and $\beta$.
If we also insist that we satisfy
\[ E^2 = m^2 + \abs{\vec{p}}^2 \]
then, with implicit summation,
\begin{align*} E^2 &= \alpha_i \alpha_j p_i p_j + (\alpha)_i \beta + \beta \alpha_i) m p_i + \beta^2 m^2 \\
&= \half(\alpha_i \alpha_j + \alpha_j \alpha_i) p_i p_j + ( \alpha_i \beta + \beta \alpha_i) m p_i + \beta^2 m^2
\end{align*}
Enforcing $E^2 = m^2 + \abs{\vec{p}}^2$ we have the conditions on
$\vec{\alpha}$ and $\beta$.

\begin{subequations}
\begin{align}
  \label{eq:79}
  \alpha_i \alpha_j + \alpha_j \alpha_i &= 2 \delta_{ij} \\
\alpha_i \beta + \beta \alpha_i &= 0 \\
\beta^2 &= 1
\end{align}
\end{subequations}
These are anti-commuting objects and not just numbers. These relations
define them, and any representation which follows these conditions is
a suitable description. The Dirac representation presents them as
matrices,
\begin{equation}
  \label{eq:80}
  \alpha =
  \begin{pmatrix}
    0 & \vec{\sigma} \\ \vec{\sigma} & 0
  \end{pmatrix}
\end{equation}
and 
\begin{equation}
  \label{eq:81}
  \beta =
  \begin{pmatrix}
    \vec{1} & 0 \\ 0 & - \vec{1}
  \end{pmatrix}
\end{equation}
for $\sigma_i$ the Pauli matrices,
\begin{equation}
  \label{eq:82}
  \sigma_1 =  \begin{bmatrix}     0 & 1 \\ 1 & 0   \end{bmatrix}, \quad
\sigma_2 =    \begin{bmatrix}     0 & -i \\ i & 0  \end{bmatrix}, \quad
\sigma_3 =    \begin{bmatrix}     1 & 0  \\ 0 & -1 \end{bmatrix}
\end{equation}
Since these act on the field the field itself must have four
components, and so is a spinor.

\section{Gamma matrices}
\label{sec:gamma-matrices}

The Dirac equation can be written in four-vector form by defining a
new object, $\gamma^{\mu}$,
\begin{equation}
  \label{eq:83}
  \gamma^0 \equiv \beta = \begin{pmatrix} \vec{1} & 0 \\ 0 & - \vec{1}  \end{pmatrix}, \qquad
\vec{\gamma} \equiv \beta \alpha = \begin{pmatrix} 0 & \vec{\sigma} \\ - \vec{\sigma} & 0 \end{pmatrix}
\end{equation}
These follow the anticommutation relations
\begin{equation}
  \label{eq:84}
  \acomm{\gamma^{\mu}}{\gamma^{\nu}} = 2 g^{\mu \nu}
\end{equation}
and the Dirac equation becomes
\begin{equation}
  \label{eq:85}
  \qty( i \gamma^{\mu} \partial_{\mu} - m) \psi = 0
\end{equation}
We can write the contraction with a gamma matrix in the slash
notation, so $\ds = \gamma^{\mu} \partial_{\mu}$. This equation
describes the free dynamics of a fermion field.

\subsection{Properties of the Dirac matrices}
\label{sec:prop-dirac-matr}

\begin{subequations}
  \begin{equation}
    \label{eq:86}
    \hcon{(\gamma^0)} = \gamma^0, \quad \hcon{(\gamma^i)} = - \gamma^i
  \end{equation}
  \begin{equation}
    \label{eq:87}
    (\gamma^0)^2 = 1, \quad (\gamma^i)^2 = -1
  \end{equation}
  \begin{equation}
    \label{eq:88}
    \hcon{(\gamma^{\mu})} = \gamma^0 \gamma^{\mu} \gamma^0
  \end{equation}
\end{subequations}
The gamma matrices must be of even dimension, and are fixed matrices,
despite looking like four-vectors, and are not invariant under a
Lorentz boost.

\section{Negative energy solutions?}
\label{sec:negat-energy-solut}

we look for plane-wave solutions of the form
\[ \psi(t, \vec{x}) = u(\vec{p}) e^{-i(Et - \vec{p} \vdot \vec{x})} =
\begin{bmatrix}
  \chi \\ \phi
\end{bmatrix}
e^{-i(Et - \vec{p} \vdot \vec{x})}
\]
Shifting back out of four-vector notation,
\[ i \pdv{\psi}{t} = (-i \vec{\alpha} \vdot \nabla + \beta m) \psi \implies E
\begin{bmatrix}
  \chi \\ \phi
\end{bmatrix}
=
\begin{bmatrix}
  m & \vec{\alpha} \vdot \vec{p} \\ \vec{\sigma} \vdot \vec{p} & -m
\end{bmatrix}
\begin{bmatrix}
  \chi \\ \phi
\end{bmatrix}
\]
For a particle at rest $\vec{p}=\vec{0}$, so
\begin{equation}
  \label{eq:89}
  E \begin{bmatrix}
  \chi \\ \phi
\end{bmatrix}
=
\begin{bmatrix}
  m & 0 \\ 0 & -m
\end{bmatrix}
\begin{bmatrix}
  \chi \\ \phi
\end{bmatrix}
\end{equation}
This has the solutions, the first two for $E=m$, the latter two for $E=-m$
\[ u =
\begin{bmatrix}
  1 \\ 0 \\ 0 \\ 0
\end{bmatrix}, \quad u = \begin{bmatrix}
  0 \\ 1 \\ 0 \\ 0
\end{bmatrix}, \quad
u =
\begin{bmatrix}
  0 \\ 0 \\ 1 \\ 0
\end{bmatrix}, \quad u = \begin{bmatrix}
  0 \\ 0 \\ 0 \\ 1
\end{bmatrix}
\]
Thus there are still negative energy solutions; we get around this
using the Exclusion Principle. The Dirac equation describes particles
with spin, and so this seems appropriate. This reasoning leads to the
idea of a Dirac Sea; where all of the negative energy levels are
already filled, and so the electon cannot fall into a negative
state. If a particle is moved up it leaves a hole in the sea, which we
interpret as an antiparticle. This argument, however, doesn't hold up
for Bosons.

\subsection{A general soultion}
\label{sec:general-soultion}
We have
\[ \implies E
\begin{bmatrix}
  \chi \\ \phi
\end{bmatrix}
=
\begin{bmatrix}
  m & \vec{\alpha} \vdot \vec{p} \\ \vec{\sigma} \vdot \vec{p} & -m
\end{bmatrix}
\begin{bmatrix}
  \chi \\ \phi
\end{bmatrix}
\]
So
\begin{equation*}
  \chi = \frac{\vec{\sigma} \vdot \vec{p}}{E-m} \phi, \qquad \phi = \frac{\vec{\sigma} \vdot \vec{p}}{E+m} \chi
\end{equation*}
These are compatible:
\begin{equation*}
  \phi = \qty( \frac{\vec{\sigma} \vdot \vec{p}}{E+m}) \qty( \frac{\vec{\sigma} \vdot \vec{p}}{E-m}) \phi = \frac{\abs{\vec{p}}^2}{E^2-m^2} \phi = \phi
\end{equation*}
Since $E^2 = m^2 + \abs{p}^2$, and $\sigma_i \sigma_j = \delta_{ij} +
i \epsilon_{ijk} \sigma_k \implies (\vec{\sigma} \vdot \vec{p})^2 =
\abs{\vec{p}}^2 + i (\vec{p} \cp \vec{p}) \vdot \vec{\sigma} =
\abs{\vec{p}}^2$.  We need to choose a basis for the solutions, the
simplest is
\begin{equation}
  \label{eq:90}
  \xi^{(1)} =
  \begin{bmatrix}
    1 \\ 0
  \end{bmatrix}, \qquad
\xi^{(2)} =
\begin{bmatrix}
  0 \\ 1
\end{bmatrix}
\end{equation}
Then the positive energy solutions, $\psi^{(1)}$ and $\psi^{(2)}$, to
\[ \phi = \frac{\vec{\sigma} \vdot \vec{p}}{E+m} \chi \]
are
\[ \chi = \xi^{(s)}, \qquad \phi = \frac{\vec{\sigma} \vdot \vec{p}}{E+m} \xi^{(s)} \]
With
\[ \psi^{(s)}(x) = u^{(s)} e^{-ip \vdot x} = \sqrt{E+m}
\begin{bmatrix}
  \xi^{(s)} \\ \frac{\vec{\sigma} \vdot \vec{p}}{E+m} \xi^{(s)}
\end{bmatrix}
e^{-ip \vdot x}
\]
The negative energy solutions, $\psi^{(3)}$ and $\psi^{(4)}$ to
\[ \phi = \frac{\vec{\sigma} \vdot \vec{p}}{E-m} \chi \]
are then 
\[ \phi = \xi^{(s)}, \qquad \chi = \frac{\vec{\sigma} \vdot \vec{p}}{E-m} \xi^{(s)} \]
With
\[ \psi^{(s+2)}(x) = u^{(s+2)} e^{-ip \vdot x} = \sqrt{-E+m}
\begin{bmatrix}
 \frac{\vec{\sigma} \vdot \vec{p}}{E-m} \xi^{(s)} \\   \xi^{(s)}
\end{bmatrix}
e^{-ip \vdot x}
\]

\subsection{Orthogonality and completeness}
\label{sec:orth-compl}

With the normalisation of $2E$ particles per unit volume,
\[ \hcon{u^{(r)}} u^{(s)} = 2 E \delta^{rs}, \qquad \hconn{v^{(r)}} v^{(s)} = 2E \delta^{rs} \]
which is a statement of orthogonality. The completeness relations are 
\begin{align*}
  \sum_{s=1,2} u^{(s)}(p) \bar{u}^{(s)} (p) &= \ps + m \\
\sum_{1,2} v^{(s)}(p) \bar{v}^{(s)}(p) &= \ps - m
\end{align*}
with $\bar{u} = \hcon{u} \gamma^0$.

\section{The Dirac Lagrangian}
\label{sec:dirac-lagrangian}

The Lagrangian of the Dirac field has the form
\begin{align}
  \label{eq:91}
  \Lag &= \bar{\psi} \qty( i \gamma^{\mu} \partial_{\mu}) \psi \nonumber\\
&= \bar{\psi}_i \qty( i \qty[ \gamma^{\mu}]_{ij} \partial_{\mu} - m \delta_{ij} ) \psi_j
\end{align}

There are two sets of Euler-Lagrange equations, one for $\psi$ and one
for $\bar{\psi}$.

\[ \pdv{\Lag}{\psi_i} - \pd{\mu} \qty(\pdv{\Lag}{(\pd{\mu}\psi_i)} )=0 \]
\[ \pdv{\Lag}{\bar{\psi}_i} - \pd{\mu} \qty(\pdv{\Lag}{(\pd{\mu}\bar{\psi}_i)} )=0 \]

Taking the $\bar{\phi}$ equation, so,
\[ \pdv{\Lag}{\bar{\psi}_i} = \qty( i [\gamma^{\mu}]_{ij} \pd{\mu} - m \delta_{ij} ) \psi_j \]
and
\[ \pdv{\Lag}{(\pd{\mu} \bar{\psi}_i)} = 0 \]
and noting that $\bar{\psi} \equiv \hcon{\psi}_i \gamma^0$, then we arrive at the equation of motion
\[ \qty( i \gamma^{\mu} \pd{\mu} - m) \psi = 0 \]
The other set of E-L equations gives the equation for antiparticle,
\[ i (\pd{\mu} \bar{\psi}) \gamma^{\mu} + m \bar{\psi} = 0 \] We can
introduce some new notation here, to indicate the direction in which
the derivative operator acts. The left-associative derivative is then
$\overleftarrow{\pd{\mu}}$, while the right-associative equivalent is
$\overrightarrow{\pd{\mu}}$. Thus, the second equation can be rewritten
\[ ( i \gamma^{\mu} \lpd{\mu} - m) \psi = 0 \]

The Dirac Lagrangian appears to treat the two fields in an
antisymmetric manner, but in principle each is as fundamental as the
other, so we can then re-write the Lagrangian, 
\[ \Lag = \bar{\psi}(i \gamma^{\mu} \pd{\mu}-m) \psi = \pd{\mu} \qty(
i \bar{\psi} \gamma^{\mu} \psi) - i \qty( \pd{\mu} \bar{\psi} )
\gamma^{\mu} \psi - m \psi \bar{\psi} \] where the first term in the
right-hand-side is zero, since it's a total derivative.
We can even write it
\begin{align*}
  \Lag &= \half \bar{\psi} i \gamma^{\mu} \pd{\mu}\psi - \half i (\pd{\mu}\bar{\psi}) \gamma^{\mu} \psi - m \bar{\psi} \psi \\ 
&= \half \bar{\psi} (i \gamma^{\mu} \rpd{\mu} - m) \psi - \half \bar{\psi} ( i \gamma^{\mu} \lpd{\mu} + m)\psi
\end{align*}

\section{Spin and angular momentum}
\label{sec:spin-angul-moment}

The angular momentum of a particle is 
\[ \vec{L} = \vec{r} \cp \vec{p} \] If this commutes with the
Hamiltonian the angular momentum is conserved, but we find
\begin{equation}
  \label{eq:92}
  \comm{H}{\vec{L}} = \comm{\vec{a}\vdot \vec{p}}{\vec{r} \cp \vec{p}} = - i \vec{\alpha} \cp \vec{p} \neq 0
\end{equation}
Clearly angular momentum is not conserved. Defining a new quantity, the instrinsic angular momentum (or spin),
\begin{equation}
  \label{eq:93}
  \vec{\Sigma} =
  \begin{bmatrix}
    \vec{\sigma} & 0 \\ 0 & \vec{\sigma}
  \end{bmatrix}
  = -i \alpha_1 \alpha_2 \alpha_3 \vec{\alpha}
= - i \gamma_1\gamma_2\gamma_3 \vec{\gamma}
\end{equation}
Then
\begin{equation}
  \label{eq:94}
  \comm{H}{\vec{\Sigma}} = \comm{\vec{\alpha} \vdot \vec{p}}{-i \alpha_1 \alpha_2 \alpha_3 \vec{\alpha}} = 2 i \vec{\alpha} \cp \vec{p}
\end{equation}
It's therefore clear that we can define a quantity
\begin{equation}
  \label{eq:95}
  \vec{J} = \vec{L} + \half \vec{\Sigma}
\end{equation}
which is conserved. This is the total angular momentum.

The basis spinors are eigenvectors of 
\[ \half \Sigma^3 = \half
\begin{bmatrix}
  1&0&0&0\\0&-1&0&0\\0&0&1&0\\0&0&0&-1
\end{bmatrix}
\]
with eigenvalues $\pm\half$.
This explains the four degrees of freedom:
\begin{itemize}
\item Positive energy; spin up
\item Positive energy, spin down
\item Negative energy, spin up
\item Negative energy, spin down
\end{itemize}

\section{Helicity of massless fermions}
\label{sec:helic-massl-ferm}

If the mass of the field is zero the wave equation simplifies to
\[ E
\begin{bmatrix}
  \chi \\ \phi
\end{bmatrix}
= 
\begin{bmatrix}
  0 & \vec{\sigma} \vdot \vec{p} \\ \vec{\sigma} \vdot \vec{p} & 0
\end{bmatrix}
\begin{bmatrix}
  \chi \\ \phi
\end{bmatrix}
\]
So
\[ E \phi = \vec{\sigma} \vdot \vec{p} \chi, \quad E \chi = \vec{\sigma} \vdot \vec{p} \phi \]
Then defining 
\[ \Psi~{{R,L}} = \half (\chi\pm\phi) \]
We have
\[ E \Psi~R = \vec{\sigma} \vdot \vec{p} \Psi~R, \quad E \Psi~L = -
\vec{\sigma} \vdot \vec{p} \Psi~L \] With $\Psi~{R,L}$ denoted the
\emph{Weyl spinors}, which are completely independent, and so can be
considered as two independent particles. Each of these is an eigenstate of the operator
\[ \frac{\vec{\sigma} \vdot \vec{p}}{\abs{\vec{p}}} = \frac{\vec{\sigma} \vdot \vec{p}}{E} \] with eigenvalues $\pm 1$.

For the full Dirac spinor the \emph{helicity} operator is defined as
\begin{equation}
  \label{eq:96}
  \Op{h} = \frac{\vec{\Sigma} \vdot \vec{p}}{\abs{\vec{p}}}
\end{equation}
which is the component of spin in the direction of the motion.  If the
eigenvalue of the helicity is $+\half$ the particle is right-handed,
and if it is $-\half$ it is left-handed. An antiparticle will have the
opposite helicity to an otherwise identical particle, since its
momentum has an opposite direction.

We can project a particular helicty from a Dirac spinor using gamma matrices. Defining
\begin{equation}
  \label{eq:97}
  \gamma^5 = i \gamma^0 \gamma^1 \gamma^2 \gamma^3 =
  \begin{bmatrix}
    0&\vec{1}\\\vec{1}&0
  \end{bmatrix}
\end{equation}
Then we can define projection operators,
\begin{equation}
  \label{eq:98}
  P~{R,L} = \half(1\pm\gamma^5) = \half
  \begin{bmatrix}
    \vec{1} & \pm \vec{1} \\ \pm \vec{1} & \vec{1}
  \end{bmatrix}
\end{equation}

The spinor $P~Lu$will be left-handed, but $P~R u$ will be
right-handed.

\section{Weyl representation}
\label{sec:weyl-representation}

The \emph{Weyl} or \emph{chiral representation} of the gamma matrices
makes the relation to spin more explicit. In this representation
\[ \gamma^0 = \begin{bmatrix}  0&\vec{1}\\\vec{1}&0 \end{bmatrix},
\quad \vec{\gamma} = \begin{bmatrix} 0 & \vec{\sigma} \\ - \vec{\sigma} & 0 \end{bmatrix}, 
\quad \gamma^5 = 
\begin{bmatrix}  -\vec{1} & 0 \\ 0 & \vec{1}\end{bmatrix} \]

The projection operator in this representation then looks like
\begin{subequations}
\begin{equation}
  \label{eq:99}
  P~L = \half ( 1 -\gamma^5)=  \begin{bmatrix}\vec{1}&0\\0&0  \end{bmatrix}
\end{equation}
\begin{equation}
  \label{eq:100}
  P~R = \half (1+\gamma^5)=
  \begin{bmatrix}
    0&0\\0&\vec{1}
  \end{bmatrix}
\end{equation}
\end{subequations}
Thus the left-handed Weyl spinor is the upper part, and the right
handed the lower part of the Dirac spinor. Thus
\begin{equation}
  \label{eq:101}
  P~R u =   \begin{bmatrix}0&0\\0&\vec{1} \end{bmatrix}
  \begin{bmatrix} \Psi~L \\ \Psi~R  \end{bmatrix} =
  \begin{bmatrix}
    0 \\ \Psi~R
  \end{bmatrix}
\end{equation}

\section{The weak interaction}
\label{sec:weak-interaction}

The weak interaction only acts on left-handed particles. Parity
transformations leave spin unchanged, but will turn left-handed
particles into right-handed particles, and so weak interactions must
be parity violating.

It's also worth noting that helicity is only conserved for massless
particles, as a massive particle therewill always be a boost into a
higher velocity frame, which will invert the momentum's
direction. This causes the helicity to flip.

The Dirac Lagrangian has the form
\[ m \bar{\psi} \psi = m (\hcon{\Psi}~L \hcon{\Psi}~R)
\begin{bmatrix}  0&\vec{1}\\\vec{1}&0 \end{bmatrix}
\begin{bmatrix} \Psi~L \\ \Psi~R \end{bmatrix}
= m \qty( \hcon{\Psi}~L \Psi~R + \hcon{\Psi}~R \Psi~L )
\]
in the Chiral representation. The mass terms mix the left and
right-handed states, and so the weak interaction cannot act on massive
particles. 

\section{The Higgs field}
\label{sec:higgs-field}

The solution to this problem is to introduce a new \emph{Higgs} field
which couples left-handed particles to right-handed ones, giving them
an effective mass. The Lagrangian has the form
\begin{equation}
  \label{eq:102}
  \Lag~{higgs} \supset  Y \bar{\psi}~L \vdot \phi \psi~R
\end{equation}
If the vacuum of the system contains a non-zero amount of this field,
so $\ev{\phi} \neq 0$, then we can generate a mass $Y\ev{\phi}$.

\section{Symmetries of the Dirac equation}
\label{sec:symm-dirac-equat}

\subsection{Lorentz transformation}
\label{sec:lorentz-transf}

Under a Lorentz transform, $x'^{\mu} = \Lambda^{\mu}_{\nu} x^{\nu}$,
so
\[ \pd{\mu} \to \pd{\mu}' = [\Lambda^{-1}]_{\mu}^{\nu} \pd{\nu}, \quad \psi(x) \to \psi'(x') = S \psi(x) \]
Now,
\[ (i \gamma^{\mu} \pd{\mu} - m)\psi(x)=0 \to (i \gamma^{\mu}
[\Lambda^{-1}]^{\nu}_{\mu} \pd{\nu} - m) S \psi(x) = 0\]
Pre-multiplying by $S^{-1}$, then
\[ \qty( i S^{-1} \gamma^{\mu} S [\Lambda^{-1}]^{\nu}_{\mu} \pd{\nu}-m) \psi(x) = 0 \]
and so
\[ S^{-1} \gamma^{\mu} S = \Lambda^{\mu}_{\nu} \gamma^{\nu} \]

Now, suppose we have an infintessimal proper transformation,
\[ \Lambda^{\mu}_{\nu} = g^{\mu}_{\nu}+ \omega^{\mu}_{\nu} \]
with the latter component anti-symmetric. Now, writing 
\[ S = 1 + \frac{i}{4} \sigma_{\mu \nu} \omega^{\mu \nu} \]
which is a parameterisation, so
\[ \gamma^{\mu} + \omega^{\mu}_{\nu} \gamma^{\nu} = (1-\frac{i}{4} \sigma_{\alpha \beta} \omega^{\alpha \beta}) \gamma^{\mu} (1+ \frac{i}{4} \sigma_{\sigma \rho} \omega^{\sigma \rho}) \]
Thus
\[ 2 i \omega^{\alpha\beta} \qty(\delta^{\mu}_{\alpha} \gamma_{\beta}
- \delta^{\mu}_{\beta} \gamma_{\alpha}) =
\comm{\gamma^{\mu}}{\sigma_{\alpha \beta}} \omega^{\alpha \beta} \]
(ignoring terms above the order of $\omega^2$), and we eventually reach
\[ \sigma_{\mu \nu} = \frac{i}{2} \comm{\gamma_{\mu}}{\gamma_{\nu}} \]
This is how a fermionic field transforms under a Lorentz boost.

The adjoint field transforms as
\[ \bar{\psi} = \hcon{\psi} \gamma^0 \to \hcon{\psi} \hcon{S} \gamma^0
= \hcon{\psi} \gamma^0 S^{-1} = \bar{\psi} S^{-1} \] since $\hcon{S}
\gamma^0 = \gamma^0 S^{-1}$, and so the quantity $\bar{\psi} \psi$ is
invariant. Since we have an invariant, we must have an associated
current, and so
\begin{equation}
  \label{eq:103}
  j^{\mu} = \bar{\psi} \gamma^{\mu} \psi \to \bar{\psi} S^{-1} \gamma^{\mu} S \psi = \Lambda^{\mu}_{\nu} \bar{\psi} \gamma^{\nu} \psi
\end{equation}

\begin{subequations}
  \begin{description}
  \item[scalars] \[ \bar{\psi} \psi \to \bar{\psi} \psi \]
  \item[pseudoscalars] \[\bar{\psi} \gamma^5 \psi \to \det(\Lambda) \bar{\psi} \gamma^5 \psi \]
  \item[vectors] \[ \bar{\psi} \gamma^{\mu} \psi \to \Lambda^{\mu}_{\nu} \bar{\psi} \gamma^{\nu} \psi \]
  \item[axial vectors] \[ \bar{\psi} \gamma^{\mu} \gamma^5 \psi \to \det(\Lambda) \Lambda^{\mu}_{\nu} \bar{\psi} \gamma^{\nu} \gamma^5 \psi \]
  \item[tensors] \[ \bar{\psi} \sigma^{\mu \nu} \gamma^5 \psi \to \Lambda^{\mu}_{\alpha} \Lambda^{\nu}_{\beta} \bar{\psi} \sigma^{\alpha \beta} \gamma^5 \psi \]
  \end{description}
\end{subequations}

\subsection{Charge conjugation, parity, and time-reversal}
\label{sec:charge-conj-parity}

Parity: As $t\to t$, $\vec{x} \to - \vec{x}$ \\
Charge conjugation: As $\psi \to \psi~c \equiv C \tr{\bar{\psi}}$ \\
Time reversal: As $t \to - t$, $\vec{x} \to \vec{x}$, and so $\psi(t,
\vec{x}) \to \psi~T(-t, \vec{x}) = T \psi^{*}(t, \vec{x})$.

The corresponding transformations are
\begin{subequations}
\begin{align}
  \label{eq:104}
  C: C \tr(\bar{\psi})(t,\vec{x}) &= i \gamma^2 \gamma^0 \tr(\bar{\psi})(t,\vec{x}) \\
P: P \psi(t, \vec{x}) &= \gamma^0 \psi(t, \vec{x}) \\
T: T \psi^{*}(t, \vec{x}) &= i \gamma^1 \gamma^3 \psi^{*}(t, \vec{x})
\end{align}
\end{subequations}

\begin{bigderiv*}
  \begin{align*}
    H & = \int T^{00} \dd[3]{x} = \int i \hOp{\psi} \pu{0} \Op{\psi} \dd[3]{x} \\
&= \int \dd[3]{x} \int \nm{p} \frac{\dd[3]{q}}{2(2 \pi)^3} \sum_{s,s'} \qty( \hOp{a}^{(s)} (\vec{p}) \hcon{u}^{(s)}(p) e^{ip\vdot x} +  \hOp{b}^{(s)} (\vec{p}) \hcon{v}^{(s)}(p) e^{-ip\vdot x} ) \\ 
& \qquad \qquad \qquad \qquad \qquad \times
 \qty( \hOp{a}^{(s')} (\vec{q}) \hcon{u}^{(s')}(p) e^{iq\vdot x} +  \hOp{b}^{(s')} (\vec{p}) \hcon{v}^{(s')}(p) e^{-iq\vdot x} ) 
\intertext{There are three orthogonality relations which simplify this,
\[ \hcon{u}^{(r)} v^{(s)} = 2E \delta^{rs} \qquad \hcon{u}^{(r)} u^{(s)} = 2 E \delta^{rs}  \qquad \hcon{u}^{(r)}v^{(s)} = 0 \]
}
H&= \int \nm{p} E(\vec{p}) \sum_s \qty(\hOp{a}^{(s)}(\vec{p}) \Op{a}^{(s)}(\vec{p}) - \Op{b}^{(s)}(\vec{p}) \hOp{b}^{(s)}(\vec{p}) ) \\
&= \int \nm{p} E(\vec{p}) \sum_s
 \qty(\hOp{a}^{(s)}(\vec{p}) \Op{a}^{(s)}(\vec{p}) - \hOp{b}^{(s)}(\vec{p}) \Op{b}^{(s)}(\vec{p}) - \comm{\Op{b}^{(s)}(\vec{p})}{\hOp{b}^{(s)}(\vec{p})})
  \end{align*}
We can't postulate a commutation relation for the creation and annihilation operators, as even after subtracting an infinite complex number the antiparticles still give a negative energy contribution. So the energy could always be reduced by adding more antiparticles, making an unstable system. We need to make their contributions positive, so must turn to anticommutation relations.
\caption{The derivation of the Hamiltonian for the free Dirac field.}
\label{der:dirac-hamiltonian}
\end{bigderiv*}


\section{Second quantisation}
\label{sec:second-quantisation-1}

Thus far the Dirac equation has only been framed in terms of quantum
mechanics. To transition to QFT we need to convert the fields to
operators. We proceed as for the scalar field.
\begin{align*}
  \Op{\psi}(x) &= \int \nm{p} \sum_s \big( \Op{a}^{(s)}(\vec{p}) u^{(s)}(p) e^{-ip\vdot x} \\ & \qquad \qquad \qquad+ \hOp{b}^{(s)}(\vec{p}) v^{(s)}(p) e^{ip\vdot x} \big)
\end{align*}
From here onwards the hat notation for operators is omitted for neatness.

\section{The energy momentum tensor}
\label{sec:energy-moment-tens-1}

From Noether's theorem the energy momentum tensor is
\begin{equation}
  \label{eq:105}
  T^{\mu \nu} = \pdv{\Lag}{(\pd{\mu}\psi)} \pu{\nu} \psi + \pu{\nu} \bar{\psi} \pdv{\Lag}{(\pd{\mu}\bar{\psi})} - g^{\mu \nu} \Lag = \bar{\psi} i \gamma^{\mu} \pu{\nu} \psi
\end{equation}
where the Dirac equation was used to remove the last term. Thus the Hamiltonian is
\begin{equation}
  \label{eq:106}
  T^{00} = i \hcon{\psi} \pu{0} \psi \iff H = \int i \hcon{\psi} \pu{0} \psi \dd[3]{x}
\end{equation}

\section{Anticommutation}
\label{sec:anticommutation}

We can postulate some anticommutation relations for the Dirac field,
\begin{subequations}
  \begin{align}
    \label{eq:107}
    \acomm{\Op{a}^{(r)}(\vec{k})}{\hOp{a}^{(s)}(\vec{p})} &  =\acomm{\Op{a}^{(r)}(\vec{k})}{\hOp{a}^{(s)}(\vec{p})} \nonumber\\
&= \delta^{rs} (2 \pi)^3 2 E(\vec{k}) \delta^3(\vec{k}-\vec{p}) \\
 \acomm{\Op{a}^{(r)}(\vec{k})}{\Op{a}^{(s)}(\vec{p})} & =  \acomm{\Op{a}^{(r)}(\vec{k})}{\Op{b}^{(s)}(\vec{p})} \nonumber \\= \cdots &= 0
  \end{align}
\end{subequations}
and then the anti-commutation relations for the fields are
\begin{subequations}
  \begin{align}
    \label{eq:108}
    \acomm{\Op{\phi}^{(r)}(\vec{x}, t)}{\Op{\pi}^{(s)}(\vec{y},t)} &= i \delta^{rs} \delta^3(\vec{x}-\vec{y}) \\
\acomm{\Op{\psi}^{(r)}(\vec{x},t)}{\Op{\psi}^{(s)}(\vec{y},t)} &= \acomm{\Op{\pi}^{(r)}(\vec{x},t)}{\Op{\pi}^{(s)}(\vec{y},t)} = 0
  \end{align}
\end{subequations}
This means the state of two fermions is antisymmetric under
interchange, which is the origin of the Fermi-Dirac statistics, and
the Pauli exclusion principle.

\section{The Dirac propagator}
\label{sec:dirac-propagator}

The Greens function for the Dirac equation satisfies
\[ S~F(x-y) \equiv G_2(x,y) \]
so
\begin{equation}
  \label{eq:109}
  (i \ds - m) S~F (x-y) = i \delta^{(4)}(x-y)
\end{equation}
and then writing
\begin{equation}
  \label{eq:110}
  S~F(x-y) = \int \nm{p} \tilde{S}~F(p) \exp(-ip \vdot(x-y) )
\end{equation}
premultiplying this by $(i \ds -m)$, and remembering that
\[ \int \nm{p} e^{-ip \dot (x-y)} = \delta^{(4)}(x-y) \]
then
\begin{equation}
  \label{eq:111}
  \tilde{S}~F(p) = \frac{i}{\ps -m } = \frac{i(\ps + m)}{p^2 - m^2}
\end{equation}

Just as with scalar fields the propagator is a time-ordered product of
fields, but we must be careful with the signs. For a fermion field
\begin{equation}
  \label{eq:112}
  \tOrd \psi^{(r)}(x) \bar{\psi}^{(s)}(y) = 
  \begin{cases}
    \psi^{(r)}(x) \bar{\psi}^{(s)}(y) & \text{for $x^0 > y^0$} \\
- \bar{\psi}^{(s)}(y) \psi^{(r)}(x) & \text{for $x^0 < y^0$}
  \end{cases}
\end{equation}
And with this definition,
\begin{fequation}[Dirac propagator]
  \label{eq:113}
  S~F^{rs}(x-y) = \bra{0} \tOrd \psi^{(r)}(x) \bar{\psi}^{(s)}(y) \ket{0}
\end{fequation}


%%% Local Variables: 
%%% mode: latex
%%% TeX-master: "../project"
%%% End: 


\chapter{Quantum Electrodynamics}
\label{cha:quant-electr}

\chapter{Quantum Electrodynamics Scattering}
\label{cha:quant-electr-scatt}

\appendices \twocolumn

\chapter{Geometry \& Linear Algebra}
\label{cha:linearalgebra}

Modern formualtions of mechanics and relativity require an extensive
quantity of linear algebra, often using constructions such as tensors.

\section{Vectors and Covectors}
\label{sec:vectors-covectors}

\begin{definition}[Vector]
  A vector is an object which is an element 
      \marginpar{$v^\alpha$}
  of a vector space; in terms of geometry this is an entity with both a concept of magntiude and of direction.
\end{definition}

\begin{definition}[Covector]

  A covector is the dual object of a vector, 
    \marginpar{$v_\alpha$}
    and maps a vector space to a scalar field.
\end{definition}

It is possible to visualise a vector as an arrow with length and
direction. In this case its corresponding covector is a means of
assigning its length, for example a series of evenly-spaced planes
through which it passes, or the notches on a tape measure.\\
Conventional notation represents the components of a vector as a
superscript, e.g. for the vector $v$ the components are
$v^\alpha$. For a covector they are subscript, $v_\alpha$.

\section{Tensors}
\label{sec:tensors}

It is possible to use multiplication by a scalar to change the
magnitude of a vector, but what if we wish to change its direction?
This is where we require tensors, which are objects composed out of
vectors.

Let's return to the concept of a vector field. A vector field is a
mathematical construction where every point in space has a vector
(think of it as an arrow) attached to it. Suppose now we want to
attach some more complicated construction to every point? For example,
what if we want a measure of the distortion of space at every point,
by attaching an ellipsoid to every point? This is where the concept of
a tensor field grows from a vector field.

\begin{definition}[Tensor]
  A $(k, l)$ \emph{tensor} is the map
  \[ T : \underbracket{ \vs{V^{*}} \otimes \cdots \otimes \vs{V^{*}}
  }_{k \text{ copies}} \otimes \overbracket{ \vs{V} \otimes \cdots
    \otimes \vs{V} }^{l \text{ times}} \to \mathbb{R}\] which is
  linear in all of its arguments.
\end{definition}

\section{Tensor Fields}
\label{sec:tensor-fields}

The concept of a tensor field is a two-stage concept. First there is the concept of the vector bundle.

\begin{definition}[Real Vector Bundle]
  This consists of
  \begin{enumerate}
  \item topological spaces, $X$, the base space, and $E$, the total
    space.
  \item a continuous surjection $\pi : E \to X$, the bundle projection,
  \item for every $x \in X$ the structure of a finite-dimensional real vector space on the fibre $\pi^{-1}(\set{x})$.
  \end{enumerate}
  where the compatibility condition is: for every point in $X$ there
  is an open neighbourhood $U$, a natural number $k$, and a
  homeomorphism,
  \[ \phi: U \times \mathbb{R}^k \to \pi^{-1}(U) \] such that for all
  $x \in U$,
  \begin{itemize}
  \item $( \pi \circ \phi )(x,v) = x$ for all vectors $v$ in
    $\mathbb{R}^k$, and
  \item the map $v \to \phi(x,v)$ is an isomorphism between the vector
    spaces $\mathbb{R}^k$ and $\pi^{-1}(\set{x})$.
  \end{itemize}
\end{definition}

An example of a vector bundle is a M\"obius strip, which is a line
bundle over a one-sphere $S^1$ (i.e. a circle). Every local point in
the strip looks like $U \times \mathbb{R}$, for $U$ an open arc which
includes the point. However, the total bundle is different from $S^1
\times \mathbb{R}$, which is a cylinder. This makes it locally
trivial, but the global geometry is more complex.\\
The fibre bundle can thus be seen as a vector space which is dependent
on some parameter in a manifold, $M$; in the case of the M\"obius strip, this is the
angular distance around the strip.

\subsection{Manifolds}
\label{sec:manifolds}

A \emph{manifold} is a topological space which resembles Euclidean
space around every point on it. In one dimension examples include
lines and circles, which are homeomorphic to $\mathbb{E}^1$, but
figures-of-eight are a counter-example, due to the crossing of the
lines. In two dimensions planes, spheres, and tori are examples, while
the Klein bottle is a counter-example. The local geometry of manifolds
is Euclidean---consider a football field, where it is possible to draw
straight lines and have a flat surface despite it being located on the
surface of a sphere, but the global geometry often is not. It is
possible to map regions of a manifold onto the Euclidean plane,
producing \emph{charts} by means of \emph{map projections}. In a
region which appears on two neighbouring charts a transformation will
be required to move between the two, and this is a \emph{transition
  map}.

Consider a point, $p$ on the manifold $M$. The \emph{tangent space} is
a vector space at the point $p$ which is at a tangent to the manifold
$M$, and can be denoted $T_{x}(M)$ or $T_xM$. The tangent bundle is
the disjoint union of all of the tangent spaces across the manifold,
and is denoted $TM$.

Now, at any given point a tensor field assigns a tensor to the space,
\[ \vs{V} \otimes \cdots \otimes \vs{V} \otimes \vs{V^{*}} \otimes
\cdots \otimes \vs{V^{*}}\] where $\vs{V}$ is the tangent space, and
$\vs{V^{*}}$ the cotangent space at the point.

%%% Local Variables: 
%%% mode: latex
%%% TeX-master: "../project"
%%% End: 



\chapter{Calculus}
\label{cha:calculus}

%
\section{Dirac Delta Function}
\label{sec:dirac-delta-function}

The dirac delta is a function which can be (non-rigorously) defined 
\begin{namedequation}
  \begin{equation}
    \label{eq:11}
    \delta(x-y) = 
    \begin{cases}
      \infty &\text{\quad for} x = y \\
      0 &\text{\quad for} x \neq y
    \end{cases}
  \end{equation}
\end{namedequation}
such that its integral is $1$:
\[ \int \delta(x-y) \dd{x} = 1 \]

It can be visualised as a limit of the Gaussian distribution,
\[ \delta(x-y) = \lim_{a \to 0} \frac{1}{ a \sqrt{\pi} } \exp( -
\frac{(x-y)^2}{a^2} \]
which is equivalent to
\[ \delta(x-y) = \frac{1}{2 \pi} \int_{- \infty}^{\infty} e^{i(x-y)k}
\dd{k} \]

\subsection{Useful Properties}
\label{sec:useful-properties}

Under integration the delta function has the property of picking out a
value of a function:

\begin{equation}
  \label{eq:12}
  \int \delta(x-y) f(x) \dd{x} = f(y)
\end{equation}

There are a number of other properties:
\begin{subequations}
\begin{align}
  \delta(ax) &= \frac{\delta(x)}{\abs{a}} \\
  \delta( g(x) ) &= \sum_i \frac{\delta(x - x_i)}{\abs{g^{\prime} (x_i)}} \\
  \delta(x^2 - a^2) &= \frac{1}{2 \abs{a}} ( \delta(x-a) + \delta(x+a) ) \\
  (x - y) \pdv{y} \delta(x-y) &= \delta(x - y)
\end{align}
\end{subequations}

There also exists a multi-dimensional form,
\begin{equation}
  \label{eq:13}
  \delta^4(x-y) = \delta(x^0 - y^0) \delta(x^1 - y^1) \delta(x^2 - y^2) \delta(x^3 - y^3)
\end{equation}

\section{Green's Functions}
\label{sec:greens-functions}

Consider a field which satisfies a differential equation of the form
\begin{equation}
  \label{eq:14}
  \Op{D} \phi(x) = \rho(x)
\end{equation}
where $\Op{D}$ is some differential operator.

For example, Poisson's equation is
\[ \vec{\nabla}^2 \phi(x) = \rho(x) \] Let the function $G(x,y)$ be a
solution of the same equation, with a point source at $x=y$,  so

\begin{align*}
  \Op{D} G(x,y) &= \delta(x-y) \\
\Op{D} \int G(x,y) \rho(y) \dd{y} &= \int \delta(x-y) \rho(y) \dd{y} = \rho(x)
\end{align*}
That is,
\[ \phi(x) = \int G(x,y) \rho(y) \dd{y} \]
is a solution to the original equation.

The function $G(x,y)$ is a Green's function. Green's functions convert
the problem of a differential equation into the problem of evaluating
an integral.

%%% Local Variables: 
%%% mode: latex
%%% TeX-master: "../project"
%%% End: 


\chapter{Special Relativity}
\label{cha:specialrelativity}

%%%%%%%%%%%%%%%%%%%%%%%%%%%%
% CHAPTER 0                %
%%%%%%%%%%%%%%%%%%%%%%%%%%%%
% Special Relativity       %
%%%%%%%%%%%%%%%%%%%%%%%%%%%%

Special relativity is the extension of classical mechanics to
extremely high energy situations, and is based around just two axioms.

\begin{definition}[Event]
  An event is something which happens at a specific place at a
  particular instant in time.
\end{definition}

\begin{definition}[Inertial Reference Frame]
  An inertial reference frame is a means of assigning a position to an
  event. They have no acceleration, such that an inertial reference
  fram is a reference frame with respect to which Newton's Second Law
  holds.
\end{definition}

\subsection{The axioms of Special Relativity}
\label{sec:sraxiom}

\begin{enumerate}
\item All inertial reference frames are equivalent for the performance of all physical experiments.
\item The speed of light has the same, constant value in all reference frames.
\end{enumerate}
The first axiom is more commonly known as the \emph{principle of relativity}.

\subsection{Minkowski Diagrams}
\label{sec:minkdiag}

Any event can be described by four coordinates, $(t;x,y,z)$, but, by
choosing an appropriate reference frame we can describe this with just
two coordinates, $(t;x)$; a diagram of this situation is shown in
figure \ref{fig:minksimple}.

\begin{figure}
\centering
\begin{tikzpicture}[scale=1.5]
\fill [muted-blue] (.5, .5) circle (0.05);
\draw[help lines, ->] (-1,-1) -- (1.7,-1) node [below] {space, $x$};
\draw[help lines, ->] (-1, -1) -- (-1, 1.2) node [right] {time, $t$};
\end{tikzpicture}
\caption{An event on a Minkowski diagram.}
\label{fig:minksimple}
\end{figure}

\subsection{Invariant Interval}
\label{sec:invint}

In Euclidean space there is a concept of \emph{distance} which is
invariant; $x^2 + y^2 + z^2$ is the same under all transformations.
In spacetime there is a generalisation of this concept in the
\emph{invariant interval}:
\begin{equation}
  \label{eq:invariantinter}
  s^2 = \Delta t^2 - \Delta x^2
\end{equation}
The value of $s^2$ provides a simplified understanding of causality.
\begin{itemize}
\item $s^2 > 0$---time-like separation.
\item $s^2 = 0$---null separation (light-like).
\item $s^2 < 0$---space-like separation.
\end{itemize}

\begin{figure} \centering
\begin{tikzpicture}[scale=1.7]
\draw (-1,-1) -- (1,1) node [above] {$s^2 = 0$};
\fill [muted-orange, opacity=0.4] (-1,-1) -- (1,1) -- (-1,1) --cycle;
\draw (-.5,0) node {$s^2 > 0$};
\fill [muted-green, opacity=0.4] (-1,-1) -- (1,-1) -- (1,1) --cycle;
\draw (.5,0) node {$s^2 < 0$};
\draw[help lines, ->] (-1,-1) -- (1.7,-1) node [below] {space, $x$};
\draw[help lines, ->] (-1, -1) -- (-1, 1.2) node [right] {time, $t$};
\end{tikzpicture}
\caption{An event on a Minkowski diagram.}
\label{fig:minksimple}
\end{figure}

\subsection{Lorentz Transformations}
\label{sec:lorentz}

Consider two frames moving relative to one another, $S$ and
$S^\prime$. The frames are in standard configuration, i.e.\ an event
in $S$ at $(0,0)$ also occurs at $(0,0)$ in $S^\prime$. To convert
from an event occuring in $S$ to one in $S^\prime$ we use a Lorentz
transform. These have the form
\begin{subequations}
\begin{align}
  x^\prime &= x \cosh \phi - t \sinh \phi \\
t^\prime &= -x \sinh \phi + t \cosh \phi
\end{align}
\end{subequations}
and with 
\[ \phi(v) = \tanh[-1](v) \]
These can also be written
\begin{subequations}
  \begin{align}
    t^\prime &= \gamma (t-vx) \\
    x^\prime &= \gamma (x-vt)
  \end{align}
\end{subequations}
\begin{subequations}
  \begin{align}
    t &= \gamma (t^\prime + v x^\prime) \\
    x &= \gamma (x^\prime + v t^\prime)
  \end{align}
\end{subequations}

\subsection{Adding velocities}
\label{sec:addingvel}

Since $e^{\pm \phi} = \cosh(\phi) \pm \sinh(\phi)$,
\begin{subequations}
  \begin{align}
    t^\prime - x^\prime &= e^{\phi} (t-x) \\
    t^\prime + x^\prime &= e^{-\phi} (t+x)
  \end{align}
\end{subequations}

\subsection{Proper Time}
\label{sec:proper-time}

In the frame $S$ there are events at $(0,0)$ and $(t,x)$. A clock is
moving at a constant speed $v$, so that it is present at both
events. Pick $S^{\prime \prime}$, the clock's rest frame; $t^{\prime
  \prime}$ is thus the reading on the clock. This is denoted $\tau$,
the proper time. In any other frame in standard configuration we can then write
\begin{equation}
  \label{eq:1}
  \tau = \gamma(t-vx)
\end{equation}
The proper time will be agreed upon by all observers, and as such is a
Lorentz scalar. Since the velocity is $v = \frac{x}{t}$,
\[ t^2 - x^2 = \tau^2 = t^{\prime 2} - x^{\prime 2} \]

\subsection{Four-vectors}
\label{sec:four-vectors}

In $4$-dimensional spacetime the prototype displacement vector has the form
\[ \qty( \Delta t , \Delta x, \Delta y, \Delta z ) \]
and the transformation from one $4$-vector to another is 
\begin{equation}
  \label{eq:2}
  \begin{bmatrix}
    \Delta t \\ \Delta x \\ \Delta y \\ \Delta z
  \end{bmatrix}
=
\begin{bmatrix}
  \gamma & + \gamma v & 0 & 0 \\
+ \gamma v & \gamma &0 & 0 \\
0 & 0 & 1 & 0 \\
0 & 0 & 0 & 1
\end{bmatrix}
\begin{bmatrix}
  \Delta t^{\prime} \\ \Delta x^{\prime} \\ \Delta y^{\prime} \\ \Delta z^{\prime}
\end{bmatrix}
\end{equation}

\begin{equation}
  \label{eq:2}
\begin{bmatrix}
  \Delta t^{\prime} \\ \Delta x^{\prime} \\ \Delta y^{\prime} \\ \Delta z^{\prime}
\end{bmatrix}
=
\begin{bmatrix}
  \gamma & - \gamma v & 0 & 0 \\
- \gamma v & \gamma &0 & 0 \\
0 & 0 & 1 & 0 \\
0 & 0 & 0 & 1
\end{bmatrix}
  \begin{bmatrix}
    \Delta t \\ \Delta x \\ \Delta y \\ \Delta z
  \end{bmatrix}
\end{equation}

It is more normal to denote each component of the 4-vector as $x^n$,
$n \in \{0,1,2,3 \}$, or collectively, $x^{\mu}$.

\subsection{Four-vector inner products}
\label{sec:four-vector-inner}

The inner product of two four-vectors is somewhat less straightforward
that a three-vector scalar product.

\begin{definition}[Four-vector inner product]
  \begin{equation}
    \label{eq:3}
    \braket{A}{B} = \sum_{\mu, \nu} \eta_{\mu \nu} A^{\mu} B^{\nu}
  \end{equation}
  for $\eta_{\mu \nu} = \diag(1, -1, -1, -1)$ which is the metric
  tensor for a non-accelerating reference frame.
\end{definition}

\subsection{Velocity and Acceleration Vectors}
\label{sec:veloc-accel-vect}

The displacement 4-vector, $\Delta x^{\mu}$, transforms properly, as
does the infinitessimal displacement, $\dd{x^{\mu}}$. The proper time,
$\tau$, is the Lorentz scalar, so we can divide each component of
$\dd{x^{\mu}}$ by $\dd{\tau}$ and get a new vector, the velocity,
\[ U = \qty( \dv{x^0}{\tau}, \dv{x^1}{\tau}, \dv{x^2}{\tau}, \dv{x^3}{\tau} ) \]
and likewise, the acceleration,
\[ A = \qty( \dv[2]{x^0}{\tau}, \dv[2]{x^1}{\tau}, \dv[2]{x^2}{\tau}, \dv[2]{x^3}{\tau} ) \]

Now, since $t = \gamma \tau$,
\[ U^{\mu} = \qty( \gamma c , \gamma \vec{v} ) \]


\subsection{Momentum}
\label{sec:momentum}

Given the four-velocity we can find a four-momentum as well.
\[ p^{\mu} = (\gamma m c, \gamma m \vec{v} ) = \qty(\frac{E}{c}, \vec{p} )\]
The norm of any four-vector will be Lorentz invariant, so 
\[ p^2 = \frac{E^2}{c^2} - \vec{p}^2 \]
In the particle's rest frame,
\[ p^{\mu} = (mc, \vec{0}) \therefore p^2 = m^2 c^2 \]
and since it is Lorentz invariant, it must be the case in all frames.
Thus
\[ \frac{E^2}{c^2} - \vec{p}^2 = m^2 c^2  \quad \therefore \quad E^2 - \vec{p}^2 c^2 = m^2 c^4 \]

\subsection{Invariant Mass}
\label{sec:invariant-mass}

In scattering interactions the \emph{invariant mass} is a useful
quantity. Given a pair of particles with momenta $p_1$ and $p_2$, the
squared invariant mass, when working in natural units, is
\begin{align}
m_{12}^2  & = (p_1 + p_2)^2 \nonumber                                                                      \\
         & = (E_1+E_2, \vec{p}_1 + \vec{p}_2)^2 \nonumber                                                 \\
         & = (E_1+E_2)^2 - (\vec{p}_1 + \vec{p}_2) \cdot (\vec{p}_1 + \vec{p}_2) \nonumber                \\
         & = E_1^2 - \vec{p}_1^2 + E_2^2 - \vec{p}_2^2 + 2E_1 E_2 - 2 \vec{p}_1 \cdot \vec{p}_2 \nonumber \\
         & = m_1^2 + m_2^2 + 2(E_1 E_2 - \vec{p}_1 \cdot \vec{p}_2) 
\end{align}

\begin{example}[Two particles, one at rest]
  $\vec{p}_2 = \vec{0}$, and $E_2 = m_2$. Thus
\[ m_{12}^2 = m_1^2 + m_2^2 + 2 m_2 E_1 \]
\end{example}
\begin{example}[The Centre of Momentum Frame]
  The total 3-mometum in this frame is $\vec{0}$, so,
  \[ \vec{p}^{\text{com}}_1 = - \vec{p}^{\text{com}}_2 \]
then
\[ m_{12}^2 = \qty( E_1^{\text{com}} + E_2^{\text{com}} )^2 \]
\end{example}
Thus, the invariant mass is the total energy in the centre of mass
frame.

\subsection{Particle Collisions}
\label{sec:particle-collisions}

The vast majority of information about fundamental particles comes
from collision experiments. There are two types of collider used in physics,
\begin{description}
\item[Fixed Target] The target is at rest, while high-energy particles are accelerated and fired into it. Examples include the Bevatron, and the SLAC
\item[Colliding Beam] Two accelerated beams are focussed and fired
  into one another, so that the particles collide, usually in the
  centre of mass frame. The LHC is an example.
\end{description}
In all particle interactions the 4-momentum is conserved, so
\begin{align*} 
p^{\mu}_1 + p^{\mu}_2 &= p^{\mu}_3 + p^{\mu}_4 \\
(p_1 + p_2)^2 &= (p_3 + p_4)^2
\end{align*}
During an interaction the original particles are annihilated, and then
new particles are created from the resulting energy. As a result, for
a final state to be achieved there must be enough energy available to
create the appropriate quantity of mass. Defining the Lorentz invariant, 
\[ s = (\vec{p}_1 + \vec{p}_2)^2 \] For a desired pair of particles to
be produced with masses $m_3$ and $m_4$ we then require
\[ \sqrt{s} \ge m_3 + m_4 \]

\begin{example}[Creating two particles]
  We wish to create two particles, each with mass $2 \giga
  \electronvolt$ by colliding two protons (each $1 \giga
  \electronvolt$).\\
  The minimum energy in the case of a fixed-target experiment must
  come entirely from the proton in the beam, so
  \[ E_1 = \frac{m_{12}^2 - m_1^2 - m_2^2}{2 m_2} = \frac{(m_3+m_4)^2
    - m_1^2 - m_2^2}{2 m_2} \] thus, $E_1 = 7 \giga \electronvolt$.
  In the case of colliding beams both particles collide in the
  centre-of-mass frame, and so
  \[ E = \frac{m_1+m_2}{2} \ge \frac{m_3+m_4}{2} = 2 \giga
  \electronvolt \]
\end{example}
Clearly a much higher beam energy is required in the case of a fixed
target to get the same results, but the engineering challenges
involved with colliding two beams have, in the past, made fixed-target
experiments more feasible.




\chapter{Quantum Mechanics}
\label{cha:quantummechanics}


\section{The State Vector}
\label{sec:state-vector}

A quantum mechanical state can be completely described by a state
vector in a Hilbert Space. State vectors can be described using
bra-ket notation



%%% Local Variables: 
%%% mode: latex
%%% TeX-master: t
%%% End: 




\end{document}



%%% Local Variables: 
%%% mode: latex
%%% TeX-master: t
%%% End: 
