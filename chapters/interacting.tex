The interacting scalar field is motivated by the Lagrangian
\begin{equation}
  \label{eq:53}
  \Lag = \half \pd{\mu} \Op{\phi} \pu{\mu} \Op{\phi} 
  -\half m^2 \Op{\phi}^2 - \frac{\lambda}{4!} \Op{\phi}^4
\end{equation}
This is a real scalar field, and the only dimensionless coupling we
can add is $\phi^4$. The energy function from the energy-momentum
tensor is
\[ \Op{T}^{00} = \pdv{\Lag}{\pd{0} \Op{\phi}} \pu{0} \Op{\phi} -
\Op{\Lag} \] and from this we can generate a new Hamiltonian,
\begin{equation}
  \label{eq:53}
  \Op{H} = \Op{H}_0 + \Op{H}_{\rm int}
\end{equation}
where $\Op{H}_0$ is the Hamiltonian of the free scalar field. The new
part is
\begin{equation}
  \label{eq:54}
  \Op{H}~{int} = - \int \Op{\Lag}~{int} \dd[3]{x} = \int \frac{\lambda}{4!} \Op{\phi}^4(x) \dd[3]{x}
\end{equation}
Ultimately we want to develop this to be able to make predictions
about particle scattering which can then be tested by experiment. If
$\lambda$ is small enough we can use perturbation theory to calculate
scattering cross-sections.

\section{The Dirac interaction picture}
\label{sec:interaction-picture}

Previously 
\begin{equation}
  \label{eq:55}
  \Op{\phi}~H (\vec{x}, t) = e^{i \Op{H}(t-t_0)} \Op{\phi}~S (\vec{x}) e^{-i \Op{H}_0 (t-t_0)}
\end{equation}
with Heisenberg picture operators changing with time, but Schrodinger
ones constant. In principle this is all that is necessary;
$\Op{\phi}~H (\vec{x}, t_0) = \Op{\phi}~S(\vec{x})$ is the operator at
the beginning, and we know how it changes with time.

This is hard to solve, so we can define operators in the interaction picture,
\begin{equation}
  \label{eq:56}
  \Op{\mathcal{O}}~{I}(\vec{x}, t) = e^{i \Op{H}_0(t-t_0)} \Op{\mathcal{O}}~S(\vec{x}) e^{-i \Op{H}_0(t-t_0)}
\end{equation}
for $\Op{H}_0$ the Hamiltonian of the free theory.
Consider
\begin{equation}
  \label{eq:57}
  \Op{\phi}~I (\vec{x},t) = e^{i \Op{H}_0(t-t_0)} \Op{\phi}~S (\vec{x}) e^{-i \Op{H}_0(t-t_0)}
\end{equation}
we've solved this, since it's the field operator for the
non-interacting theory.
\begin{equation}
  \label{eq:58}
  \Op{\phi}~I (x) = \int \nm{k} \qty( \Op{a}(\vec{k}) e^{i k \vdot x} + \hOp{a} (\vec{k}) e^{i k \vdot x} )
\end{equation}
This is related to the field operator for the interacting theory in
the Heisenberg picture by
\begin{align*}
  \Op{\phi}~H (\vec{x}, t) &= e^{i \Op{H}(t-t_0)} \Op{\phi}~S (\vec{x}) e^{-i \Op{H}(t-t_0)} \\ 
&= e^{i \Op{H}(t-t_0)} \qty[ e^{-i \Op{H}_0(t-t_0)} \Op{\phi}~I e^{i \Op{H}_0(t-t_0)} ] e^{-i \Op{H}(t-t_0)} \\
&= \hOp{U}(t,t_0) \Op{\phi}~I (\vec{x}, t) \Op{U}(t, t_0)
\end{align*}
with the time evolution operator defined
\begin{fequation}[Time evolution operator]
  \label{eq:59}
  \Op{U}(t, t_0) = e^{i \Op{H}_0 (t-t_0)} e^{-i \Op{H}(t-t_0)}
\end{fequation}
The interacting and free Hamiltonians do not commute, and so
$\Op{U}(t, t_0) \neq e^{-i \Op{H}~{int}(t-t_0)}$, and the CBH
expansion must be used.  We can obtain a differential equation by
differentiating $\Op{U}$,
\begin{align*}
  i \pdv{\Op{U}}{t} &= i \qty( \pdv{t} e^{i \Op{H}_0(t-t_0)} ) e^{-i\Op{H}(t-t_0)} + i e^{\Op{H}_0(t-t_0)} \pdv{t} e^{-i \Op{H}(t-t_0)} \\
&= -e^{i \Op{H}_0(t-t_0)} \Op{H}_0 e^{-i \Op{H}(t-t_0)} + e^{i \Op{H}_0(t-t_0)} \Op{H} e^{-i \Op{H}(t-t_0)} \\
&= \underbracket{e^{i \Op{H}_0 (t-t_0)} \Op{H}~{int} e^{-i \Op{H} (t-t_0)}}_{\OP{H}~{int, I}} \times \underbracket{e^{i \Op{H}_0(t-t_0)} e^{{-i \Op{H}(t-t_0)}}}_{\Op{U}(t-t_0)}
\end{align*}
We now have the problem of solving $ i \pdv{t} \Op{U}(t-t_0) =
\Op{H}~{int,I} \Op{U}(t,t_0)$ for $\Op{U}(t_0, t_0)=1$, in integral form
\begin{equation}
  \label{eq:60}
  \Op{U}(t-t_0) = 1 - i \int_{t_0}^t \dd{t_1 \Op{H}~{int,I}(t_1) } \Op{U}(t_1, t_0)
\end{equation}
but this can clearly continue \emph{ad infinitum}, so
\begin{align*}
\Op{U} = \sum_{n=0}^{\infty} (-i)^n & \int_{t_0}^t \dd{t_1}  \\ & \int_{t_0}^{t_1} \dd{t_2} \dots \int_{t_0}^{t_{n-1}} \dd{t_n} \Op{H}~{int,I}(t_1) \dots \Op{H}~{int,I}(t_n)
\end{align*}
we can simplify this by changing the area we integrate over, when $n=2$ for example,
\begin{align*}
  \int_{t_0}^t \dd{t_1} \int_{t_0}^{t_1} \dd{t_2} & \Op{H}~{int,I}(t_1) \Op{H}~{int,I}(t_2) \\
&= \half \int_{t_0}^t \dd{t_1} \dd{t_2} \Op{T} \qty{\underbracket{\Op{H}~{int,I} (t_1) \Op{H}~{int,I}(t_2)}_{{\text{symmetric under } t_1 \leftrightarrow t_2}} }
\end{align*}
Then
\begin{align*}
  \Op{U}(t,t_0) &= \sum_{n=0}^{\infty} \frac{(-i)^n}{n!} \int_{t_0}^t \dd{t_1} \cdots \dd{t_n} \Op{T}\qty{\Op{H}~{int,I}(t_1) \cdots \Op{H}~{int,I}(t_n)} \\ &= \Op{T}e^{\qty( -i \int_{t_0}^t \dd{t'} \Op{H}~{int, I} (t') )}
\end{align*}

Thus, operators in the interaction picture evolve according to
\begin{equation}
  \label{eq:62}
  \Op{\phi}~{I}(\vec{x},t) = \Op{U}(t,t_0) \Op{\phi}~H (\vec{x}) \hOp{U}(t,t_0)
\end{equation}
with state vectors given
\begin{equation}
  \label{eq:63}
  \ket{\phi}~I = \Op{U}(t,t_0) \ket{\phi}~H
\end{equation}
and
\begin{equation}
  \label{eq:64}
  \Op{U}(t,t_0) = \Op{T} \exp( -i \int_{t_0}^t \dd{t'} \Op{H}~{int,I}(t'))
\end{equation}
Since $\Op{U}$ is an operator containing both creation and
annihilation operators the number of particles can change with time.

\section{The S matrix}
\label{sec:s-matrix}

Consider an initial state, $\ket{i}$ of particles, at a time
$t=-\infty$, which interact with each other before reaching a final
state, $\ket{f}$ at time $t=\infty$.  In the Heisenberg picture these
states are constant and the final and initial states would be
equal. After the interaction we make a new measurement of the energy
and momentum of the final state and it collapses to the final state
with a probability $\abs{\braket{f}{i}}}^2$; we need the Interaction
picture to calculate the states, however.
\begin{subequations}
  \begin{align}
    \label{eq:61}
    \Op{\phi}~{in} (\vec{x}) &= \lim_{t \to - \infty} \Op{\phi}~{H}(\vec{x},t) = \lim_{t \to - \infty} \Op{\phi}~{I}(\vec{x},t) = \Op{\phi}~S (\vec{x}) \\
   \Op{\phi}~{out} (\vec{x}) &= \lim_{t \to   \infty} \Op{\phi}~{H}(\vec{x},t) = \lim_{\substack{ t \to   \infty \\ t_0 \to - \infty}} \hOp{U}(t,t_0) \Op{\phi}~{I}(\vec{x},t) \Op{U}(t,t_0)
  \end{align}
\end{subequations}
and then the required projection is
\begin{equation}
  \label{eq:65}
  S_{fi} = \braket{f}{i}~{H} = \lim_{\substack{t \to \infty \\ t_0 \to - \infty}} \bra{f} \Op{U}(t,t_0) \ket{i}~{I} = \bra{f} \Op{S} \ket{i}~{I}
\end{equation}
which is the S-matrix.
\begin{align}
  \label{eq:66}
  S_{fi} &=  \lim_{\substack{t \to \infty \\ t_0 \to - \infty}} \bra{f} \Op{U}(t,t_0) \ket{i} \nonumber\\
&= \bra{f} \Op{T} \exp( -i \int_{t_0}^t \dd{t'} \Op{H}~{int,I}(t')) \ket{i} \nonumber\\
&= \bra{f} \Op{T} \exp( -i \int \frac{\lambda}{4!} \Op{\phi}~{I}^4(x) \dd[4]{x} )\ket{i}
\end{align}
This can now be calculated using perturbation theory,
\begin{align}
  \label{eq:67}
  \bra{f}\Op{S}\ket{i} = &\braket{f}{i} - i \frac{\lambda}{4!} \int \dd[4]{x} \bra{f} \Op{T}\Op{\phi}^4~{I}(x) \ket{i} \nonumber\\
&+\qty( -i \frac{\lambda}{4!} )^2 \int \dd[4]{x} \dd[4]{x'} \bra{f} \Op{T}\Op{\phi}^4~{I}(x) \Op{\phi}^4~I(x') \ket{i}
\end{align}
We can make use of Wick's theorem to compute solutions involving
normal-ordered products and propagators.

\section{The vacuum}
\label{sec:vacuum}

In the free theory the lowest energy state was $\ket{0}$, and was
related to the field function $\op{\phi}$; thus, in the interacting
picture we have a new vacuum, $\ket{\Omega}$, and any state in the
interaction picture is not an eigenstate of the free theory, since
they interact with the virtual particles from the vacuum. We shall
ignore this problem for now.

\section{Wick's Theorem}
\label{sec:wicks-theorem}

% Wick's theorem states that two operators, $\Op{A}$ and $\Op{B}$ may be
% contracted according to
% \begin{equation}
%   \label{eq:68}
%   \wicon{A}{B} = \Op{A}\Op{B} - \normbracket{\Op{A}\Op{B}}
% \end{equation}

Consider $\Op{T} \Op{\phi}(x) \Op{\phi}(y)$, and then split the
positive and negative frequency components,
\begin{equation}
  \label{eq:69}
  \Op{\phi}(x) = \Op{\phi}^+(x) + \Op{\phi}^-(x)
\end{equation}
with $\Op{\phi}^+ = \int \nm{k} \Op{a}(\vec{k}) e^{-ik \vdot x}$
and $\Op{\phi}^- = \int \nm{k} \Op{a}(\vec{k}) e^{ik \vdot x}$.
For $x^0 > y^0$,
\begin{align}
  \label{eq:70}
  \Op{T} \Op{\phi}(x) \Op{\phi}(y) &= \quad\Op{\phi}^+(x) \Op{\phi}^+(y) + \Op{\phi}^+(x) \Op{\phi}^-(y) \nonumber\\
                                   &\quad+   \Op{\phi}^-(x) \Op{\phi}^+(y) + \Op{\phi}^-(x) \Op{\phi}^-(y) \nonumber\\
&= \quad\Op{\phi}^+(x) \Op{\phi}^+(y) + \Op{\phi}^-(y) \Op{\phi}^+(x) \nonumber\\
                                   &\quad+   \Op{\phi}^-(x) \Op{\phi}^+(y) + \Op{\phi}^-(x) \Op{\phi}^-(y) \nonumber\\
&\quad+ \comm{\phi^+(x)}{\phi^-(y)} \nonumber\\
&= \normbracket{\Op{\phi}(x) \Op{\phi}(y)} + D(x-y)
\end{align}
If $x^0<y^0$ then
$
  \tOrd \Op{\phi}(x) \Op{\phi}(y) = \normbracket{\Op{\phi}(x) \Op{\phi}(y)} + D(y-x)
$, so for any $x^0$ and $y^0$, 
\begin{equation}
  \label{eq:72}
   \tOrd \Op{\phi}(x) \Op{\phi}(y) = \normbracket{\Op{\phi}(x) \Op{\phi}(y)} + \Delta~F(x-y)
\end{equation}
Thus, for fields we can write Wick's theorem as
\begin{align}
  \tOrd \Op{\phi}(x_1) \Op{\phi}(x_2) \cdots \Op{\phi}(x_n) =& \normbracket{ \Op{\phi}(x_1) \Op{\phi}(x_2) \cdots \Op{\phi}(x_n)}\nonumber\\ &+ \text{all contractions.}
\end{align}

\providecommand{\dww}[4]{\normbracket{\Op{\phi}(x_{#1}) \Op{\phi}(x_{#2})} \,  \Delta~F (x_{#3} - x_{#4}) }
\begin{example}[Wick Contractions]
\begin{align*} \allowdisplaybreaks
 \tOrd \Op{\phi}(x_1) \Op{\phi}(x_2) &\Op{\phi}(x_3) \Op{\phi}(x_4) \\ =& \quad\, \normbracket{\Op{\phi}(x_1) \Op{\phi}(x_2) \Op{\phi}(x_3) \Op{\phi}(x_4)} \\
&+ \dww{3}{4}{1}{2} \\&+ \dww{2}{4}{1}{3} \\ &+ \dww{2}{3}{1}{4} \\&+ \dww{1}{4}{2}{3} \\ &+ \dww{1}{3}{2}{4} \\&+ \dww{1}{2}{3}{4} \\
&+ \Delta~F (x_1 - x_2) \Delta~F (x_3 - x_4) \\ &+ \Delta~F (x_1 - x_3) \Delta~F(x_2-x_4) \\ &+ \Delta~F(x_1-x_4) \Delta~F (x_2-x_3) \\
\bra{0}  \tOrd \Op{\phi}(x_1) \Op{\phi}(x_2) &\Op{\phi}(x_3) \Op{\phi}(x_4) \ket{0} \\
=&\quad\, \Delta~F (x_1 - x_2) \Delta~F (x_3 - x_4) \\ &+ \Delta~F (x_1 - x_3) \Delta~F(x_2-x_4) \\ &+ \Delta~F(x_1-x_4) \Delta~F (x_2-x_3) \\
\end{align*}
\end{example}
A common notation, used to simplify the appearance of contractions is
\begin{equation}
  \label{eq:73}
  \contraction{}{\Op{\phi}}{(x)}{\Op{\phi}}{}
  \Op{\phi}(x) \Op{\phi}(y) = \Delta~F (x-y)
\end{equation}

\section{$2 \to 2$ scattering}
  Consider a system which has an initial state with two particles at
  momenta $\vec{k}_1$ and $\vec{k}_2$, and a final state with momenta
  $\vec{p}_1$ and $\vec{p}_2$.
\begin{center}
  \begin{tfeyn}
      \tfcol{k2,k1} 
      \tfcol{p2,p1}
      \tf[f]{k2,v1,p1}
      \tf[f]{k1,v2,p2}
      \fill (v1) circle (0.2);
      \draw (k1.west) node {$\vec{k}_1$}; \draw (k2.west) node {$\vec{k}_2$};
\draw (p1.east) node {$\vec{p}_1$}; \draw (p2.east) node {$\vec{p}_2$};
  \end{tfeyn}
\end{center}
The initial state is 
\[ \ket{\vec{k}_1, \vec{k}_2} = \hOp{a}(\vec{k}_2) \hOp{a}(\vec{k}_1) \ket{0} \]
while the final state is
\[ \ket{\vec{p}_1, \vec{p}_2} = \hOp{a}(\vec{p}_2) \hOp{a}(\vec{p}_1) \ket{0} \]
The first term in the expansion of the $S$-matrix is then
{ \tiny
\begin{align*}
  &\braket{\vec{p}_1, \vec{p}_2}{\vec{k}_1, \vec{k}_2} = \bra{0} \Op{a}(\vec{p}_2) \Op{a}(\vec{p}_1) \hOp{a}(\vec{k}_2) \hOp{a}(\vec{k}_1) \ket{0}\\
  &= \bra{0} \Op{a}\vec{p}_2\qty( \comm{\Op{a}(\vec{p}_1)}{\hOp{a}(\vec{k}_2)} + \hOp{a}(\vec{k}_2) \Op{a}(\vec{p}_1) ) \hOp{a}(\vec{k}_1) \ket{0} \\
&= \comm{\Op{a}(\vec{p}_1)}{\hOp{a}(\vec{k}_2)}  \bra{0} \Op{a}(\vec{p}_2) \hOp{a}(\vec{k}_1) \ket{0}  \\
&\qquad+  \bra{0} \Op{a}(\vec{p}_2) \hOp{a}(\vec{k}_2) \ket{0}  \comm{\Op{a}(\vec{p}_1)}{\hOp{a}(\vec{k}_2)} \ket{0}\\
&= \comm{\Op{a}(\vec{p}_1)}{\hOp{a}(\vec{k}_2)} \comm{\Op{a}(\vec{p}_1)}{\hOp{a}(\vec{k}_2)} + \comm{\Op{a}(\vec{p}_1)}{\hOp{a}(\vec{k}_2)} \comm{\Op{a}(\vec{p}_1)}{\hOp{a}(\vec{k}_2)}
\end{align*}
}
Since
{\tiny
\[ \comm{\Op{a}(\vec{p}_1)}{\hOp{a}(\vec{k}_1)} =(2\pi)^3 2 E(\vec{p}_1) \delta^3(\vec{p}_1 - \vec{k}_1) \]
}
We have {\tiny
\begin{equation*}
  (2 \pi)^6 E(\vec{k}_1) E(\vec{k}_2) \qty( \delta^3 (\vec{p}_1 - \vec{k}_1) \delta^3(\vec{p}_2 - \vec{k}_2) + \delta^3(\vec{p}_1 - \vec{k}_2) \delta^3(\vec{p}_2 - \vec{k}_1) )
\end{equation*} }
In diagrammatic form this is
\begin{equation*}
   \begin{tfeynma}[1em]
     \tfcol{k2,k1} \tfcol{p2,p1} \tf{k1,p1} \tf{k2,p2}
   \end{tfeynma}
 +
 \begin{tfeynma}[1em]
     \tfcol{k2,k1} \tfcol{p2,p1} \tf{k1,p2} 
\fill [white] (0.7em,2) circle (0.3);
\tf{k2,p1}
 \end{tfeynma}
\end{equation*}
This isn't scattering, so can be excluded from the calculation.
The second term is
\[ -i \frac{\lambda}{4!} \int \dd[4]{x} \bra{\vec{p}_1,\vec{p}_2} T \Op{\phi}^4(x) \ket{\vec{k}_1, \vec{k}_2} \]
By Wick's theorem,{\small
\[ T \Op{\phi}^4(x) =  \normbracket{\Op{\phi}^4(x)} + 6 \normbracket{\Op{\phi}^2(x)} \Delta~F(x-x) + 3 \Delta~F(x-x) \Delta~F(x-x)\]
}
The normal-ordered product gives
\begin{align*} -& i \frac{\lambda}{4} \int \dd[4]{x} \nm{q_1} \nm{q_2} \nm{q_3} \nm{q_4}  e^{i (q_1+q_2-q_3-q_4) \vdot x}\\
& \times \bra{0} \Op{a}(\vec{p}_2) \Op{a}(\vec{p}_1) \hOp{a}(\vec{q}_1) \hOp{a}(\vec{q}_2) \Op{a}(\vec{q}_3) \Op{a}(\vec{q}_4) \hOp{a}(\vec{k}_2) \hOp{a}(\vec{k}_1) \ket{0}
\end{align*}
Then, since, for example $\Op{a}(\vec{q}_4) \hOp{a}(\vec{k}_2) =
\comm{\Op{a}(\vec{q}_4)}{\hOp{a}(\vec{k}_4)} + \hOp{a}(\vec{k}_2) \Op{a}(\vec{q}_4)$, and working through
each of the terms,
 \begin{align*}
    \bra{0} & \Op{a}(\vec{p}_2) \Op{a}(\vec{p}_1) \hOp{a}(\vec{q}_1) \hOp{a}(\vec{q}_2) \Op{a}(\vec{q}_3) \Op{a}(\vec{q}_4) \hOp{a}(\vec{k}_2) \hOp{a}(\vec{k}_1) \ket{0} \\
 &= 4 \comm{\Op{a}(\vec{p}_1)}{\hOp{a}(\vec{q}_1)} \comm{\Op{a}(\vec{p}_2)}{\hOp{a}(\vec{q}_2)} \\ &\qquad\times\comm{\Op{a}(\vec{q}_3)}{\hOp{a}(\vec{k}_2)} \comm{\Op{a}(\vec{q}_4)}{\hOp{a}(\vec{k}_1)}
 \end{align*}
 Then, given the commutation relations, $
 \comm{\Op{a}(\vec{p}_1)}{\hOp{a}(\vec{q}_1)} = (2 \pi)^3 3
 E(\vec{p}_1) \delta^3(\vec{p}_1 - \vec{q}_1)$ and so forth, 
 \begin{align*}
   &= - i \lambda \int  \exp(i [p_1+p_2-k_1-k_2] \vdot x ) \\
&= - i \lambda  (2 \pi)^4 \delta^4(p_1 + p_2 - k_1 - k_2) \\
&=
\begin{tfeynma}[1em]
  \tfcol{k2,k1} \tfcol{p2,p1} \tf{k2,p1} \tf{k1,p2}
\end{tfeynma}
 \end{align*}
 The next part is the
\begin{align*} 
-i & \frac{\lambda}{4!} \int \bra{\vec{p}_1, \vec{p}_2} \normbracket{\Op{\phi}^2(x)} \ket{\vec{k}_1 \vec{k}_2} \Delta~F (x-x) \\
&= -2i \frac{\lambda}{4!} \int \dd[4]{x} \nm{q_1} \nm{q_2} e^{i(q_1-q_2) \vdot x} \Delta~F(x-x) \\ &\qquad\bra{0} \Op{a}(\vec{p}_2) \Op{a}(\vec{p}_1) \hOp{a}(\vec{q}_1) \Op{a}(\vec{q}_2)  \hOp{a}(\vec{k}_2) \hOp{a}(\vec{k}_1) \ket{0}\\
&= - i \frac{\lambda}{12} \int \frac{\dd[4]{k}}{(2 \pi)^4} \nm{q_1} \nm{q_2} \frac{\delta^4(q_1 - q_2)}{k^2 - m^2 + i \epsilon} \\ & \qquad \bra{0} \Op{a}(\vec{p}_2) \Op{a}(\vec{p}_1) \hOp{a}(\vec{q}_1) \Op{a}(\vec{q}_2)  \hOp{a}(\vec{k}_2) \hOp{a}(\vec{k}_1) \ket{0} \\
&= i \frac{\lambda}{12} \int \frac{\dd[4]{k}}{(2 \pi)^4} 2(2 \pi)^3  \frac{\delta^4(q_1 - q_2)}{k^2 - m^2 + i \epsilon} \\
& \qquad \bigg( E(\vec{k}_1) \delta^3 (\vec{p}_2 - \vec{p}_1) \delta^3 (\vec{p}_1 - \vec{k}_1) \delta^3(\vec{q}_2 - \vec{k}_2) \\
& \qquad \quad E(\vec{k}_1) \delta^3(\vec{p}_2 - \vec{k}_1) \delta^3 (\vec{p}_1 - \vec{q}_1) \delta^3 (\vec{q}_2 - \vec{k}_2) \\
& \qquad \quad E(\vec{k}_2) \delta^3(\vec{p}_2 - \vec{q}_1) \delta^3(\vec{p}_1 - \vec{k}_2) \delta^3 (\vec{q}_2 - \vec{k}_1) \\
& \qquad \quad E(\vec{k}_2) \delta^3(\vec{p}_2 - \vec{k}_2) \delta^3(\vec{p}_1 - \vec{q}_1) \delta^3 (\vec{q}_2 - \vec{k}_1) \bigg) \\
&= i \frac{\lambda}{6} \int \frac{\dd[4]{k}}{(2 \pi)^4} \frac{1}{k^2-m^2+ i \epsilon}\\ 
&\qquad [ E(\vec{k}_1) (\delta^3(\vec{p}_1 - \vec{k}_1) \delta^4(p_2 - k_2) + \delta^3(\vec{p}_2-\vec{k}_1) \delta^4(p_1-k_2) ) \\
&\qquad   E(\vec{k}_2) (\delta^3(\vec{p}_1 - \vec{k}_2) \delta^4(p_2 - k_1) + \delta^3(\vec{p}_2 - \vec{k}_2) \delta^4(p_1-k_1) )] \\
&= 
  \begin{tfeynma}   \tfcol{k2,k1}   \tfcol{p2,p1}   \tf{k1,p1} \tf{k2,p2} \draw ($(k2)!0.5!(p2)$) node (c2) {}; \draw (c2.center) to (c2.north west) to [loop] (c2.north east) to (c2.center);  \end{tfeynma}
+ \begin{tfeynma}   \tfcol{k2,k1}   \tfcol{p2,p1}   \tf{k1,p1} \tf{k2,p2} \draw ($(k1)!0.5!(p1)$) node (c2) {}; \draw (c2.center) to (c2.south east) to [loop] (c2.south west) to (c2.center); \end{tfeynma}
+ \begin{tfeynma}   \tfcol{k2,k1}   \tfcol{p2,p1}   \tf{k1,p2} \draw ($(k1)!0.5!(p2)$) node (c2) {}; \fill [white] (c2) circle (0.3);  \tf{k2,p1}   \draw (c2.center) to (c2.north west) to [loop] (c2.north east) to (c2.center); \end{tfeynma}
+ \begin{tfeynma}   \tfcol{k2,k1}   \tfcol{p2,p1}   \tf{k2,p1} \draw ($(k1)!0.5!(p2)$) node (c2) {}; \fill [white] (c2) circle (0.3);  \tf{k1,p2}  \draw [rounded corner](c2.center) to (c2.north west) to [loop] (c2.north east) to (c2.center);  \end{tfeynma}
\end{align*}
None of these lines transfer momentum, so there is no scattering here
either. Notice that the integral is also divergent (a point which is
addressed later). 

The last part is also not scattering:
\begin{equation*}
  -i \frac{\lambda}{6} \int \braket{\vec{p}_1, \vec{p}_2}{\vec{k}_1, \vec{k}_2} \Delta~F^2(x-x) \dd[4]{x} = \begin{tfeynma}[1em]   \tfcol{k2,k1}   \tfcol{p2,p1}   \tf{k1,p1} \tf{k2,p2} \end{tfeynma} 
  \begin{tfeynma}[0.3em][0.5ex]
    \tfcol{k2,k1} \tfcol{p2,p1} \tf{k1,p2} \tf{p1,k2} \tf[loop]{k1,p1} \tf[loop]{p2,k2}
  \end{tfeynma} +
\begin{tfeynma}[1em]   \tfcol{k2,k1}   \tfcol{p2,p1}   \tf{k1,p2} \draw ($(k1)!0.5!(p2)$) node (c2) {}; \fill [white] (c2) circle (0.3);  \tf{k2,p1} ; \end{tfeynma}
  \begin{tfeynma}[0.3em][0.5ex]
    \tfcol{k2,k1} \tfcol{p2,p1} \tf{k1,p2} \tf{p1,k2} \tf[loop]{k1,p1} \tf[loop]{p2,k2}
  \end{tfeynma}
\end{equation*}
Here the figure-of-eight diagrams are produced by the propagator
$\Delta~F^2(x-x)$, and are a consequence of using the free rather than
the interacting vacuum.

The next term in the perturbative expansion is
\[ \qty(-i \frac{\lambda}{4!})^2 \int \bra{\vec{p}_1,\vec{p}_2} T
\Op{\phi}^4(x) \Op{\phi}^4(y) \ket{\vec{k}_1, \vec{k}_2} \] The
time-ordering is now important since there are two events, $x$ and
$y$. These connected scattering events have the form
\begin{equation*}
\begin{tfeynma}[1em]
  \tfcol{k2,k1} 
  \tfcol{p2,p1}
  \tf[left]{k2,p2}
  \tf[right]{k1,p1}
\end{tfeynma}
+
\begin{tfeynma}[.45em]
  \tfcol{k2,k1} 
  \tfcol{p2,p1}
  \tf[left]{k1,k2}
  \tf[right]{p1,p2}
\end{tfeynma}
+
\begin{tfeynma}
    \tfcol{k2,k1} 
    \tfcol{p2,p1}
    \tf{p1,k2}
    \tf{p2, k1}
    \draw ($(k2)!0.75!(p1)$) node (c2) {};
    \draw  (c2.center)to (c2.north west) to [loop] (c2.north) to (c2.center);
\end{tfeynma} + \cdots
\end{equation*}
The momentum cirulating in the loop is unconstrained, and so must be
integrated over, but this integral will be divergent.

\section{Feynman Rules}
\label{sec:feynman-rules}

\begin{enumerate}
\item For each propagator, 
  \[ \begin{tfeynma}    \tfcol{a} \tfcol{b} \tf{a,b}  \end{tfeynma} = \frac{i}{k^2-m^2+i \epsilon}\]
\item For each vertex,
\[  \begin{tfeynma}\tfcol{k2,k1} \tfcol{p2,p1} \tf{k2,p1} \tf{k1,p2} \end{tfeynma} = -i \lambda \]
\item Momentum must be conserved at every vertex, so e.g.~
\[ (2 \pi)^4 \delta^4(p_1 + p_2 - k_1 - k_2) \]
\item Every unconstrained momentum must be integrated over,
\[ \int \frac{\dd[4]{k}}{(2 \pi)^4} \]
\item Factors must be included for the number of symmetrical
  arrangements of diagram possible.
\end{enumerate}

\section{The true vacuum}
\label{sec:true-vacuum}

The true vacuum of the full interacting Hamiltonian is represented as
$\ket{\Omega}$, and defined such that
\begin{equation}
  \label{eq:71}
  \Op{H}~{int} \ket{\Omega} = 0
\end{equation}

We can then define the \emph{$n$-point Green's function}, or $n$-point
correlator,
\begin{equation}
  \label{eq:74}
  G_n(x_1, x_2, \dots, x_n) \equiv \bra{\Omega} T \Op{\phi}~H(x_1) \Op{\phi}~H(x_2) \dots \Op{\phi}~H(x_n) \ket{\Omega}
\end{equation}
Assuming that the product of $x$s are time ordered already, the
Heisenberg fields can be converted to interaction fields by
\begin{equation}
  \label{eq:75}
  \Op{\phi}~H(\vec{x}) = \hOp{U}(t,t_0) \Op{\phi}~I(\vec{x}) \Op{U}(t,t_0)
\end{equation}
Then, using the relations
\[ \Op{U}(t_1,t_2) \Op{U}(t_2,t_3) = \Op{U}(t_1, t_3) \]
and
\[ \Op{U}(t_1,t_3) \hOp{U}(t_2, t_3) = \Op{U}(t_1, t_2) \]
it is possible to shorten each pair
\[ \begin{split}
&\Op{\phi}~H(x_1) \Op{\phi}~H(x_2) \\&\quad= 
\hOp{U}(t_1,t_0) \Op{\phi}~I(\vec{x}) \Op{U}(t_1,t_0) 
\hOp{U}(t_2,t_0) \Op{\phi}~I(\vec{x}) \Op{U}(t_2,t_0)
\\ &\quad=
\hOp{U}(t_1,t_0) \Op{\phi}~I(\vec{x}) \Op{U}(t_1,t_2) \dots
\end{split}
\]

Now, taking $\exp(-i \Op{H} t) \ket{0}$, and inserting a complete set of energy states,
\begin{equation}
  \label{eq:76}
  \begin{split}
    e^{(-i \Op{H} t )} \ket{0} &= e^{(-i \Op{H} t)} \ket{\Omega} \braket{\Omega}{0}+ \sum_{n \neq 0}  e^{(-i \Op{H} t)} \ket{n} \braket{n}{0}\\
&=  e^{(-i E_0 t)} \ket{\Omega} \braket{\Omega}{0}+ \sum_{n \neq 0}  e^{(-i E_n t)} \ket{n} \braket{n}{0}
  \end{split}
\end{equation}
Taking $t \to \infty$ then all but the first term vanish
(Riemann-Lebesgue lemma), so
\[ \lim_{t \to \infty} e^{-i \Op{H} t} \ket{0} = \lim_{t \to \infty} e^{-i E_0 t} \ket{\Omega} \braket{\Omega}{0} \]
Then
\begin{align*}
  \ket{\Omega} &= \lim_{t \to \infty} \qty( e^{-i E_0 t} \ket{\Omega}
  \braket{\Omega}{0})^{-1} e^{-i H t} \ket{0} \\
&=  \lim_{t \to \infty} \qty( e^{-i E_0 (t+t_0)} \braket{\Omega}{0} )^{-1} e^{-i H(t+t_0)}\ket{0} \\
&=  \lim_{t \to \infty} \qty( e^{-i E_0 (t+t_0)} \braket{\Omega}{0} )^{-1} e^{-i \Op{H}(t+t_0)} e^{-i \Op{H}_0(t+t_0)}\ket{0} \\
&=  \lim_{t \to \infty} \qty( e^{-i E_0 (t+t_0)} \braket{\Omega}{0} )^{-1} \Op{U}(t_0-t) \ket{0}
\end{align*}
By similar logic,
\begin{equation*}
  \bra{\Omega} = \lim_{t \to \infty} \bra{0} \Op{U}(t,t_0) \qty( e^{-i E_0(t-t_0)} \braket{0}{\Omega} )^{-1}
\end{equation*}

Thus
\begin{align*}
  \bra{\Omega} & \Op{\phi}~H(x_1) \Op{\phi}~H(x_2) \dots \Op{\phi}~H(x_n)  \ket{\Omega} \\
&=  \lim_{t \to \infty} \qty(\abs{\braket{0}{\Omega}}^2 e^{-i 2E_0 t} )^{-1}
\\ & \qquad \qquad \bra{0}T \Op{\phi}~I(x_1) \Op{\phi}~I(x_2) \dots \Op{\phi}~I(x_n) \Op{U}(t,-t) \ket{0}
\end{align*}

This is true for all $n$, so choosing $n=0$,
\begin{align*}
  \braket{\Omega} = \lim_{t \to \infty} \qty( \abs{ \braket{0}{\Omega} }^2 e^{-i 2 E_0 t} )^{-1} \bra{0} \Op{U}(t,-t) \ket{0} 
\end{align*}
This implies
\begin{equation}
  \label{eq:77}
  \lim_{t \to \infty} \qty( \abs{ \braket{0}{\Omega} }^2 e^{-i 2 E_0 t} )^{-1} = \lim_{t \to \infty} \frac{1}{\bra{0} \Op{U}(t,-t) \ket{0}}
\end{equation}
The \emph{$n$-point Green's function} is then
\begin{equation}
  \label{eq:78}
  G_n (x_1, x_2, \dots, x_n) \frac{\bra{0} T \Op{\phi}~I(x_1) \Op{\phi}~I(x_2) \dots \Op{\phi}~I(x_n) \Op{S}\ket{0}}{\bra{0} \Op{S} \ket{0}}
\end{equation}
This division is the justification for the removal of disconnected
diagrams, as they appear in both the numerator and the denominator.

\section{The LSZ Reduction formula}
\label{sec:lsz-reduct-form}

The $n$-point Green's function is related to the S-matrix expectation
values via the LSZ reduction formula
\begin{align*}
&  \bra{\vec{p}_1, \vec{p}_2, \dots, \vec{p}_n} \Op{S} \ket{\vec{k}_1, \vec{k}_2, \dots, \vec{k}_n} \\
&= i^{m+n} \int \dd[4]{x_1} \dots \dd[4]{x_m} \dd[4]{y_1} \dots \dd[4]{y_n} \\
& \times \exp(-i \qty[ k_1 \vdot x_1 + \cdots + k_m \vdot x_m - p_1 \vdot y_1 - \cdots p_n \vdot y_n])\\
& \times \qty( \partial_{x_1}^2 + m^2) \cdots \qty(\partial_{x_m}^2 + m^2) \qty( \partial_{y_1}^2 + m^2) \cdots (\partial_{y_n}^2 + m^2) \\
& \times G_n(x_1, x_2, \dots, x_n)
\end{align*}

%%% Local Variables: 
%%% mode: latex
%%% TeX-master: "../project"
%%% End: 
