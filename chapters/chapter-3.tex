%% Classical Fields

A physical field is normally governed by two restrictions; a
differential equation, and boundary conditions, but they can be
described using the Lagrangian and the principle of least action.

\begin{definition}[Action]
  Action is the quantity
  \[ S = \int \Lag(\phi, \partial_\mu \phi) \dd[4]{x} \]
  for $\Lag$ the Lagrangian, itegrated over all space-time.
\end{definition}

A field adopts a configuration to reduce the action to a minimum. To
find this we look for the infinitessimal variations which leave the
action unchanged, i.e.
\begin{equation}
  \label{eq:1}
  \phi(x) \to \phi(x) + \delta \phi(x) \quad | \quad S \to S + \delta S 
\end{equation}
with $\delta S = 0$.

Then
\begin{align*}
  \delta S &= 
     \delta \phi \pdv{\phi} \int \Lag(\phi, \partial_{\nu} \phi) \dd[4]{x} \\ 
     &\quad+ \delta(\partial_{\mu} \phi ) \pdv{(\partial \mu
        \phi)} \int \Lag(\phi, \partial_{\nu} \phi) \dd[4]{x} \\
     &= \int \qty( \delta \phi \pdv{\Lag}{\phi} + \partial_{\mu} (\delta
        \phi) \pdv{\Lag}{(\partial_{\mu} \phi)} ) \dd[4]{x}
\end{align*}
since, for a well-behaved field we have $\delta (\partial_{\mu} \phi)
= \partial_{\mu}(\delta \phi)$, by the mixed-derivatives theorem.
Now, 
\begin{equation}
  \label{eq:2}
  \partial_{\mu} \qty( \delta \phi \pdv{\Lag}{(\partial_{\mu} \phi)} ) = 
  \partial_{\mu} (\delta \phi) \pdv{\Lag}{(\partial_{\mu} \phi)} + \delta \phi \partial_{\mu} \qty( \pdv{\Lag}{(\partial_{\mu} \phi)} )
\end{equation}
which, given that a total derivative evaluated across all of space
will vanish, since it is equivalent to being evaluated at either end
of the region, and so
\begin{equation}
  \label{eq:3}
  \delta S = \int \qty[ \delta \phi \pdv{\Lag}{\phi} + \partial_{\mu} \qty(\delta \phi \pdv{\Lag}{(\partial_{\mu} \phi)}) - \delta \phi \partial_{\mu} \qty( \pdv{\Lag}{(\partial_{\mu}\phi)})] \dd[4]{x}
\end{equation}

This is true for all $\delta \phi$, and so we can get the
Euler-Lagrange equations,
\begin{fequation}[Euler-Lagrange Equations]
  \label{eq:euler-lagrange}
%\begin{equation}
  \pdv{\Lag}{\phi} - \partial_{\mu} \qty( \pdv{\Lag}{(\partial_{\mu} \phi )}) = 0 
%\end{equation}
\end{fequation}

\begin{example}[Using the Euler-Lagrange Equations]
  Consider a Lagrangian
  \[ \Lag = \half \partial_{\mu} \phi \partial^{\mu} \phi \] the terms
  the terms for the Lagrangian are then
  \[ \pdv{\Lag}{\phi} = 0 \] and \[ \pdv{\Lag}{(\partial_{\mu} \phi)}
  = \partial^{\mu} \phi \] Which gives the wave equation:
  \[ \partial_{\mu} \partial^{\mu} \phi = \qty( \pdv[2]{t} -
  \vec{\nabla}^2 ) \phi = 0 \]
\end{example}

\section{Noether's Theorem}
\label{sec:noethers-theorem}

\begin{theorem}[Noether's Theorem]
  If action is unchanged under a transformation then there exists a
  conserved current which is associated with the symmetry.
\end{theorem}

Consider the infinitessimal transformation of both coordinates and the
fields,
\begin{align*}
  x^{\mu} \to x^{\prime \mu} &= x^{\mu} + \delta x^{\mu} = x^{\mu} + \tensor{X}{^\mu_\nu} \omega^{\nu} 
  \\ \phi \to \phi^{\prime} &= \phi + \delta \phi = \phi + \tensor{\Phi}{_{\nu}} \omega^{\nu}
\end{align*}
which are parametrised by an infintessimal parameter $\omega^{\nu}$.

The change in the field can be due to both a change in the function
defining the field, $\delta_{\phi} \phi$, and in the coordinates,
$\partial_{\mu} \phi \delta x^{\mu}$. The change in the Lagrangian is
then
\begin{align*}
  \delta \Lag &= \partial_{\mu} \Lag \delta x^{\mu} + \pdv{\Lag}{\phi} \delta_{\phi}\phi + \pdv{\Lag}{(\partial_{\nu} \phi)} \delta_{\phi} (\partial_{\nu} \phi) \\
&= \partial_{\mu} \Lag \delta x^{\mu} + \partial_{\nu} \qty( \pdv{\Lag}{(\partial_{\nu} \phi)}) \delta_{\phi} \phi + \pdv{\Lag}{(\partial_{\nu} \phi)} \partial_{\nu} (\delta_{\phi} \phi ) \\ 
&= \partial_{\mu} \Lag \delta x^{\mu} + \partial_{\nu} \qty( \pdv{\Lag}{(\partial_{\nu} \phi )} \delta_{\phi} \phi )
\end{align*}
This is the change in the Lagrangian, the change in the action is 
\begin{equation}
  \label{eq:4}
  \delta S = \delta \qty( \int \dd[4]{x} \Lag )
\end{equation}
the integration measure changes, and requires a Jacobian:
\[ \dd[4]{x^{\prime}} = \qty| \pdv{x^{\prime}}{x} | \dd[4]{x} =
(1+ \partial_{\mu} \delta x^{\mu} ) \dd[4]{x} \]
so
\begin{align*}
  \delta S &= \int \dd[4]{x} (\delta \Lag + \Lag \partial_{\mu} \delta x^{\mu} \\
  &= \int \dd[4]{x} \qty( \partial_{\mu} \Lag \delta x^{\mu}
  + \partial_{\nu} \qty( \pdv{\Lag}{(\partial_{\nu}
    \phi)} \partial_{\phi} \phi ) + \Lag \partial_{\mu} \delta x^{\mu}
  ) \\ &= \int \dd[4]{x} \partial_{\mu} \qty( \pdv{\Lag}{(\partial_{\mu} \phi)} \delta_{\phi} \phi + \Lag \delta x^{\mu} )
\end{align*}
Now writing $\delta x^{\mu} = \tensor{X}{^{\mu}_{\nu}} \omega^{\nu}$, and $\delta \phi = \tensor{\Phi}{_{\nu}}\omega^{\nu}$, then
\[ \delta_{\phi} \phi = \delta \phi - \partial_{\mu} \phi - \delta x^{\mu} = (\Phi_{\nu} - \partial_{\mu} \phi \tensor{X}{^{\mu}_{\nu}} ) \omega^{\nu} \]
which can be written in terms of the divergence of a current,
\[ \delta S = - \int \dd[4]{x} \partial_{\mu} \tensor{j}{^{\mu}_{\nu}}
\omega^{\nu} \]
where
\begin{equation}
  \label{eq:5}
  \tensor{j}{^{\mu}_{\nu}} = - \pdv{\Lag}{(\partial_{\mu} \phi)} 
                          ( \tensor{\Phi}{_{\nu}} - \partial_{\rho} \phi
                            \tensor{X}{^{\rho}_{\nu}}) - \Lag \tensor{X}{^{\mu}_{\nu}}
\end{equation}
which can be rearranged to give
\begin{fequation}
  \begin{equation}
\label{eq:6}
 \tensor{j}{^{\mu}_{\nu}} = \qty( \pdv{\Lag}{(\partial_{\mu} \phi) } \partial_{\rho} \phi - \tensor{g}{^{\mu}_{\rho}} \Lag ) \tensor{X}{^{\rho}_{\nu}} - \pdv{\Lag}{(\partial_{\mu} \phi)} \Phi_{\nu} 
\end{equation}
\end{fequation}
To ensure that the action is invariant under this transformation this current must be conserved, i.e. 
\begin{fequation}
  \begin{equation}
    \label{eq:7}
    \partial_{\mu} \tensor{j}{^{\mu}_{\nu}} = 0
  \end{equation}
\end{fequation}
which is Noether's Theorem.

\begin{example}[The energy-momentum tensor.]
  Consider space-time translations:
  \begin{align}
    x^{\mu} \to x^{\prime \mu} = x^{\mu} + \omega^{\nu} & \implies \tensor{X}{^{\mu}_{\nu}} = \tensor{g}{^{\mu}_{\nu}} \\
\phi \to \phi^{\prime} = \phi & \implies \Phi_{\nu} = 0
  \end{align}
  Substituting these into the expression for conserved current we have
    \begin{equation}
      \label{eq:8}
      \tensor{T}{^{\mu}^{\nu}} = \pdv{\Lag}{(\partial_{\mu} \phi)} \partial_{\nu} \phi - \tensor{g}{^{\mu}^{\nu}} \Lag
    \end{equation}
    which is central to General relativity. There are a number of
    physical consequences of this tensor's existence. Consider a part
    of the tensor,
    \[ \tensor{J}{^{\mu}} = \tensor{T}{^{0}^{\mu}} =
    \pdv{\Lag}{(\partial_0 \phi)} \partial^{\mu}\phi -
    \tensor{g}{^0^{\mu}} \Lag \] and then integrate its divergence
    over a three-dimensional volume $V$,
    \[ \int_V \partial_{\mu} J^{\mu} \dd{V} = \int_V \qty(
    \pdv{J^0}{t} + \vec{\nabla} \cdot \vec{J} ) \dd{V} \]

    By Noether's theorem we know the divergence is zero, so, applying
    Gauss's Theorem,
    \[ \int_V \pdv{J^0}{t} \dd{V} = - \int_V \vec{\nabla} \cdot
    \vec{J} \dd{V} = - \int_A \vec{J} \cdot \dd{\vec{A}} \] Thus any
    change in the total $J^0$ in the volume must be due to a current
    $J$ flowing through the surface of the volume.

The conserved quantity associated with time transformations is the Hamiltonian,
\begin{equation}
  \label{eq:9}
  H = \int \tensor{T}{^0^0} \dd[3]{x}
\end{equation}
while the quantity associated with space transformations is the three momentum operator
\begin{equation}
  \label{eq:10}
  P^i = \int \tensor{T}{^0^i} \dd[3]{x}
\end{equation}
\end{example}

%%% Local Variables: 
%%% mode: latex
%%% TeX-master: "../project"
%%% End: 

