
Before the advent of quantum field theory the presence of negative
energy solutions to the Klein-Gordon equation was considered a major
problem in quantum theory. These states are normally considered to be
positive energy anti-particle states, a view known as the
Feynman-St\"uckelberg interpretation. Only quantum field theory can
adequately manage the production of particle-antiparticle pairs.

\section{The Dirac equation}
\label{sec:dirac-equation}

Dirac attempted to avoid the problem of negative energy states by
finding a field equation linear in its operators. This must take the
form
\[ E = \vec{\alpha} \vdot \vec{p} + \beta m \to i \pdv{\psi}{t} = \qty(-i \vec{\alpha} \vdot \vec{\nabla} + \beta m) \psi \] and we need to find $\vec{\alpha}$ and $\beta$.
If we also insist that we satisfy
\[ E^2 = m^2 + \abs{\vec{p}}^2 \]
then, with implicit summation,
\begin{align*} E^2 &= \alpha_i \alpha_j p_i p_j + (\alpha)_i \beta + \beta \alpha_i) m p_i + \beta^2 m^2 \\
&= \half(\alpha_i \alpha_j + \alpha_j \alpha_i) p_i p_j + ( \alpha_i \beta + \beta \alpha_i) m p_i + \beta^2 m^2
\end{align*}
Enforcing $E^2 = m^2 + \abs{\vec{p}}^2$ we have the conditions on
$\vec{\alpha}$ and $\beta$.

\begin{subequations}
\begin{align}
  \label{eq:79}
  \alpha_i \alpha_j + \alpha_j \alpha_i &= 2 \delta_{ij} \\
\alpha_i \beta + \beta \alpha_i &= 0 \\
\beta^2 &= 1
\end{align}
\end{subequations}
These are anti-commuting objects and not just numbers. These relations
define them, and any representation which follows these conditions is
a suitable description. The Dirac representation presents them as
matrices,
\begin{equation}
  \label{eq:80}
  \alpha =
  \begin{pmatrix}
    0 & \vec{\sigma} \\ \vec{\sigma} & 0
  \end{pmatrix}
\end{equation}
and 
\begin{equation}
  \label{eq:81}
  \beta =
  \begin{pmatrix}
    \vec{1} & 0 \\ 0 & - \vec{1}
  \end{pmatrix}
\end{equation}
for $\sigma_i$ the Pauli matrices,
\begin{equation}
  \label{eq:82}
  \sigma_1 =  \begin{bmatrix}     0 & 1 \\ 1 & 0   \end{bmatrix}, \quad
\sigma_2 =    \begin{bmatrix}     0 & -i \\ i & 0  \end{bmatrix}, \quad
\sigma_3 =    \begin{bmatrix}     1 & 0  \\ 0 & -1 \end{bmatrix}
\end{equation}
Since these act on the field the field itself must have four
components, and so is a spinor.

\section{Gamma matrices}
\label{sec:gamma-matrices}

The Dirac equation can be written in four-vector form by defining a
new object, $\gamma^{\mu}$,
\begin{equation}
  \label{eq:83}
  \gamma^0 \equiv \beta = \begin{pmatrix} \vec{1} & 0 \\ 0 & - \vec{1}  \end{pmatrix}, \qquad
\vec{\gamma} \equiv \beta \alpha = \begin{pmatrix} 0 & \vec{\sigma} \\ - \vec{\sigma} & 0 \end{pmatrix}
\end{equation}
These follow the anticommutation relations
\begin{equation}
  \label{eq:84}
  \acomm{\gamma^{\mu}}{\gamma^{\nu}} = 2 g^{\mu \nu}
\end{equation}
and the Dirac equation becomes
\begin{equation}
  \label{eq:85}
  \qty( i \gamma^{\mu} \partial_{\mu} - m) \psi = 0
\end{equation}
We can write the contraction with a gamma matrix in the slash
notation, so $\ds = \gamma^{\mu} \partial_{\mu}$. This equation
describes the free dynamics of a fermion field.

\subsection{Properties of the Dirac matrices}
\label{sec:prop-dirac-matr}

\begin{subequations}
  \begin{equation}
    \label{eq:86}
    \hcon{(\gamma^0)} = \gamma^0, \quad \hcon{(\gamma^i)} = - \gamma^i
  \end{equation}
  \begin{equation}
    \label{eq:87}
    (\gamma^0)^2 = 1, \quad (\gamma^i)^2 = -1
  \end{equation}
  \begin{equation}
    \label{eq:88}
    \hcon{(\gamma^{\mu})} = \gamma^0 \gamma^{\mu} \gamma^0
  \end{equation}
\end{subequations}
The gamma matrices must be of even dimension, and are fixed matrices,
despite looking like four-vectors, and are not invariant under a
Lorentz boost.

\section{Negative energy solutions?}
\label{sec:negat-energy-solut}

we look for plane-wave solutions of the form
\[ \psi(t, \vec{x}) = u(\vec{p}) e^{-i(Et - \vec{p} \vdot \vec{x})} =
\begin{bmatrix}
  \chi \\ \phi
\end{bmatrix}
e^{-i(Et - \vec{p} \vdot \vec{x})}
\]
Shifting back out of four-vector notation,
\[ i \pdv{\psi}{t} = (-i \vec{\alpha} \vdot \nabla + \beta m) \psi \implies E
\begin{bmatrix}
  \chi \\ \phi
\end{bmatrix}
=
\begin{bmatrix}
  m & \vec{\alpha} \vdot \vec{p} \\ \vec{\sigma} \vdot \vec{p} & -m
\end{bmatrix}
\begin{bmatrix}
  \chi \\ \phi
\end{bmatrix}
\]
For a particle at rest $\vec{p}=\vec{0}$, so
\begin{equation}
  \label{eq:89}
  E \begin{bmatrix}
  \chi \\ \phi
\end{bmatrix}
=
\begin{bmatrix}
  m & 0 \\ 0 & -m
\end{bmatrix}
\begin{bmatrix}
  \chi \\ \phi
\end{bmatrix}
\end{equation}
This has the solutions, the first two for $E=m$, the latter two for $E=-m$
\[ u =
\begin{bmatrix}
  1 \\ 0 \\ 0 \\ 0
\end{bmatrix}, \quad u = \begin{bmatrix}
  0 \\ 1 \\ 0 \\ 0
\end{bmatrix}, \quad
u =
\begin{bmatrix}
  0 \\ 0 \\ 1 \\ 0
\end{bmatrix}, \quad u = \begin{bmatrix}
  0 \\ 0 \\ 0 \\ 1
\end{bmatrix}
\]
Thus there are still negative energy solutions; we get around this
using the Exclusion Principle. The Dirac equation describes particles
with spin, and so this seems appropriate. This reasoning leads to the
idea of a Dirac Sea; where all of the negative energy levels are
already filled, and so the electon cannot fall into a negative
state. If a particle is moved up it leaves a hole in the sea, which we
interpret as an antiparticle. This argument, however, doesn't hold up
for Bosons.


%%% Local Variables: 
%%% mode: latex
%%% TeX-master: "../project"
%%% End: 
