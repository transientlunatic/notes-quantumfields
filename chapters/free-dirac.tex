
Before the advent of quantum field theory the presence of negative
energy solutions to the Klein-Gordon equation was considered a major
problem in quantum theory. These states are normally considered to be
positive energy anti-particle states, a view known as the
Feynman-St\"uckelberg interpretation. Only quantum field theory can
adequately manage the production of particle-antiparticle pairs.

\section{The Dirac equation}
\label{sec:dirac-equation}

Dirac attempted to avoid the problem of negative energy states by
finding a field equation linear in its operators. This must take the
form
\[ E = \vec{\alpha} \vdot \vec{p} + \beta m \to i \pdv{\psi}{t} = \qty(-i \vec{\alpha} \vdot \vec{\nabla} + \beta m) \psi \] and we need to find $\vec{\alpha}$ and $\beta$.
If we also insist that we satisfy
\[ E^2 = m^2 + \abs{\vec{p}}^2 \]
then, with implicit summation,
\begin{align*} E^2 &= \alpha_i \alpha_j p_i p_j + (\alpha)_i \beta + \beta \alpha_i) m p_i + \beta^2 m^2 \\
&= \half(\alpha_i \alpha_j + \alpha_j \alpha_i) p_i p_j + ( \alpha_i \beta + \beta \alpha_i) m p_i + \beta^2 m^2
\end{align*}
Enforcing $E^2 = m^2 + \abs{\vec{p}}^2$ we have the conditions on
$\vec{\alpha}$ and $\beta$.

\begin{subequations}
\begin{align}
  \label{eq:79}
  \alpha_i \alpha_j + \alpha_j \alpha_i &= 2 \delta_{ij} \\
\alpha_i \beta + \beta \alpha_i &= 0 \\
\beta^2 &= 1
\end{align}
\end{subequations}
These are anti-commuting objects and not just numbers. These relations
define them, and any representation which follows these conditions is
a suitable description. The Dirac representation presents them as
matrices,
\begin{equation}
  \label{eq:80}
  \alpha =
  \begin{pmatrix}
    0 & \vec{\sigma} \\ \vec{\sigma} & 0
  \end{pmatrix}
\end{equation}
and 
\begin{equation}
  \label{eq:81}
  \beta =
  \begin{pmatrix}
    \vec{1} & 0 \\ 0 & - \vec{1}
  \end{pmatrix}
\end{equation}
for $\sigma_i$ the Pauli matrices,
\begin{equation}
  \label{eq:82}
  \sigma_1 =  \begin{bmatrix}     0 & 1 \\ 1 & 0   \end{bmatrix}, \quad
\sigma_2 =    \begin{bmatrix}     0 & -i \\ i & 0  \end{bmatrix}, \quad
\sigma_3 =    \begin{bmatrix}     1 & 0  \\ 0 & -1 \end{bmatrix}
\end{equation}
Since these act on the field the field itself must have four
components, and so is a spinor.

\section{Gamma matrices}
\label{sec:gamma-matrices}

The Dirac equation can be written in four-vector form by defining a
new object, $\gamma^{\mu}$,
\begin{equation}
  \label{eq:83}
  \gamma^0 \equiv \beta = \begin{pmatrix} \vec{1} & 0 \\ 0 & - \vec{1}  \end{pmatrix}, \qquad
\vec{\gamma} \equiv \beta \alpha = \begin{pmatrix} 0 & \vec{\sigma} \\ - \vec{\sigma} & 0 \end{pmatrix}
\end{equation}
These follow the anticommutation relations
\begin{equation}
  \label{eq:84}
  \acomm{\gamma^{\mu}}{\gamma^{\nu}} = 2 g^{\mu \nu}
\end{equation}
and the Dirac equation becomes
\begin{equation}
  \label{eq:85}
  \qty( i \gamma^{\mu} \partial_{\mu} - m) \psi = 0
\end{equation}
We can write the contraction with a gamma matrix in the slash
notation, so $\ds = \gamma^{\mu} \partial_{\mu}$. This equation
describes the free dynamics of a fermion field.

\subsection{Properties of the Dirac matrices}
\label{sec:prop-dirac-matr}

\begin{subequations}
  \begin{equation}
    \label{eq:86}
    \hcon{(\gamma^0)} = \gamma^0, \quad \hcon{(\gamma^i)} = - \gamma^i
  \end{equation}
  \begin{equation}
    \label{eq:87}
    (\gamma^0)^2 = 1, \quad (\gamma^i)^2 = -1
  \end{equation}
  \begin{equation}
    \label{eq:88}
    \hcon{(\gamma^{\mu})} = \gamma^0 \gamma^{\mu} \gamma^0
  \end{equation}
\end{subequations}
The gamma matrices must be of even dimension, and are fixed matrices,
despite looking like four-vectors, and are not invariant under a
Lorentz boost.

\section{Negative energy solutions?}
\label{sec:negat-energy-solut}

we look for plane-wave solutions of the form
\[ \psi(t, \vec{x}) = u(\vec{p}) e^{-i(Et - \vec{p} \vdot \vec{x})} =
\begin{bmatrix}
  \chi \\ \phi
\end{bmatrix}
e^{-i(Et - \vec{p} \vdot \vec{x})}
\]
Shifting back out of four-vector notation,
\[ i \pdv{\psi}{t} = (-i \vec{\alpha} \vdot \nabla + \beta m) \psi \implies E
\begin{bmatrix}
  \chi \\ \phi
\end{bmatrix}
=
\begin{bmatrix}
  m & \vec{\alpha} \vdot \vec{p} \\ \vec{\sigma} \vdot \vec{p} & -m
\end{bmatrix}
\begin{bmatrix}
  \chi \\ \phi
\end{bmatrix}
\]
For a particle at rest $\vec{p}=\vec{0}$, so
\begin{equation}
  \label{eq:89}
  E \begin{bmatrix}
  \chi \\ \phi
\end{bmatrix}
=
\begin{bmatrix}
  m & 0 \\ 0 & -m
\end{bmatrix}
\begin{bmatrix}
  \chi \\ \phi
\end{bmatrix}
\end{equation}
This has the solutions, the first two for $E=m$, the latter two for $E=-m$
\[ u =
\begin{bmatrix}
  1 \\ 0 \\ 0 \\ 0
\end{bmatrix}, \quad u = \begin{bmatrix}
  0 \\ 1 \\ 0 \\ 0
\end{bmatrix}, \quad
u =
\begin{bmatrix}
  0 \\ 0 \\ 1 \\ 0
\end{bmatrix}, \quad u = \begin{bmatrix}
  0 \\ 0 \\ 0 \\ 1
\end{bmatrix}
\]
Thus there are still negative energy solutions; we get around this
using the Exclusion Principle. The Dirac equation describes particles
with spin, and so this seems appropriate. This reasoning leads to the
idea of a Dirac Sea; where all of the negative energy levels are
already filled, and so the electon cannot fall into a negative
state. If a particle is moved up it leaves a hole in the sea, which we
interpret as an antiparticle. This argument, however, doesn't hold up
for Bosons.

\subsection{A general soultion}
\label{sec:general-soultion}
We have
\[ \implies E
\begin{bmatrix}
  \chi \\ \phi
\end{bmatrix}
=
\begin{bmatrix}
  m & \vec{\alpha} \vdot \vec{p} \\ \vec{\sigma} \vdot \vec{p} & -m
\end{bmatrix}
\begin{bmatrix}
  \chi \\ \phi
\end{bmatrix}
\]
So
\begin{equation*}
  \chi = \frac{\vec{\sigma} \vdot \vec{p}}{E-m} \phi, \qquad \phi = \frac{\vec{\sigma} \vdot \vec{p}}{E+m} \chi
\end{equation*}
These are compatible:
\begin{equation*}
  \phi = \qty( \frac{\vec{\sigma} \vdot \vec{p}}{E+m}) \qty( \frac{\vec{\sigma} \vdot \vec{p}}{E-m}) \phi = \frac{\abs{\vec{p}}^2}{E^2-m^2} \phi = \phi
\end{equation*}
Since $E^2 = m^2 + \abs{p}^2$, and $\sigma_i \sigma_j = \delta_{ij} +
i \epsilon_{ijk} \sigma_k \implies (\vec{\sigma} \vdot \vec{p})^2 =
\abs{\vec{p}}^2 + i (\vec{p} \cp \vec{p}) \vdot \vec{\sigma} =
\abs{\vec{p}}^2$.  We need to choose a basis for the solutions, the
simplest is
\begin{equation}
  \label{eq:90}
  \xi^{(1)} =
  \begin{bmatrix}
    1 \\ 0
  \end{bmatrix}, \qquad
\xi^{(2)} =
\begin{bmatrix}
  0 \\ 1
\end{bmatrix}
\end{equation}
Then the positive energy solutions, $\psi^{(1)}$ and $\psi^{(2)}$, to
\[ \phi = \frac{\vec{\sigma} \vdot \vec{p}}{E+m} \chi \]
are
\[ \chi = \xi^{(s)}, \qquad \phi = \frac{\vec{\sigma} \vdot \vec{p}}{E+m} \xi^{(s)} \]
With
\[ \psi^{(s)}(x) = u^{(s)} e^{-ip \vdot x} = \sqrt{E+m}
\begin{bmatrix}
  \xi^{(s)} \\ \frac{\vec{\sigma} \vdot \vec{p}}{E+m} \xi^{(s)}
\end{bmatrix}
e^{-ip \vdot x}
\]
The negative energy solutions, $\psi^{(3)}$ and $\psi^{(4)}$ to
\[ \phi = \frac{\vec{\sigma} \vdot \vec{p}}{E-m} \chi \]
are then 
\[ \phi = \xi^{(s)}, \qquad \chi = \frac{\vec{\sigma} \vdot \vec{p}}{E-m} \xi^{(s)} \]
With
\[ \psi^{(s+2)}(x) = u^{(s+2)} e^{-ip \vdot x} = \sqrt{-E+m}
\begin{bmatrix}
 \frac{\vec{\sigma} \vdot \vec{p}}{E-m} \xi^{(s)} \\   \xi^{(s)}
\end{bmatrix}
e^{-ip \vdot x}
\]

\subsection{Orthogonality and completeness}
\label{sec:orth-compl}

With the normalisation of $2E$ particles per unit volume,
\[ \hcon{u^{(r)}} u^{(s)} = 2 E \delta^{rs}, \qquad \hconn{v^{(r)}} v^{(s)} = 2E \delta^{rs} \]
which is a statement of orthogonality. The completeness relations are 
\begin{align*}
  \sum_{s=1,2} u^{(s)}(p) \bar{u}^{(s)} (p) &= \ps + m \\
\sum_{1,2} v^{(s)}(p) \bar{v}^{(s)}(p) &= \ps - m
\end{align*}
with $\bar{u} = \hcon{u} \gamma^0$.

\section{The Dirac Lagrangian}
\label{sec:dirac-lagrangian}

The Lagrangian of the Dirac field has the form
\begin{align}
  \label{eq:91}
  \Lag &= \bar{\psi} \qty( i \gamma^{\mu} \partial_{\mu}) \psi \nonumber\\
&= \bar{\psi}_i \qty( i \qty[ \gamma^{\mu}]_{ij} \partial_{\mu} - m \delta_{ij} ) \psi_j
\end{align}

There are two sets of Euler-Lagrange equations, one for $\psi$ and one
for $\bar{\psi}$.

\[ \pdv{\Lag}{\psi_i} - \pd{\mu} \qty(\pdv{\Lag}{(\pd{\mu}\psi_i)} )=0 \]
\[ \pdv{\Lag}{\bar{\psi}_i} - \pd{\mu} \qty(\pdv{\Lag}{(\pd{\mu}\bar{\psi}_i)} )=0 \]

Taking the $\bar{\phi}$ equation, so,
\[ \pdv{\Lag}{\bar{\psi}_i} = \qty( i [\gamma^{\mu}]_{ij} \pd{\mu} - m \delta_{ij} ) \psi_j \]
and
\[ \pdv{\Lag}{(\pd{\mu} \bar{\psi}_i)} = 0 \]
and noting that $\bar{\psi} \equiv \hcon{\psi}_i \gamma^0$, then we arrive at the equation of motion
\[ \qty( i \gamma^{\mu} \pd{\mu} - m) \psi = 0 \]
The other set of E-L equations gives the equation for antiparticle,
\[ i (\pd{\mu} \bar{\psi}) \gamma^{\mu} + m \bar{\psi} = 0 \] We can
introduce some new notation here, to indicate the direction in which
the derivative operator acts. The left-associative derivative is then
$\overleftarrow{\pd{\mu}}$, while the right-associative equivalent is
$\overrightarrow{\pd{\mu}}$. Thus, the second equation can be rewritten
\[ ( i \gamma^{\mu} \lpd{\mu} - m) \psi = 0 \]

The Dirac Lagrangian appears to treat the two fields in an
antisymmetric manner, but in principle each is as fundamental as the
other, so we can then re-write the Lagrangian, 
\[ \Lag = \bar{\psi}(i \gamma^{\mu} \pd{\mu}-m) \psi = \pd{\mu} \qty(
i \bar{\psi} \gamma^{\mu} \psi) - i \qty( \pd{\mu} \bar{\psi} )
\gamma^{\mu} \psi - m \psi \bar{\psi} \] where the first term in the
right-hand-side is zero, since it's a total derivative.
We can even write it
\begin{align*}
  \Lag &= \half \bar{\psi} i \gamma^{\mu} \pd{\mu}\psi - \half i (\pd{\mu}\bar{\psi}) \gamma^{\mu} \psi - m \bar{\psi} \psi \\ 
&= \half \bar{\psi} (i \gamma^{\mu} \rpd{\mu} - m) \psi - \half \bar{\psi} ( i \gamma^{\mu} \lpd{\mu} + m)\psi
\end{align*}

\section{Spin and angular momentum}
\label{sec:spin-angul-moment}

The angular momentum of a particle is 
\[ \vec{L} = \vec{r} \cp \vec{p} \] If this commutes with the
Hamiltonian the angular momentum is conserved, but we find
\begin{equation}
  \label{eq:92}
  \comm{H}{\vec{L}} = \comm{\vec{a}\vdot \vec{p}}{\vec{r} \cp \vec{p}} = - i \vec{\alpha} \cp \vec{p} \neq 0
\end{equation}
Clearly angular momentum is not conserved. Defining a new quantity, the instrinsic angular momentum (or spin),
\begin{equation}
  \label{eq:93}
  \vec{\Sigma} =
  \begin{bmatrix}
    \vec{\sigma} & 0 \\ 0 & \vec{\sigma}
  \end{bmatrix}
  = -i \alpha_1 \alpha_2 \alpha_3 \vec{\alpha}
= - i \gamma_1\gamma_2\gamma_3 \vec{\gamma}
\end{equation}
Then
\begin{equation}
  \label{eq:94}
  \comm{H}{\vec{\Sigma}} = \comm{\vec{\alpha} \vdot \vec{p}}{-i \alpha_1 \alpha_2 \alpha_3 \vec{\alpha}} = 2 i \vec{\alpha} \cp \vec{p}
\end{equation}
It's therefore clear that we can define a quantity
\begin{equation}
  \label{eq:95}
  \vec{J} = \vec{L} + \half \vec{\Sigma}
\end{equation}
which is conserved. This is the total angular momentum.

The basis spinors are eigenvectors of 
\[ \half \Sigma^3 = \half
\begin{bmatrix}
  1&0&0&0\\0&-1&0&0\\0&0&1&0\\0&0&0&-1
\end{bmatrix}
\]
with eigenvalues $\pm\half$.
This explains the four degrees of freedom:
\begin{itemize}
\item Positive energy; spin up
\item Positive energy, spin down
\item Negative energy, spin up
\item Negative energy, spin down
\end{itemize}

\section{Helicity of massless fermions}
\label{sec:helic-massl-ferm}

If the mass of the field is zero the wave equation simplifies to
\[ E
\begin{bmatrix}
  \chi \\ \phi
\end{bmatrix}
= 
\begin{bmatrix}
  0 & \vec{\sigma} \vdot \vec{p} \\ \vec{\sigma} \vdot \vec{p} & 0
\end{bmatrix}
\begin{bmatrix}
  \chi \\ \phi
\end{bmatrix}
\]
So
\[ E \phi = \vec{\sigma} \vdot \vec{p} \chi, \quad E \chi = \vec{\sigma} \vdot \vec{p} \phi \]
Then defining 
\[ \Psi~{{R,L}} = \half (\chi\pm\phi) \]
We have
\[ E \Psi~R = \vec{\sigma} \vdot \vec{p} \Psi~R, \quad E \Psi~L = -
\vec{\sigma} \vdot \vec{p} \Psi~L \] With $\Psi~{R,L}$ denoted the
\emph{Weyl spinors}, which are completely independent, and so can be
considered as two independent particles. Each of these is an eigenstate of the operator
\[ \frac{\vec{\sigma} \vdot \vec{p}}{\abs{\vec{p}}} = \frac{\vec{\sigma} \vdot \vec{p}}{E} \] with eigenvalues $\pm 1$.

For the full Dirac spinor the \emph{helicity} operator is defined as
\begin{equation}
  \label{eq:96}
  \Op{h} = \frac{\vec{\Sigma} \vdot \vec{p}}{\abs{\vec{p}}}
\end{equation}
which is the component of spin in the direction of the motion.  If the
eigenvalue of the helicity is $+\half$ the particle is right-handed,
and if it is $-\half$ it is left-handed. An antiparticle will have the
opposite helicity to an otherwise identical particle, since its
momentum has an opposite direction.

We can project a particular helicty from a Dirac spinor using gamma matrices. Defining
\begin{equation}
  \label{eq:97}
  \gamma^5 = i \gamma^0 \gamma^1 \gamma^2 \gamma^3 =
  \begin{bmatrix}
    0&\vec{1}\\\vec{1}&0
  \end{bmatrix}
\end{equation}
Then we can define projection operators,
\begin{equation}
  \label{eq:98}
  P~{R,L} = \half(1\pm\gamma^5) = \half
  \begin{bmatrix}
    \vec{1} & \pm \vec{1} \\ \pm \vec{1} & \vec{1}
  \end{bmatrix}
\end{equation}

The spinor $P~Lu$will be left-handed, but $P~R u$ will be
right-handed.

\section{Weyl representation}
\label{sec:weyl-representation}

The \emph{Weyl} or \emph{chiral representation} of the gamma matrices
makes the relation to spin more explicit. In this representation
\[ \gamma^0 = \begin{bmatrix}  0&\vec{1}\\\vec{1}&0 \end{bmatrix},
\quad \vec{\gamma} = \begin{bmatrix} 0 & \vec{\sigma} \\ - \vec{\sigma} & 0 \end{bmatrix}, 
\quad \gamma^5 = 
\begin{bmatrix}  -\vec{1} & 0 \\ 0 & \vec{1}\end{bmatrix} \]

The projection operator in this representation then looks like
\begin{subequations}
\begin{equation}
  \label{eq:99}
  P~L = \half ( 1 -\gamma^5)=  \begin{bmatrix}\vec{1}&0\\0&0  \end{bmatrix}
\end{equation}
\begin{equation}
  \label{eq:100}
  P~R = \half (1+\gamma^5)=
  \begin{bmatrix}
    0&0\\0&\vec{1}
  \end{bmatrix}
\end{equation}
\end{subequations}
Thus the left-handed Weyl spinor is the upper part, and the right
handed the lower part of the Dirac spinor. Thus
\begin{equation}
  \label{eq:101}
  P~R u =   \begin{bmatrix}0&0\\0&\vec{1} \end{bmatrix}
  \begin{bmatrix} \Psi~L \\ \Psi~R  \end{bmatrix} =
  \begin{bmatrix}
    0 \\ \Psi~R
  \end{bmatrix}
\end{equation}

\section{The weak interaction}
\label{sec:weak-interaction}

The weak interaction only acts on left-handed particles. Parity
transformations leave spin unchanged, but will turn left-handed
particles into right-handed particles, and so weak interactions must
be parity violating.

It's also worth noting that helicity is only conserved for massless
particles, as a massive particle therewill always be a boost into a
higher velocity frame, which will invert the momentum's
direction. This causes the helicity to flip.

The Dirac Lagrangian has the form
\[ m \bar{\psi} \psi = m (\hcon{\Psi}~L \hcon{\Psi}~R)
\begin{bmatrix}  0&\vec{1}\\\vec{1}&0 \end{bmatrix}
\begin{bmatrix} \Psi~L \\ \Psi~R \end{bmatrix}
= m \qty( \hcon{\Psi}~L \Psi~R + \hcon{\Psi}~R \Psi~L )
\]
in the Chiral representation. The mass terms mix the left and
right-handed states, and so the weak interaction cannot act on massive
particles. 

\section{The Higgs field}
\label{sec:higgs-field}

The solution to this problem is to introduce a new \emph{Higgs} field
which couples left-handed particles to right-handed ones, giving them
an effective mass. The Lagrangian has the form
\begin{equation}
  \label{eq:102}
  \Lag~{higgs} \supset  Y \bar{\psi}~L \vdot \phi \psi~R
\end{equation}
If the vacuum of the system contains a non-zero amount of this field,
so $\ev{\phi} \neq 0$, then we can generate a mass $Y\ev{\phi}$.

\section{Symmetries of the Dirac equation}
\label{sec:symm-dirac-equat}

\subsection{Lorentz transformation}
\label{sec:lorentz-transf}

Under a Lorentz transform, $x'^{\mu} = \Lambda^{\mu}_{\nu} x^{\nu}$,
so
\[ \pd{\mu} \to \pd{\mu}' = [\Lambda^{-1}]_{\mu}^{\nu} \pd{\nu}, \quad \psi(x) \to \psi'(x') = S \psi(x) \]
Now,
\[ (i \gamma^{\mu} \pd{\mu} - m)\psi(x)=0 \to (i \gamma^{\mu}
[\Lambda^{-1}]^{\nu}_{\mu} \pd{\nu} - m) S \psi(x) = 0\]
Pre-multiplying by $S^{-1}$, then
\[ \qty( i S^{-1} \gamma^{\mu} S [\Lambda^{-1}]^{\nu}_{\mu} \pd{\nu}-m) \psi(x) = 0 \]
and so
\[ S^{-1} \gamma^{\mu} S = \Lambda^{\mu}_{\nu} \gamma^{\nu} \]

Now, suppose we have an infintessimal proper transformation,
\[ \Lambda^{\mu}_{\nu} = g^{\mu}_{\nu}+ \omega^{\mu}_{\nu} \]
with the latter component anti-symmetric. Now, writing 
\[ S = 1 + \frac{i}{4} \sigma_{\mu \nu} \omega^{\mu \nu} \]
which is a parameterisation, so
\[ \gamma^{\mu} + \omega^{\mu}_{\nu} \gamma^{\nu} = (1-\frac{i}{4} \sigma_{\alpha \beta} \omega^{\alpha \beta}) \gamma^{\mu} (1+ \frac{i}{4} \sigma_{\sigma \rho} \omega^{\sigma \rho}) \]
Thus
\[ 2 i \omega^{\alpha\beta} \qty(\delta^{\mu}_{\alpha} \gamma_{\beta}
- \delta^{\mu}_{\beta} \gamma_{\alpha}) =
\comm{\gamma^{\mu}}{\sigma_{\alpha \beta}} \omega^{\alpha \beta} \]
(ignoring terms above the order of $\omega^2$), and we eventually reach
\[ \sigma_{\mu \nu} = \frac{i}{2} \comm{\gamma_{\mu}}{\gamma_{\nu}} \]
This is how a fermionic field transforms under a Lorentz boost.

The adjoint field transforms as
\[ \bar{\psi} = \hcon{\psi} \gamma^0 \to \hcon{\psi} \hcon{S} \gamma^0
= \hcon{\psi} \gamma^0 S^{-1} = \bar{\psi} S^{-1} \] since $\hcon{S}
\gamma^0 = \gamma^0 S^{-1}$, and so the quantity $\bar{\psi} \psi$ is
invariant. Since we have an invariant, we must have an associated
current, and so
\begin{equation}
  \label{eq:103}
  j^{\mu} = \bar{\psi} \gamma^{\mu} \psi \to \bar{\psi} S^{-1} \gamma^{\mu} S \psi = \Lambda^{\mu}_{\nu} \bar{\psi} \gamma^{\nu} \psi
\end{equation}

\begin{subequations}
  \begin{description}
  \item[scalars] \[ \bar{\psi} \psi \to \bar{\psi} \psi \]
  \item[pseudoscalars] \[\bar{\psi} \gamma^5 \psi \to \det(\Lambda) \bar{\psi} \gamma^5 \psi \]
  \item[vectors] \[ \bar{\psi} \gamma^{\mu} \psi \to \Lambda^{\mu}_{\nu} \bar{\psi} \gamma^{\nu} \psi \]
  \item[axial vectors] \[ \bar{\psi} \gamma^{\mu} \gamma^5 \psi \to \det(\Lambda) \Lambda^{\mu}_{\nu} \bar{\psi} \gamma^{\nu} \gamma^5 \psi \]
  \item[tensors] \[ \bar{\psi} \sigma^{\mu \nu} \gamma^5 \psi \to \Lambda^{\mu}_{\alpha} \Lambda^{\nu}_{\beta} \bar{\psi} \sigma^{\alpha \beta} \gamma^5 \psi \]
  \end{description}
\end{subequations}

\subsection{Charge conjugation, parity, and time-reversal}
\label{sec:charge-conj-parity}

Parity: As $t\to t$, $\vec{x} \to - \vec{x}$ \\
Charge conjugation: As $\psi \to \psi~c \equiv C \tr{\bar{\psi}}$ \\
Time reversal: As $t \to - t$, $\vec{x} \to \vec{x}$, and so $\psi(t,
\vec{x}) \to \psi~T(-t, \vec{x}) = T \psi^{*}(t, \vec{x})$.

The corresponding transformations are
\begin{subequations}
\begin{align}
  \label{eq:104}
  C: C \tr(\bar{\psi})(t,\vec{x}) &= i \gamma^2 \gamma^0 \tr(\bar{\psi})(t,\vec{x}) \\
P: P \psi(t, \vec{x}) &= \gamma^0 \psi(t, \vec{x}) \\
T: T \psi^{*}(t, \vec{x}) &= i \gamma^1 \gamma^3 \psi^{*}(t, \vec{x})
\end{align}
\end{subequations}

\begin{bigderiv*}
  \begin{align*}
    H & = \int T^{00} \dd[3]{x} = \int i \hOp{\psi} \pu{0} \Op{\psi} \dd[3]{x} \\
&= \int \dd[3]{x} \int \nm{p} \frac{\dd[3]{q}}{2(2 \pi)^3} \sum_{s,s'} \qty( \hOp{a}^{(s)} (\vec{p}) \hcon{u}^{(s)}(p) e^{ip\vdot x} +  \hOp{b}^{(s)} (\vec{p}) \hcon{v}^{(s)}(p) e^{-ip\vdot x} ) \\ 
& \qquad \qquad \qquad \qquad \qquad \times
 \qty( \hOp{a}^{(s')} (\vec{q}) \hcon{u}^{(s')}(p) e^{iq\vdot x} +  \hOp{b}^{(s')} (\vec{p}) \hcon{v}^{(s')}(p) e^{-iq\vdot x} ) 
\intertext{There are three orthogonality relations which simplify this,
\[ \hcon{u}^{(r)} v^{(s)} = 2E \delta^{rs} \qquad \hcon{u}^{(r)} u^{(s)} = 2 E \delta^{rs}  \qquad \hcon{u}^{(r)}v^{(s)} = 0 \]
}
H&= \int \nm{p} E(\vec{p}) \sum_s \qty(\hOp{a}^{(s)}(\vec{p}) \Op{a}^{(s)}(\vec{p}) - \Op{b}^{(s)}(\vec{p}) \hOp{b}^{(s)}(\vec{p}) ) \\
&= \int \nm{p} E(\vec{p}) \sum_s
 \qty(\hOp{a}^{(s)}(\vec{p}) \Op{a}^{(s)}(\vec{p}) - \hOp{b}^{(s)}(\vec{p}) \Op{b}^{(s)}(\vec{p}) - \comm{\Op{b}^{(s)}(\vec{p})}{\hOp{b}^{(s)}(\vec{p})})
  \end{align*}
We can't postulate a commutation relation for the creation and annihilation operators, as even after subtracting an infinite complex number the antiparticles still give a negative energy contribution. So the energy could always be reduced by adding more antiparticles, making an unstable system. We need to make their contributions positive, so must turn to anticommutation relations.
\caption{The derivation of the Hamiltonian for the free Dirac field.}
\label{der:dirac-hamiltonian}
\end{bigderiv*}


\section{Second quantisation}
\label{sec:second-quantisation-1}

Thus far the Dirac equation has only been framed in terms of quantum
mechanics. To transition to QFT we need to convert the fields to
operators. We proceed as for the scalar field.
\begin{align*}
  \Op{\psi}(x) &= \int \nm{p} \sum_s \big( \Op{a}^{(s)}(\vec{p}) u^{(s)}(p) e^{-ip\vdot x} \\ & \qquad \qquad \qquad+ \hOp{b}^{(s)}(\vec{p}) v^{(s)}(p) e^{ip\vdot x} \big)
\end{align*}
From here onwards the hat notation for operators is omitted for neatness.

\section{The energy momentum tensor}
\label{sec:energy-moment-tens-1}

From Noether's theorem the energy momentum tensor is
\begin{equation}
  \label{eq:105}
  T^{\mu \nu} = \pdv{\Lag}{(\pd{\mu}\psi)} \pu{\nu} \psi + \pu{\nu} \bar{\psi} \pdv{\Lag}{(\pd{\mu}\bar{\psi})} - g^{\mu \nu} \Lag = \bar{\psi} i \gamma^{\mu} \pu{\nu} \psi
\end{equation}
where the Dirac equation was used to remove the last term. Thus the Hamiltonian is
\begin{equation}
  \label{eq:106}
  T^{00} = i \hcon{\psi} \pu{0} \psi \iff H = \int i \hcon{\psi} \pu{0} \psi \dd[3]{x}
\end{equation}

\section{Anticommutation}
\label{sec:anticommutation}

We can postulate some anticommutation relations for the Dirac field,
\begin{subequations}
  \begin{align}
    \label{eq:107}
    \acomm{\Op{a}^{(r)}(\vec{k})}{\hOp{a}^{(s)}(\vec{p})} &  =\acomm{\Op{a}^{(r)}(\vec{k})}{\hOp{a}^{(s)}(\vec{p})} \nonumber\\
&= \delta^{rs} (2 \pi)^3 2 E(\vec{k}) \delta^3(\vec{k}-\vec{p}) \\
 \acomm{\Op{a}^{(r)}(\vec{k})}{\Op{a}^{(s)}(\vec{p})} & =  \acomm{\Op{a}^{(r)}(\vec{k})}{\Op{b}^{(s)}(\vec{p})} \nonumber \\= \cdots &= 0
  \end{align}
\end{subequations}
and then the anti-commutation relations for the fields are
\begin{subequations}
  \begin{align}
    \label{eq:108}
    \acomm{\Op{\phi}^{(r)}(\vec{x}, t)}{\Op{\pi}^{(s)}(\vec{y},t)} &= i \delta^{rs} \delta^3(\vec{x}-\vec{y}) \\
\acomm{\Op{\psi}^{(r)}(\vec{x},t)}{\Op{\psi}^{(s)}(\vec{y},t)} &= \acomm{\Op{\pi}^{(r)}(\vec{x},t)}{\Op{\pi}^{(s)}(\vec{y},t)} = 0
  \end{align}
\end{subequations}
This means the state of two fermions is antisymmetric under
interchange, which is the origin of the Fermi-Dirac statistics, and
the Pauli exclusion principle.

\section{The Dirac propagator}
\label{sec:dirac-propagator}

The Greens function for the Dirac equation satisfies
\[ S~F(x-y) \equiv G_2(x,y) \]
so
\begin{equation}
  \label{eq:109}
  (i \ds - m) S~F (x-y) = i \delta^{(4)}(x-y)
\end{equation}
and then writing
\begin{equation}
  \label{eq:110}
  S~F(x-y) = \int \nm{p} \tilde{S}~F(p) \exp(-ip \vdot(x-y) )
\end{equation}
premultiplying this by $(i \ds -m)$, and remembering that
\[ \int \nm{p} e^{-ip \dot (x-y)} = \delta^{(4)}(x-y) \]
then
\begin{equation}
  \label{eq:111}
  \tilde{S}~F(p) = \frac{i}{\ps -m } = \frac{i(\ps + m)}{p^2 - m^2}
\end{equation}

Just as with scalar fields the propagator is a time-ordered product of
fields, but we must be careful with the signs. For a fermion field
\begin{equation}
  \label{eq:112}
  \tOrd \psi^{(r)}(x) \bar{\psi}^{(s)}(y) = 
  \begin{cases}
    \psi^{(r)}(x) \bar{\psi}^{(s)}(y) & \text{for $x^0 > y^0$} \\
- \bar{\psi}^{(s)}(y) \psi^{(r)}(x) & \text{for $x^0 < y^0$}
  \end{cases}
\end{equation}
And with this definition,
\begin{fequation}[Dirac propagator]
  \label{eq:113}
  S~F^{rs}(x-y) = \bra{0} \tOrd \psi^{(r)}(x) \bar{\psi}^{(s)}(y) \ket{0}
\end{fequation}


%%% Local Variables: 
%%% mode: latex
%%% TeX-master: "../project"
%%% End: 
