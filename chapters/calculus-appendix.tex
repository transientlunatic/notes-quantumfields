
\section{Dirac Delta Function}
\label{sec:dirac-delta-function}

The dirac delta is a function which can be (non-rigorously) defined 
\begin{namedequation}
  \begin{equation}
    \label{eq:11}
    \delta(x-y) = 
    \begin{cases}
      \infty &\text{\quad for} x = y \\
      0 &\text{\quad for} x \neq y
    \end{cases}
  \end{equation}
\end{namedequation}
such that its integral is $1$:
\[ \int \delta(x-y) \dd{x} = 1 \]

It can be visualised as a limit of the Gaussian distribution,
\[ \delta(x-y) = \lim_{a \to 0} \frac{1}{ a \sqrt{\pi} } \exp( -
\frac{(x-y)^2}{a^2} \]
which is equivalent to
\[ \delta(x-y) = \frac{1}{2 \pi} \int_{- \infty}^{\infty} e^{i(x-y)k}
\dd{k} \]

\subsection{Useful Properties}
\label{sec:useful-properties}

Under integration the delta function has the property of picking out a
value of a function:

\begin{equation}
  \label{eq:12}
  \int \delta(x-y) f(x) \dd{x} = f(y)
\end{equation}

There are a number of other properties:
\begin{subequations}
\begin{align}
  \delta(ax) &= \frac{\delta(x)}{\abs{a}} \\
  \delta( g(x) ) &= \sum_i \frac{\delta(x - x_i)}{\abs{g^{\prime} (x_i)}} \\
  \delta(x^2 - a^2) &= \frac{1}{2 \abs{a}} ( \delta(x-a) + \delta(x+a) ) \\
  (x - y) \pdv{y} \delta(x-y) &= \delta(x - y)
\end{align}
\end{subequations}

There also exists a multi-dimensional form,
\begin{equation}
  \label{eq:13}
  \delta^4(x-y) = \delta(x^0 - y^0) \delta(x^1 - y^1) \delta(x^2 - y^2) \delta(x^3 - y^3)
\end{equation}

\section{Green's Functions}
\label{sec:greens-functions}

Consider a field which satisfies a differential equation of the form
\begin{equation}
  \label{eq:14}
  \Op{D} \phi(x) = \rho(x)
\end{equation}
where $\Op{D}$ is some differential operator.

For example, Poisson's equation is
\[ \vec{\nabla}^2 \phi(x) = \rho(x) \] Let the function $G(x,y)$ be a
solution of the same equation, with a point source at $x=y$,  so

\begin{align*}
  \Op{D} G(x,y) &= \delta(x-y) \\
\Op{D} \int G(x,y) \rho(y) \dd{y} &= \int \delta(x-y) \rho(y) \dd{y} = \rho(x)
\end{align*}
That is,
\[ \phi(x) = \int G(x,y) \rho(y) \dd{y} \]
is a solution to the original equation.

The function $G(x,y)$ is a Green's function. Green's functions convert
the problem of a differential equation into the problem of evaluating
an integral.

%%% Local Variables: 
%%% mode: latex
%%% TeX-master: "../project"
%%% End: 
