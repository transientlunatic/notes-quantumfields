The classical theory of electromagnetism is provided by the Maxwell
equations, with the Faraday tensor
\begin{equation}
  \label{eq:114}
  \tensor{F}{^{\mu \nu}} =
  \begin{bmatrix}
    0   & -E_1 &-E_2 & -E_3 \\ E_1 & 0 & -B_3 & B_2 \\
    E_2 & B_3 & 0 & -B_1 \\ E_3 & -B_2 & B_1 & 0
  \end{bmatrix}
\end{equation}
which is antisymmetric in its arguments, and where the four-current is
defined 
\begin{equation}
  \label{eq:115}
\pd{\mu} F^{\mu \nu} = j^{\nu}
\end{equation}
In terms of the four-potential, $A^\mu$,
\begin{equation}
  \label{eq:116}
F^{\mu \nu} = \pu{\mu} A^{\nu} - \pu{\nu} A^{\mu}
\end{equation}
so it follows that
\[
\pd{\mu} F^{\mu \nu} = \pd{\mu} \pu{\mu} A^{\nu} - \pd{\mu} \pu{\nu} A^{\mu}
\]
By making a gauge transformation on the potential $\ten{F}$ will still
be left unchanged, so making the gauge transform
\[ A^{\mu} \to A^{\mu} + \lambda \pu{\mu} \phi \]
it is possible to choose a $\lambda$ which has
\[ \pd{\mu} A^{\mu} = 0 \]
which is the Lorenz gauge, where \[ \partial^2 A^{\mu} = j^{\mu} \]
If there is no source then \[ \partial^2 A^{\nu} = 0 \] which has solutions
\[ A^{\mu} = \epsilon^{\mu} e^{i q \vdot x}, \qquad q^2 = 0 \]
The Lorenz condition implies that $q_{\mu} \epsilon^{\mu} = 0$, so
$\epsilon^{\mu}$ has only three degrees of freedom: two which are
transverse, and one which is longitudinal. There is still some freedom
to change $A^{\mu}$, even after the Lorenz gauge choice, so
\[ A^{\mu} \to A^{\mu} + \du{\mu}\chi\]
as long as $\partial^2 \chi = 0$. Usually $\chi$ is chosen so
$\nabla \vdot \vec{A} = 0$, which is the Coulomb gauge, which implies
$\vec{q} \vdot \vec{\epsilon}=0$ which are two transverse polarisation
state.

\section{The free photon field}
\label{sec:free-photon-field}

We want a Lagrangian which will give $\pd{\mu} F^{\mu \nu}=0$ when
there are no sources, then
\begin{equation}
  \label{eq:117}
\Lag = - \frac{1}{4} F_{\mu \nu} F^{\mu \nu}
\end{equation}
Then, since $F_{\mu \nu}= \pd{\mu} A_{\nu} - \pd{\nu} A_{\mu}$, so
\[ \pdv{F_{\mu \nu}}{A_\rho}=0 \implies \pdv{\Lag}{A_{\rho}} = 0 \]
Thus
\begin{align*}
  \pdv{F_{\mu \nu}}{(\partial_{\sigma} A_{\rho})} &= g_{\mu}^{\sigma}g_{\nu}^{\rho}-g_{\mu}^{\rho}g_{\nu}^{\sigma} \\
\implies \pdv{\Lag}{(\partial_{\sigma} A_{\rho})} &= - \half F^{\mu \nu} \pdv{F_{\mu \nu}}{(\partial_{\sigma} A_{\rho})} \\ &= - \half F^{\mu \nu} ( g_{\mu}^{\sigma}g_{\nu}^{\rho}-g_{\mu}^{\rho}g_{\nu}^{\sigma}) \\
&= - \half (F^{\sigma \rho} - F^{\rho \sigma}) = - F^{\sigma \rho}
\end{align*}

The photon is a vector boson, so in terms of creation and annihilation
vectors,
\begin{equation}
  \label{eq:118}
\Op{A}_{\mu}(x) = \int \nm{k} \qty( \Op{a}_{\mu}(\vec{k}) e^{-ik \vdot x} + \hOp{a}_{\mu}(\vec{k}) e^{ik \vdot x})
\end{equation}
Since $\pdv*{\Lag}{A_{\mu}}=-F^{0 \mu}$ we need to be careful with the
canonically conjugate states, so $\pi^0=0$, and $\pi^i = -F^{0i}=E_i$,
the electric field.

The Hamiltonian is
\begin{equation}
  \label{eq:119}
\Op{H} = \half \int \nm{\vec{k}} E(\vec{k})  \qty( - \Op{a}_{\mu}(\vec{k}) e^{-ik \vdot x} - \hOp{a}_{\mu}(\vec{k}) e^{ik \vdot x})
\end{equation}
The commutation relations are 
\begin{subequations}
  \begin{align}
    \comm{\Op{a}_{\mu}(\vec{k})}{\hOp{a}_{\nu}(\vec{p})} &= - g_{\mu \nu} (2 \pi)^3 2 E(\vec{k}) \delta^3(\vec{k}-\vec{p}) \\
\comm{\Op{a}_{\mu}(\vec{k})}{\Op{a}_{\nu}(\vec{p})} &= \comm{\hOp{a}_{\mu}(\vec{k})}{\hOp{a}_{\nu}(\vec{p})} = 0
  \end{align}
\end{subequations}

We now impose gauge conditions on the Lagrangian, so
\begin{equation}
  \label{eq:120}
\Lag = - \frac{1}{4} F_{\mu \nu} F^{\mu \nu} - \frac{\lambda}{2} (\pd{\mu}A^{\mu})^2
\end{equation}
for $\lambda$ the Lagrange multiplier which forces the gauge
condition, $\pd{\mu} A^{\mu}=0$. For general $\lambda$ the propagator
is
\begin{align*}
  D~F(x-y) &= \bra{0} \tOrd A_{\mu}(x) A_{\nu}(y) \ket{0} \\
&= i \int \frac{\dd[4]{k}}{(2 \pi)^4} \qty(-g^{\mu \nu} + (1-\frac{1}{\lambda}) \frac{k^{\mu}k^{\nu}}{k^2} ) \frac{e^{-ik \vdot(x-y)}}{k^2+i \epsilon}
\end{align*}
In the Feynman gauge, with $\lambda=1$,
\begin{equation}
  \label{eq:121}
\bra{0} \tOrd A_{\mu}(x) A_{\nu}(y) \ket{0} = \int \frac{\dd[4]{k}}{(2 \pi)^4} \frac{- i g_{\mu \nu}}{k^2+i \epsilon} e^{-ik \cdot (x-y)}
\end{equation}

\section{Dirac equation}
\label{sec:dirac-equation-1}

There is one remaining symmetry of the Dirac lagrnagian which has not
yet been investigated: a phase-shift of the electron field,
$\psi \to e^{i \theta} \psi$, and
$\bar{\psi} \to \bar{\psi} e^{-i \theta}$. Then
\[ \Lag = \bar{\psi} (i \gamma^{\mu} \pd{\mu}-m) \psi \to
\bar{\psi}e^{-i \theta} (i \gamma^{\mu} \pd{\mu} -m) e^{i \theta} \psi
= \Lag \]
The Lagrangian doesn't change, and so the physics stays the same. The
conserved current associated with the symmetry via Noether's theorem
is $j^{\mu}=e \bar{\psi} \gamma^{\mu} \psi$. This is a global U(1)
symmetry. 

If the transformation is local, depending on the point in spacetime,
then
\[ \psi \to e^{i \theta(x)} \psi, \qquad \bar{\psi} \to \bar{\psi} e^{-i \theta(x)} \]
Thus
\[ \Lag \to \bar{\psi} e^{-i \theta(x)} (i \gamma^{\mu} \pd{\mu} - m)
e^{i \theta(x)} \psi = \Lag - \bar{\psi} \gamma^{\mu}(\pd{\mu}
\gamma^{\mu}(\pd{\mu} \theta(x)) \psi \]
The free Dirac Lagrangian is no longer invariant; if we really want
this to be a symmetry of the theory something extra is
required. Postulating $A^{\mu}$ as a new field which couples to the
electron according to
\[ \Lag = \bar{\psi} (i \gamma^{\mu} \pd{\mu} - e \gamma_{\mu} A^{\mu} - m) \psi \]
for $e$ the charge on the electron, which has the form of the covariant derivative,
\begin{fequation}[Covariant derivative]
  \label{eq:122}
D^{\mu} \equiv \pu{\mu}+ie A^{\mu}
\end{fequation}

Now
\begin{align*}
  \Lag &= \bar{\psi} (i \gamma^{\mu} D_{\mu} - m)\psi \\
& \to \bar{\psi} \qty( i \gamma^{\mu} e^{-i \theta(x)} D'_{\mu}e^{i \theta(x)} - m) \psi
\end{align*}
Thus, to preserve the Lagrangian we need $D_{\mu}$ to transform too,
\[ D_{\mu} \to D_{\mu}' = e^{i \theta(x)} D_{\mu} e^{-i \theta(x)} \]
then
\[ \pd{\mu}+ie A_{\mu}' = e^{i \theta(x)}( \pd{\mu}+ie A_{\mu}) e^{-i \theta(x)} = \pd{\mu} - i \pd{\mu}\theta(x) + i e A_{\mu} \]
therefore we need
\[ A_{\mu} \to A_{\mu}' = A_{\mu} - \frac{1}{e} \pd{\mu} \theta(x) \]
The gauge transform for the classical photon. 

Coupling the electron to a photon makes the theory locally U(1)
symmetric, so.

\begin{subequations}
  \begin{equation}
\label{eq:123}
    \Lag~{QED} = - \frac{1}{4} F_{\mu \nu} F^{\mu \nu} + \bar{\psi} ( i \gamma^{\mu} \pd{\mu} - m) \psi - e \bar{\psi} \gamma^{\mu} \psi A_{\mu} 
  \end{equation}
\begin{equation} \label{eq:124}
  ( i \gamma^{\mu} \pd{\mu} - m - e \gamma^{\mu} A_{\mu}) \psi = 0 
\end{equation}
\begin{equation}
\label{eq:125}
\partial^2 A^{\mu} &= e \bar{\psi} \gamma^{\mu} \psi
  \end{equation}
\end{subequations}


%%% Local Variables:
%%% mode: latex
%%% TeX-master: "../project"
%%% End:
